% Intended LaTeX compiler: lualatex
\documentclass[version=last,draft=true,paper=A4,portrait,twoside=true,twocolumn=false,headinclude=false,footinclude=false,fontsize=12,BCOR=20mm,DIV=calc,pagesize=auto,titlepage=firstiscover,mpinclude=true,open=right,chapterprefix=true,numbers=autoendperiod,headsepline=false,parskip=false]{scrbook}
\usepackage{graphicx}
\usepackage{grffile}
\usepackage{longtable}
\usepackage{wrapfig}
\usepackage{rotating}
\usepackage[normalem]{ulem}
\usepackage{amsmath}
\usepackage{textcomp}
\usepackage{amssymb}
\usepackage{capt-of}
\usepackage{hyperref}
\usepackage{polyglossia}
\usepackage[autostyle=true,english=american,french=guillemets,thresholdtype=words,threshold=3]{csquotes}
\AtBeginEnvironment{quote}{\itshape}
\AtBeginEnvironment{foreigndisplayquote}{\itshape}
\usepackage{amsmath}
\usepackage{amsthm}
\usepackage{amssymb}
\usepackage{centernot}
\usepackage{hyperref}
\hypersetup{colorlinks,urlcolor=bibleblue,linkcolor=bibleblue,citecolor=bibleblue,filecolor=black}
\usepackage{balance}
\usepackage{array}
\usepackage{tabularx}
\usepackage{booktabs}
\usepackage[most]{tcolorbox}
\usepackage[french]{fmtcount}
\fmtcountsetoptions{french=france}
\usepackage[singlespacing]{setspace}
\usepackage[super]{nth}
\usepackage{ragged2e}
\usepackage[all]{nowidow}
\usepackage{enumitem}
\usepackage{adforn}
\usepackage{float}
\usepackage{titling}
\usepackage{xcolor}
\usepackage{graphicx}
\graphicspath{ {/home/sync0/Dropbox/paris_1/} }
\usepackage{tikz}
\usetikzlibrary{positioning}
\tikzset{main node/.style={circle,fill=gray!45,draw,minimum size=0.5cm,inner sep=0pt},}
\usepackage{lipsum}
\usepackage[backend=biber,bibstyle=authortitle,citestyle=verbose-trad1,hyperref=auto,doi=false,isbn=false,url=false]{biblatex}
\addbibresource{~/Documents/pdfs/bibliography.bib}
\urlstyle{sf}
\renewcommand{\subtitlepunct}{\addcolon\addspace}
\renewbibmacro*{date}{\printdate\iffieldundef{origyear}{}{\setunit*{\addspace}\printtext[parens]{\printorigdate}}}
\usepackage{scrlayer-scrpage}
\pagestyle{scrheadings}
\clearscrheadfoot
\automark[chapter]{part}
\cehead{\MakeLowercase{\thetitle}}
\cohead{\MakeLowercase{\headmark}}
\ohead{\pagemark}
\usepackage{fontspec}
\usepackage{unicode-math}
\usepackage[oldstyle]{libertine}
\defaultfontfeatures{Scale=MatchLowercase}
\setmonofont{Source Code Pro}
\setmathfont[Scale=MatchUppercase]{libertinusmath-regular.otf}
\newfontfamily{\titlefamily}[Scale=2]{Linux Biolinum O}
\newfontfamily{\sbfseries}[UprightFont={* Semibold}]{Linux Libertine O}
\newcommand\hugetitle{\fontsize{45}{50}\selectfont}
\newcommand\HUGE{\fontsize{40}{40}\selectfont}
\newcommand\hugechapter{\fontsize{30}{35}\selectfont}
\AfterTOCHead{\singlespacing}
\setkomafont{labelinglabel}{\normalsize\itshape}
\setkomafont{minisec}{\usekomafont{subsection}}
\setkomafont{pagehead}{\normalsize\mdseries\scshape}
\setkomafont{pagenumber}{\normalsize\rmfamily\upshape}
\setkomafont{sectioning}{\rmfamily\mdseries}
\setkomafont{caption}{\small}
\setkomafont{captionlabel}{\sffamily\mdseries\scshape\lowercase}
\setkomafont{chapter}{\hugechapter\rmfamily}
\renewcommand{\raggedchapter}{\centering}
\renewcommand*\chapterformat{\thechapter\autodot\par\enskip}
\RedeclareSectionCommand[afterskip=6\baselineskip]{chapter}
\setkomafont{section}{\Large\scshape\lowercase}
\setkomafont{subsection}{\large}
\setkomafont{subsubsection}{\large\itshape}
\AtBeginDocument{\renewcaptionname{english}\contentsname{Contents}}
\addtokomafont{chapterentry}{\mdseries\scshape\lowercase}
\setkomafont{chapterentrypagenumber}{\normalsize}
\usepackage{tocstyle}
\settocfeature{raggedhook}{\raggedright}
\selecttocstyleoption{tocgraduated}
\usetocstyle{nopagecolumn}
\newtcolorbox{modified}[1][]{grow to right by=0mm,grow to left by=-1em,boxrule=1pt,boxsep=0pt,breakable,enhanced jigsaw,borderline west={0pt}{0pt}{lightgrey},lower separated=false,arc=00mm,colframe=white, #1}
\newtcolorbox{note}[2][]{grow to right by=0mm,grow to left by=-1em,boxrule=0pt,boxsep=0pt,opacityback=0.0,breakable,parbox=false,enhanced jigsaw,borderline west={4pt}{0pt}{lightgrey},title={#2},coltitle={black},fonttitle={\sffamily},attach title to upper={},halign title=right,after title={\smallskip\par}#1}
\newtcolorbox{question}[2][]{grow to right by=0mm,grow to left by=-1em,boxrule=0pt,boxsep=0pt,opacityback=0.0,breakable,parbox=false,enhanced jigsaw,borderline west={4pt}{0pt}{darkgrey},title={#2},coltitle={black},fonttitle={\sffamily},attach title to upper={},halign title=right,after title={\smallskip\par}#1}
\newtcolorbox{definition}[3][]{grow to right by=0mm,grow to left by=-1em,boxrule=0pt,boxsep=0pt,opacityback=0.0,breakable,enhanced jigsaw,borderline west={4pt}{0pt}{midgrey},title={#2},coltitle={black},fonttitle={\sffamily\bfseries},fontupper={\normalsize},fontlower={\itshape},lower separated=false,attach title to upper={},after title={\hspace{1em}{\rmfamily\mdseries\itshape #3}\par}#1}
\renewcommand*\labelitemi{\adforn{33}}
\renewcommand*\labelitemii{\adforn{73}}
\renewcommand*\labelitemiii{\adforn{73}}
\renewcommand*\labelitemiv{\adforn{73}}
\definecolor{bibleblue}{HTML}{00339a}
\definecolor{whitegrey}{HTML}{f7f7f7}
\definecolor{lightgrey}{HTML}{cccccc}
\definecolor{midgrey}{HTML}{969696}
\definecolor{darkgrey}{HTML}{636363}
\definecolor{blackgrey}{HTML}{252525}
\newcommand{\notimplies}{\centernot\implies}
\setcounter{secnumdepth}{3}
\setcounter{tocdepth}{1}
\setmainlanguage{english}
\setotherlanguages{french,italian,spanish}
\MakeOuterQuote{"}
\MakeForeignQuote{french}{«}{»}
\usepackage[protrusion=true,tracking=true]{microtype}
\author{Carlos Alberto Rivera Carreño}
\date{}
\title{}
\hypersetup{
 pdfauthor={Carlos Alberto Rivera Carreño},
 pdftitle={},
 pdfkeywords={},
 pdfsubject={},
 pdfcreator={Emacs 26.1 (Org mode 9.2.2)}, 
 pdflang={English}}
\begin{document}

%   This program is free software: you can redistribute it and/or modify
%   it under the terms of the GNU General Public License as published by
%   the Free Software Foundation, either version 3 of the License, or
%   (at your option) any later version.

%   This program is distributed in the hope that it will be useful,
%   but WITHOUT ANY WARRANTY; without even the implied warranty of
%   MERCHANTABILITY or FITNESS FOR A PARTICULAR PURPOSE. See the
%   GNU General Public License for more details.

%   You should have received a copy of the GNU General Public License
%   along with this program. If not, see <http://www.gnu.org/licenses/>.
\begin{titlepage}
 \centering
\begin{french}
 {\large \textsc{université paris i panthéon sorbonne} \par}
  \vspace*{0.01\textheight}
 {\large \textsc{ufr} 02 : Sciences économiques  \par}
  \vspace*{0.01\textheight}
 {\large Master 2 : Économie et sciences humaines \par}
  \vspace*{0.01\textheight}
 {\large 2018--2019 \par}
\end{french}
  \vspace*{0.3\textheight}
 {\huge \textsc{vers la machine à gouverner}  \par}
  \vspace*{0.02\textheight}
 {\Large Herbert Simon and the Impossibility of a Democratic Computer \par}
\vfill
\begin{french}
 {\large Présenté et sountenu par : \par}
\end{french}
 {\Large Carlos Alberto Rivera Carreño \par}
  \vspace*{0.05\textheight}
\begin{french}
 {\large Directeur de mémoire : \par}
\end{french}
 {\Large Jean-Sébastien Lenfant \par}
\end{titlepage}

\pagestyle{empty}

\begin{french}
L'Université Paris 1 Panthéon Sorbonne n'entend donner aucune approbation,
ni désapprobation aux opinions émises dans ce mémoire ; elle doivent être
considérées comme propres à leur auteur. 
\end{french}
\vfill

\newpage
\vspace*{\fill}
\noindent
\includegraphics[height=1.5cm]{gpl3.png}\par
\vspace{1\baselineskip}
This text is free: you can redistribute it and/or modify it
under the terms of the \textsc{gnu} General Public License as published by
the Free Software Foundation, either version 3 of the License or any later
version.

This text is distributed in the hope that it will be useful, but \textbf{without
any warranty}; without even the implied warranty of \textbf{merchantability or 
fitness for a particular purpose}. See the \textsc{gnu} General 
Public License for more details.

You should have received a copy of the \textsc{gnu} General Public License along
with this text. If not, see \url{http://www.gnu.org/licenses/}.

\vspace{1\baselineskip}
\noindent
Copyright \textcopyright \textsc{sync0} 2018. 

% \newpage 
\newpage\null\newpage

\begin{FlushRight}
\begin{spanish}
\textit{Al padre Camilo Torres.}
\end{spanish}
\end{FlushRight}

\newpage
\tableofcontents 
\frontmatter
\pagestyle{plain}
\chapter*{Abbreviations} 
\section*{Archival Sources}

When citing archival materials, I indicate their location within the
following archives:

\begin{labeling}[:]{HSMC} 
\item[HSCM] Herbert Simon Papers, Carnegie Mellon University Archives.
\item[ENPC] École nationale des ponts et chaussées.
\item[BNF] Bibliothèque nationale de France. 
\end{labeling}

I accessed them through their respective on-line repositories.

\begin{labeling}[:]{HSMC} 
\item[HSCM] \href{https://digitalcollections.library.cmu.edu/portal/index.jsp}{Carnegie Mellon University Libraries Digital Collections.} 
\item[ENPC] \href{https://patrimoine.enpc.fr/}{Bibliothèque numérique patrimoniale des ponts et chaussées}.
\item[BNF] \href{https://gallica.bnf.fr/}{Gallica.}
\end{labeling}

Only the pdf version of this document includes hyperlinks to the source
files, when available.

\chapter{Acknowledgements} 
\lipsum
% \markboth{\MakeMarkcase{Preface}}{\MakeMarkcase{Préface}}
\chapter{Note aux lecteurs francophones} 
Le fait que ce mémoire fut préparé au sein d'une université française
m'oblige moralement à \ldots aux lecteurs francophones. 

Ce mémoire traite de la relation entre l'histoire du concept de travail,
l'histoire de l'orinateur, l'histoire de l'intelligence, et la pensée de
Herbert Simon. J'essai de replacer l'histoire de travail au sein des
questions sur  l'application des analogies entre les sciences sociales et
les sciences naturelles.  

Dans le premier chapitre je raconte \ldots 

Dans le deuxième chapitre je raconte \ldots 

\lipsum

\chapter[Preface]{Preface:\\ Science and History} 
% \markboth{\MakeMarkcase{Preface}}{\MakeMarkcase{Préface}}
% \mainmatter

\lipsum
\mainmatter
\pagestyle{scrheadings}
\chapter{Introduction: Economy of the Body; Economy of the Machine}
\label{sec:orgfd1c8b9}
\section{Does Labor Exist?}
\label{sec:org2c135c3}

\section{Society and Nature}
\label{sec:orgc1bf313}
\chapter{Did Adam Smith Invent the Computer?}
\label{sec:orgd9677b7}
The introduction to this section describes the computer as a technical
object. The idea is to questoin the belief that the way to define a
computer is to think of it in terms of its components. 

A story that although was present for a long time, seems today relegated to
the confines of a few books on the history of computing. 

A document from the imprimerie Firmin Didot describes the process.

Test: this is an article from diderot encyclopedia:

\section{De Prony's Tables and Human Computers}
\label{sec:org85e6235}
This section describes de Prony' project in a holistic way. 

The documents used in this section are:

\subsection{Short historical context}
\label{sec:org8c9dc0a}
Describe the way de Prony's project was interpreted in the context of the
French revolution. This is how is done 
\subsection{Short technical description}
\label{sec:orgc730024}
What was de Prony actually doing
\section{Can Machine Labor Replace Human Labor?}
\label{sec:org46af764}
This section describes the reading that Charles Babbage makes of de Prony.\footcite{babbage1832}

\section{Herbert Simon Reads Babbage}
\label{sec:orge775bf6}
This section discusses Herbert Simon's reading of Charles Babbage. If
possible, we should look for Simon's reading of de Prony's project. After
all, his article from 1958 discusses Babbage's citing of the project in his
book on European manufactures.

\chapter{UnDemocratic Machines}
\label{sec:org61c0b09}
\section{Herbert Simon: Computer as Mind: Mind as Computer}
\label{sec:orgb8f98e0}

Simon sought a synthesis in the form of a mathematical,
behavioral-functional, and problem centered social science that would bring
together not just choice and control but also theory and practice, research
and reform

The next step towards that synthesis was a new model of man that would
replace the homo oeconomicus of the sciences of choice and the homo
administrativus of the sciences of control. This new model of man is what
Heyck terms homo adaptivus: the bounded, rational, limited but capable,
problem-solver, with problem-solving understood as a process of adaptation
to the environment. 

In Simon's new model of man he synthesized his previous work on
administrative decision-making, contemporary cybernetics and sevomechanism
theory, and the work of Gestalt psychology of problem-solving and
productive thinking.\footcite[p. 185, par. 2]{heyck2005} 

From



\section{Simon on Organizations}
\label{sec:org4be2b9f}
\section{Could Computers Be Democratic?}
\label{sec:org0e29f3e}
\chapter{Conclusion}
\label{sec:org39d6f90}
\lipsum

\backmatter
\chapter{Conclusion} 
\lipsum
\chapter{References} 
\printbibliography[heading=none]
\end{document}