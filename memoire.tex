% Intended LaTeX compiler: lualatex
\documentclass[version=last,draft=true,paper=A4,portrait,twoside=true,twocolumn=false,headinclude=false,footinclude=false,fontsize=11,BCOR=20mm,DIV=calc,pagesize=auto,titlepage=firstiscover,mpinclude=true,open=right,chapterprefix=true,numbers=autoendperiod,headsepline=false,parskip=false]{scrbook}
\usepackage{graphicx}
\usepackage{grffile}
\usepackage{longtable}
\usepackage{wrapfig}
\usepackage{rotating}
\usepackage[normalem]{ulem}
\usepackage{amsmath}
\usepackage{textcomp}
\usepackage{amssymb}
\usepackage{capt-of}
\usepackage{hyperref}
\usepackage{polyglossia}
\setmainlanguage{english}
\setotherlanguages{french,italian}
\usepackage[autostyle=true,english=american,french=guillemets,thresholdtype=words,threshold=3]{csquotes}
\MakeOuterQuote{"}
\MakeForeignQuote{french}{«}{»}
\AtBeginEnvironment{quote}{\itshape}
\usepackage{amsmath}
\usepackage{amsthm}
\usepackage{amssymb}
\usepackage{centernot}
\usepackage{hyperref}
\hypersetup{colorlinks,urlcolor=bibleblue,linkcolor=bibleblue,citecolor=bibleblue,filecolor=black}
\usepackage{balance}
\usepackage{array}
\usepackage{tabularx}
\usepackage{booktabs}
\usepackage[most]{tcolorbox}
\usepackage[french]{fmtcount}
\fmtcountsetoptions{french=france}
\usepackage[singlespacing]{setspace}
\usepackage[super]{nth}
\usepackage[protrusion=true,tracking=true]{microtype}
\usepackage{ragged2e}
\usepackage[all]{nowidow}
\usepackage{enumitem}
\usepackage{adforn}
\usepackage{float}
\usepackage{xcolor}
\usepackage{graphicx}
\graphicspath{ {/home/sync0/Dropbox/paris_1/} }
\usepackage{lipsum}
\usepackage{tikz}
\usetikzlibrary{positioning}
\tikzset{main node/.style={circle,fill=gray!45,draw,minimum size=0.5cm,inner sep=0pt},}
\usepackage{fontspec}
\usepackage{unicode-math}
\usepackage[oldstyle]{libertine}
\defaultfontfeatures{Scale=MatchLowercase}
\newfontfamily\titlefamily[Scale=2]{Linux Biolinum O}
\newcommand\HUGE{\fontsize{30}{33}\selectfont}
\usepackage{scrlayer-scrpage}
\pagestyle{scrheadings}
\clearscrheadfoot
\automark[chapter]{part}
\cehead{vers la machine à gourverner}
\cohead{\MakeLowercase{\headmark}}
\ohead{\pagemark}
\AfterTOCHead{\singlespacing}
\setkomafont{labelinglabel}{\normalfont\sffamily\bfseries}
\setkomafont{minisec}{\usekomafont{subsection}}
\setkomafont{pagehead}{\normalfont\sffamily\mdseries\scshape}
\setkomafont{pagenumber}{\large\rmfamily\upshape}
\setkomafont{sectioning}{\sffamily\mdseries}
\setkomafont{caption}{\small}
\setkomafont{captionlabel}{\sffamily\mdseries\scshape\lowercase}
\setkomafont{part}{\HUGE\scshape\lowercase}
\renewcommand*{\partformat}{}
\renewcommand\partmarkformat{}
\setkomafont{chapter}{\Huge\scshape\lowercase}
\renewcommand{\raggedchapter}{\centering}
\renewcommand*\chapterformat{\thechapter\autodot\par\enskip}
\renewcommand*\chaptermarkformat{}
\setkomafont{section}{\Large\scshape\lowercase}
\setkomafont{subsection}{\large}
\renewcommand*{\subsubsectionformat}{\sffamily\mdseries\upshape \thesubsubsection\autodot\enskip}
\setkomafont{subsubsection}{\large\itshape}
\AtBeginDocument{\renewcaptionname{english}\contentsname{contents}}
\addtokomafont{chapterentry}{\large\sffamily\mdseries\scshape\lowercase}
\usepackage{tocstyle}
\settocfeature{raggedhook}{\raggedright}
\selecttocstyleoption{tocgraduated}
\usetocstyle{nopagecolumn}
\newtcolorbox{note}[2][]{grow to right by=0mm,grow to left by=-1em,boxrule=0pt,boxsep=0pt,opacityback=0.0,breakable,parbox=false,enhanced jigsaw,borderline west={4pt}{0pt}{lightgrey},title={#2},coltitle={black},fonttitle={\sffamily},attach title to upper={},halign title=right,after title={\smallskip\par}#1}
\newtcolorbox{question}[2][]{grow to right by=0mm,grow to left by=-1em,boxrule=0pt,boxsep=0pt,opacityback=0.0,breakable,parbox=false,enhanced jigsaw,borderline west={4pt}{0pt}{darkgrey},title={#2},coltitle={black},fonttitle={\sffamily},attach title to upper={},halign title=right,after title={\smallskip\par}#1}
\newtcolorbox{definition}[3][]{grow to right by=0mm,grow to left by=-1em,boxrule=0pt,boxsep=0pt,opacityback=0.0,breakable,enhanced jigsaw,borderline west={4pt}{0pt}{midgrey},title={#2},coltitle={black},fonttitle={\sffamily\bfseries},fontupper={\normalfont},fontlower={\itshape},lower separated=false,attach title to upper={},after title={\hspace{1em}{\rmfamily\mdseries\itshape #3}\par}#1}
\renewcommand*\labelitemi{\adforn{33}}
\renewcommand*\labelitemii{\adforn{73}}
\renewcommand*\labelitemiii{\adforn{73}}
\renewcommand*\labelitemiv{\adforn{73}}
\definecolor{bibleblue}{HTML}{00339a}
\definecolor{whitegrey}{HTML}{f7f7f7}
\definecolor{lightgrey}{HTML}{cccccc}
\definecolor{midgrey}{HTML}{969696}
\definecolor{darkgrey}{HTML}{636363}
\definecolor{blackgrey}{HTML}{252525}
\newcommand{\notimplies}{\centernot\implies}
\setcounter{secnumdepth}{3}
\setcounter{tocdepth}{1}
\recalctypearea
\author{Carlos Alberto Rivera Carreño}
\date{}
\title{}
\hypersetup{
 pdfauthor={Carlos Alberto Rivera Carreño},
 pdftitle={},
 pdfkeywords={},
 pdfsubject={},
 pdfcreator={Emacs 26.1 (Org mode 9.2)}, 
 pdflang={English}}
\begin{document}

\begin{titlepage}
 \centering
\begin{french}
 {\large \textsc{université paris i panthéon sorbonne} \par}
 \vspace{0.5\baselineskip}
 {\large \textsc{ufr} 02 : Sciences économiques  \par}
 \vspace{0.5\baselineskip}
 {\large Master 2 : Économie et sciences humaines \par}
 \vspace{0.5\baselineskip}
 {\large 2018-2019 \par}
\end{french}
 \vspace{6\baselineskip}
 {\huge\sffamily \textsc{vers la machine à gouverner}  \par}
 {\Large\sffamily Herbert Simon and the Impossibility of a Democratic Computer \par}
\vfill
\begin{french}
 {\large Présenté et sountenu par : \par}
\end{french}
 {\Large Carlos Alberto Rivera Carreño \par}
 \vspace{1\baselineskip}
\begin{french}
 {\large Directeur de mémoire : \par}
\end{french}
 {\Large Jean-Sébastien Lenfant \par}
\end{titlepage}

\pagestyle{empty}

\begin{french}
L'Université Paris 1 Panthéon Sorbonne n'entend donner aucune approbation,
ni désapprobation aux opinions émises dans ce mémoire ; elle doivent être
considérées comme propres à leur auteur. 
\end{french}
\vfill

\newpage
\vspace*{\fill}
\noindent
\includegraphics[height=1.5cm]{gpl3.png}\par
\vspace{1\baselineskip}
This text is free: you can redistribute it and/or modify it
under the terms of the \textsc{gnu} General Public License as published by
the Free Software Foundation, either version 3 of the License or any later
version.

This text is distributed in the hope that it will be useful, but \textbf\{without
any warranty\}; without even the implied warranty of \textbf\{merchantability or 
fitness for a particular purpose\}. See the \textsc{gnu} General 
Public License for more details.

You should have received a copy of the \textsc{gnu} General Public License along
with this text. If not, see \url{http://www.gnu.org/licenses/}.

\vspace{1\baselineskip}
\noindent
Copyright \textcopyright \textsc{sync0} 2018. 

\newpage 
\begin{FlushRight}
\begin{italian}
\textit{Per il professore Giorgio Israel. \newline Benché i nostri destini fossero uniti da quell'anno fatale del 1492, \linebreak ho mancato il nostro incontro. \linebreak Sono arrivato in ritardo, come la coscienza della nostra generazione. \linebreak I campi sanguinanti sono così prossimi, \linebreak ma nessuno vede niente. \linebreak Nell'era dell'informazione, diventiamo più ignoranti. \linebreak Questo è il prezzo del biglietto d'ingresso. \linebreak Grazie a Lei l'ho capito, maestro.}
\end{italian}
\end{FlushRight}

\newpage
\tableofcontents 

\frontmatter
\pagestyle{plain}
\chapter{Note aux lecteurs francophones} 
Le fait que ce mémoire fut préparé au sein d'une université française
m'oblige moralement à \ldots aux lecteurs francophones. 

Ce mémoire traite de la relation entre l'histoire du concept de travail,
l'histoire de l'orinateur, l'histoire de l'intelligence, et la pensée de
Herbert Simon. J'essai de replacer l'histoire de travail au sein des
questions sur  l'application des analogies entre les sciences sociales et
les sciences naturelles.  

Dans le premier chapitre je raconte \ldots 

Dans le deuxième chapitre je raconte \ldots 

\lipsum
\chapter{Acknowledgements} 
\lipsum

\chapter{Introduction: Science and Analogy} 

\lipsum
\mainmatter
\pagestyle{scrheadings}
\chapter{Did Adam Smith Invent the Computer?}
\label{sec:org8a25a05}
A story that although was present for a long time, seems today relegated to
the confines of a few books on the history of computing. 
\section{What is a Computer?}
\label{sec:orgf2f096f}
\lipsum
\section{Simon and Babbage}
\label{sec:orgf0e314b}
\lipsum
\section{De Prony's Tables}
\label{sec:orgaa8cd25}
\lipsum
\chapter{Economy of the Body; Economy of the Machine}
\label{sec:orgdcd6fe4}
\section{Does Labor Exist?}
\label{sec:org6356e5b}
\section{Society and Nature}
\label{sec:org6623df1}
\section{Historicizing Efficiency}
\label{sec:org67fb676}
\chapter{UnDemocratic Machines}
\label{sec:org944142c}
\section{What is Industrial Democracy?}
\label{sec:org128bc0e}
I want to talk about economic democracy as Alain Supiot does.  

\lipsum
\section{Administrative Behavior}
\label{sec:orgd75fe7d}
\lipsum
\section{Artificial Intelligence}
\label{sec:orgba34f0f}
\lipsum

\backmatter
\chapter{Conclusion} 
\lipsum
\chapter{References} 
\lipsum
\end{document}