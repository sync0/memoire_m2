% Intended LaTeX compiler: lualatex
\documentclass[version=last,draft=false,paper=A4,portrait,twoside=true,twocolumn=false,headinclude=false,footinclude=false,mpinclude=true,fontsize=12,BCOR=20mm,DIV=calc,pagesize=auto,open=right,chapterprefix=true,numbers=autoendperiod,headsepline=false,headings=twolinechapter,parskip=false]{scrbook}
\usepackage{graphicx}
\usepackage{grffile}
\usepackage{longtable}
\usepackage{wrapfig}
\usepackage{rotating}
\usepackage[normalem]{ulem}
\usepackage{amsmath}
\usepackage{textcomp}
\usepackage{amssymb}
\usepackage{capt-of}
\usepackage{hyperref}
\usepackage{polyglossia}
\usepackage[most]{tcolorbox}
\usepackage[singlespacing]{setspace}
\usepackage[french]{fmtcount}
\fmtcountsetoptions{french=france}
\usepackage[super]{nth}
\usepackage[backend=biber,bibstyle=authoryear,citestyle=authoryear-icomp,doi=false,isbn=false,url=false]{biblatex}
\addbibresource{~/Dropbox/research/bibliography.bib}
\usepackage[autostyle=true,english=american,french=guillemets,thresholdtype=words,threshold=3]{csquotes}
\SetCiteCommand{\autocite}
\usepackage[oldstyle]{libertine}
\newcommand{\sie}[1]{\textsc{\romannumeral #1}\textsuperscript{e} siècle}
\newenvironment{summary}{\begin{addmargin}{2em}\begin{minipage}{\linewidth}\begin{itshape}}{\end{itshape}\end{minipage}\end{addmargin}}
\newtcolorbox{modified}[1][]{grow to right by=0mm,grow to left by=-1em,boxrule=1pt,boxsep=0pt,breakable,enhanced jigsaw,borderline west={0pt}{0pt}{lightgrey},lower separated=false,arc=00mm,colframe=white, #1}
\newtcolorbox{note}[2][]{grow to right by=0mm,grow to left by=-1em,boxrule=0pt,boxsep=0pt,opacityback=0.0,breakable,parbox=false,enhanced jigsaw,borderline west={4pt}{0pt}{lightgrey},title={#2},coltitle={black},attach title to upper={},halign title=right,after title={\smallskip\par}#1}
\newtcolorbox{question}[2][]{grow to right by=0mm,grow to left by=-1em,boxrule=0pt,boxsep=0pt,opacityback=0.0,breakable,parbox=false,enhanced jigsaw,borderline west={4pt}{0pt}{darkgrey},title={#2},coltitle={black},attach title to upper={},halign title=right,after title={\smallskip\par}#1}
\newtcolorbox{definition}[3][]{grow to right by=0mm,grow to left by=-1em,boxrule=0pt,boxsep=0pt,opacityback=0.0,breakable,enhanced jigsaw,borderline west={4pt}{0pt}{midgrey},title={#2},coltitle={black},fonttitle={\sffamily\bfseries},fontupper={\normalsize},fontlower={\itshape},lower separated=false,attach title to upper={},after title={\hspace{1em}{\rmfamily\mdseries\itshape #3}\par}#1}
\definecolor{bibleblue}{HTML}{00339a}
\definecolor{whitegrey}{HTML}{f7f7f7}
\definecolor{lightgrey}{HTML}{cccccc}
\definecolor{midgrey}{HTML}{969696}
\definecolor{darkgrey}{HTML}{636363}
\definecolor{blackgrey}{HTML}{252525}
\setmainlanguage{english}
\setotherlanguages{french}
\MakeForeignQuote{french}{«}{»}
\usepackage[protrusion=true,tracking=true]{microtype}
\author{Carlos Alberto Rivera Carreño}
\date{}
\title{}
\hypersetup{
 pdfauthor={Carlos Alberto Rivera Carreño},
 pdftitle={},
 pdfkeywords={},
 pdfsubject={},
 pdfcreator={Emacs 26.2 (Org mode 9.2.3)}, 
 pdflang={English}}
\begin{document}


\chapter{Introduction}
\label{sec:orgb707a63}
Sunday night September 23, 1962, \emph{The Jetsons}, an animated sitcom, aired for
the first time, introducing audiences across the United States to the
futuristic life of the Jetson family. Just like their previous sitcom \emph{The
Flintstones} had done for the Stone Age, animators William Hanna and Joseph
Barbera celebrated the American way of life in a future of private flying
cars, nuclear family arrangements, and---oddly enough---salary work. George
Jetson, the proverbial American \emph{common man}, worked daylong pressing a
button at a company, while his wife Jane Jetson, despite relieved from
drudgery thanks to a robot maid and a fully automated apartment, was
relegated to the household. With all the technological advances, why is it
necessary for George, or for anybody, to work ? Why would there be
vacations in the future, if technology can afford us a lifetime of leisure?
The problem with the Jetsons is its presentism: Neither political nor
social, but only technological change is possible in its picture of the
future.

Contrary to this technicist narrative, this thesis seeks to show that
technical changes are embedded in wider social and ideological discussions:
What is work? What is the place of work in society? What is the nature of
subordinated work? What is a worker? Whatever has prevented thinking about
these issues, be it the resilience of a ``broad church'' positivism in the
social sciences or the belief in the transparency and self-sufficiency of
``facts'', the transformations on the future of labor by the development of
new ``intelligent'' machines have to be thought by referring to wider
issues. By looking at how thinking about the prospects of machine
intelligence went hand in hand with changes in thinking about the status of
work and worker and social changes in the technical organization of
production. My main argument is that the dominant ``technicist'' view of the
influence of ICTs on the future of work and its organization is inadequate
because it ignores that thoughts on machine intelligence, automation, and
sociotechnical systems for the conduct of workers have been strongly
connected since at least the nineteenth century.

\section{Organization of the Thesis}
\label{sec:orga868409}

In the first chapter, I will argue the dominant technicist view, which
analyses the influence of automation technologies on the future of work in
terms of the possibilities to automate different (categories of) tasks, is
inadequate because it ignores the conventionalist dimension of the wage
relation by construing work from a realist epistemology. After
establishing, in part two chapter one, that the technicist view ignores the
historical specificity of the concept of work and labor, in the second
chapter, I sketch the historical relation between the history of the
artificial intelligence and the concept of labor to suggest that thinking
about the possibility of thinking machines and the construction of
sociotechnical systems for the conduct of workers have been related since
at least the nineteenth century. The division of labor as embodied in the
allocation of tasks in de Prony's project shows, the first large \emph{computing}
project was not only a project that showed the efficiency of the
distribution of mental labor, but a new form of subordination of the mind
to a particular organization of mental work in which workers were not only
deprived of control but of full knowledge over the productive process.

\chapter{Would You Bet Against Sex Robots?}
\label{sec:orga4a6acb}
From the victory of \textsc{ibm}'s Deep Blue over Gary Kasparov at chess in
1997 to the victory of AlphaGo over Lee Sedol at go in 2016, the progress
of artificial intelligence (\textsc{ai}) in the past two decades has
reignited controversies over the substitutions of humans by super
computers. Does the progress of information and communication technologies
(\textsc{ict}s) drives us to a world in which ``men are absolutely
nothing'', while the ``king, turning a handwheel alone in an island,
carries through automatons all the work of
England''\autocite[330]{sismondi1819_2}?\footnote{My translation.} Regarding the effects on
labor, the disappearance of obsolete jobs and the alleged impoverishment of
the laboring masses preoccupy the optimists in the prospects of automation
\autocite{ford2009}. 

The problem with contemporary debates about the automation of human
activity, both lay and academic, is their naively ``technicist''
perspective: Could we mechanize this or that task, and at what cost? What
would be the effect of increasing ``computer capital'' on productivity
\autocite{frey_osborne2013}? Can we estimate the impact on the rate of
unemployment \autocite{acemoglu_restrepo2018}? In this chapter, I will
present what I call the \emph{technicist view} by discussing a paper by economist
David Autor, and, then, I will discuss the limitations of this perspective
to understand the influence of technology on labor.

\section{The Technicist View}
\label{sec:org4ea58bd}
For the purposes of this thesis, the technicist view is defined as the
application of a realist epistemology to the question of the influence of
technology on labor. This view construes \emph{labor} as a historical and
transcultural universal, whose technical characteristics are improved upon
by technology; these changes are thought in terms of the technical
possibilities to automatize a particular task.\footnote{For a discussion of historical universals see the first two chapters
of \textcite{foucault2004}.} Moreover, since jobs
are understood as collections of tasks, if all the tasks composing a job
can be automatized by a firm in a cost-efficient manner, this job is then
believed to disappear in the near future. I contend that many mainstream
economists and many computer scientists who discuss the problem of the
influence of artificial intelligence on the future of labor, be they
pessimists or optimists, hold either this view or one that resembles it.
Since MIT economist David Autor is one of the mainstream economists that
has studied the longest the question of the consequences of automation on
labor, I will analyze his paper \citetitle{autor2015} as an ideal type of the
way mainstream neoclassical economists understand the problem.\footnote{I decided to present the view of an economists and not of, say, a
computer scientist because contemporary society grants economists the last
word on matters regarding labor.}


In \citetitle{autor2015}, Autor tries to explain the reasons for the secular
high levels of aggregate employment, despite the advances in labor-saving
technologies. Autor believes that focusing exclusively on job losses is
misguided because understanding the relation between technology and
employment requires thinking about the complementarities of different tasks
with these technologies, the price and income elasticities of different
kinds of outputs, and the labor supply responses. His argument is that
there are certain tasks that cannot be automated but benefit from the
productivity rise of new technologies. Automation, thus, complements and
raises the value of jobs composed primarily of un-automatable tasks by
increasing the demand for labor and raising earnings. Consequently, Autor
believes that commentators often overstate ``the extent of machine
substitution for human labor'' because they ``ignore the strong
complementarities between automation and labor'' \autocite[p.
5]{autor2015}.

To understand how automation affects different jobs in the skills
hierarchy, Autor distinguishes between two sets of tasks that have ``proven
stubbornly challenging to computerize'': ``Abstract'' tasks are those
placed higher in the skill hierarchy, which characterize professional,
technical and managerial occupations that require high education levels,
problem-solving capabilities, inductive reasoning, communication ability,
expert mastery, intuition, and creativity; while ``manual'' tasks are those
placed lower in the skill hierarchy, which characterize unskilled
occupations (serving, cleaning, janitorial work, in-person health
assistance, etc.) that require situational adaptability, visual and
language recognition, and in-person interactions \autocite[p.
9]{autor2015}.

Autor believes that abstract task-intensive occupations are strongly
complemented by computer technologies because these enable them ``to
further specialize in their area of comparative advantage'' by spending
less time on ``acquiring and crunching information''
\autocite[15]{autor2015}. Even if wage gains could be mitigated due to
lower expenditures on the outputs of abstract task-intensive activities,
Autor believes that evidence suggests that demand for these services has
kept pace. Most importantly, workers in abstract task-intensive occupations
benefit from the high barriers, such as college and graduate degrees, that
require long years to prepare (i.e., inelastic labor supply).

As for manual task-intensive occupations, which hardly rely on information
or data processing, Autor believes that they benefit from advances in
computer technologies mostly indirectly. For example, the coupling of high
income elasticity of demand for manual tasks-intensive work with rising
aggregate incomes thanks to productivity growth in other areas could raise
demand for manual tasks-intensive occupations. Following
\textcite{baumol1967}, Autor believes that ``wages in these occupations \emph{must}
rise over time'' to compensate workers because otherwise they would choose
a different occupation.\footnote{This claim only holds assuming that demand for manual task-intensive
occupations is relatively inelastic \autocite[17]{autor2015}.} That said, Autor recognizes these wage
increases will be limited by the low barriers to entry in these occupations
and the labor supply increase from workers affected by automation and
offshoring \autocite[17]{autor2015}.

Autor discusses these two categories of tasks in the context of ``job
polarization'' \autocite{goos_manning2003}, which refers to the simultaneous
growth of high-education high-wage jobs (professional, managerial,
technical) and low-education low-wage jobs (low paid personal services), at
the expense of middle-wage middle education jobs (skilled blue collar,
clerical, sales). He finds evidence for this phenomenon in the United
States, where middle-skill occupations (sales, office and administrative
workers, production workers, and operatives) have declined from 60 percent
of employment in 1979 to 46 percent in 2012. Citing \textcite{nordhaus2007},
he claims that the dramatic fall in the cost of computing since 1980 rose
the incentives to substitute computers for human labor at the explicit
codifiable tasks---which, he labels ``routine tasks''---that characterize
many middle-skilled cognitive and manual activities. It is precisely these
tasks, Autor believes, that are increasingly performed by machines, which
explains the decline in clerical and administrative support. Job
polarization, however, is unlikely to erase all middle-skill jobs because,
Autor conjectures, many of the tasks that compose middle skill jobs cannot
be unbundled and outsourced to machinery. As a result, middle skill jobs
that combine routine technical tasks and non-routine tasks where humans
still hold a comparative advantage (interpersonal interaction, flexibility,
adaptability, and problem solving) are likely to persist in the following
decades. As for the problem of job polarization leading to wage
polarization, he believes that, while possible, that trend would not
continue indefinitely, except in certain times and specific labor markets.

Furthermore, citing Michael Polanyi's concept of ``tacit knowledge'', Autor
tries to explain the difficulties faced by automation. This concept refers
to a type of knowledge that, while necessary to accomplish certain tasks,
cannot be consciously recalled by whoever possesses it---we don't know what
we know. Autor, then, mobilizes this concept to explain why tasks
``demanding flexibility, judgment, and common sense'' have proved the most
difficult to automate \autocite[11]{autor2015}. Nevertheless, he
acknowledges that ``environmental control'' and machine learning could
bypass ``Polanyi's Paradox'' and, thus, facilitate the penetration of
automation in heretofore untouched areas: the former by simplifying the
complex environment in which machines operate, and the latter by inferring
heuristic rules of conduct through data analysis that dispense with the
need to know these rules in advance.

Autor concludes that even though in the last few decades technology has led
to job polarization, this trend is unlikely to continue because machines
raise the value of the tasks that humans uniquely provide---problem
solving, adaptability, and creativity---through complementarity effects.
Therefore, focusing exclusively on substitution effects is misguided. He
ends the article by claiming that the economic problem of the future will
be one of distribution and not of scarcity.


\section{The Historicity of Labor}
\label{sec:orgde59c1a}

Autor's article was presented to understand the way mainstream economists
think about the impact of technology on labor. I claim that the problem
with this technicist view is that by relying on a realist epistemology, it
ignores the crucial conventionalist character of social representations and
institutions. For example, the English word \emph{work} refers to both the act of
working and the thing produced, \emph{labor} refers to work in relation to the
``social question,'' and \emph{job} or \emph{occupation} refer (mostly) to the
recognition of a productive social role by law.\footnote{These definitions are provided in loose form precisely to highlight
their polysemy.} Therefore, by
interpreting a job as simply a collection of tasks, Autor ignores that jobs
are codified in laws that institute certain status, rights, and
responsibilities. And, as is the case for matters in jurisprudence, issues
of definitions, interpretation, and representations are crucial.


To think about the conventionalist character of these social objects, it is
 useful to mobilize Karl Polanyi's concept of fictitious commodities.
 Polanyi introduced this concept to explain how in capitalism \emph{labor}, \emph{land},
 and \emph{money} became commoditized \autocite{polanyi1944_2001}. This concept is
 important because many mainstream economists proceed to analyze the market
 ``dynamics'' of the factors of production, without inquiring about the
 social conditions of possibility that have granted them their status as
 commodities. For example, when Autor claims that ``the elasticity of labor
 supply can mitigate wage gains'', one is left to wonder why there should
 be a labor supply at all. Contemporary economic analyses takes for granted
 that there will be people willing to work for a wage, but as studies of
 labor in bronze age Mesopotamian society have shown, getting others to
 work for one's account is an extraordinarily complex matter
 \autocite[5]{steinkeller2015}. In fact, Steinkeller claims that in
 ancient Mesopotamia, the problem was often lack of population to do all
 the work (demanded mostly by the palaces and temples) and not, as it is
 today, \emph{unemployment} \autocite[9--19]{steinkeller2015}. The reason is
 to be found in the social and institutional configuration, in which most
 of the peasant citizenry was not alienated from their means of
 subsistence---specifically, arable land. By linking citizenship to
 landownership, Mesopotamian institutions secured for the peasantry a
 certain independence from both public authorities and potential
 ``employers.'' Except for the annual corvee labor provided as is today's
 military service---a mostly non-remunerated public duty---, the peasantry
 had few incentives to work for others, when they could rely on themselves
 and their households.\footnote{Regarding the importance of corvee labor in ancient Mesopotamia and
ancient Egypt, Hudson remarks that the ``Hollywood'' image of slaves
building the pyramids under the scorching sun and the whip of the overseer
is mistaken, for the majority of the labor needed to build these
infrastructures was provided by semi-free labor. He claims that although
slaves and other dependents \emph{did} exist in these societies, they were never
numerous enough to provide the necessary labor requirements
\autocite[649]{hudson2015}.}

Another example of the conventionalist character of labor is the origins of
the work contract. Despite their contemporary ubiquity, the work contract
is a very recent institution, even in Western civilization. As
\textcite{supiot2004_2016} shows for French Civil Law, the ancestor of the
modern work contract (\emph{contrat de travail}) is a lease agreement (\emph{contrat de
louage}), whose origins lie in the Roman \emph{locatio operarum}. Supiot claims
that, regarding the \emph{locatio operarum}, Roman Law reasoned in analogy to the
institution of slavery: Since freemen sold the products of their labor but
not \emph{their} labor, the destitute who worked for the account of others were
thought to lease themselves (\emph{locat se}) as if they leased their slave (\emph{locat
servum}) \autocite[8]{supiot2004_2016}. Following the introduction of
the principles of freedom of business (\emph{liberté du commerce}) and freedom of
industry (\emph{liberté de l'industry}) to work (\emph{travail}) by the Decree d'Allarde
(16 February 1791), the French civil code assimilated the labor relation to
a type of lease. As for the phantasmic object of the work contrat---Is it
the \emph{worker}? Is it the \emph{work}?---, Supiot claims that the possibility to
contractualize work derives from the application of the concept of property
(\emph{patrimoine}) to the relation of \emph{work} to the worker: Since work is the
private property of the workers, they have the right to freely dispose of
it on the market \autocite[45--66]{supiot1994_2011}.\footnote{This discussion provides only an intuition but not an accurate
description of the significance of these terms in the context of the
jurisprudence. The differences between Civil Law and Common Law compound
the difficulty of discussing these terms unambiguously since they are not
equivalent across legal systems. As an example of the challenge of cross
country comparisons in the European Union, Supiot remarks that even the
concept of work contract differs from country to country
\autocite{supiot_et_al1999_2016}.}

As these examples show, the social conditions that push the majority of the
population to sell their labor in exchanges for wages require an
explanation. These social conditions are what french sociology of work
(\emph{sociologie du travail}) calls the \emph{salariat} (the institution of wage
earning). I content that the \emph{salariat} is often elided in mainstream
economic analyses of the influence of technology on work. But, this
omission is unfortunate because the kernel of the problem of automation is
the relocation of humans within the social conditions of production. An
analysis in terms of the concrete tasks that can (or cannot) be automated
misses the point. Certainly, the technical aspect of new technologies is
crucial, but the discussions about the technical possibilities have to be
inscribed within broader legal, social, and political issues. Thinking that
technological change is simply more of the same but faster, while people
are juggled around by the caprices of the job market is both naive and
wrong. After all, changes in ideas, institutions, and social
representations accompany the introduction of new technologies and their
applications in the workplace.


Although Autor raises some important points such as his discussion of
substitutions and complementarities, his analysis is marred by the
\emph{epistemological realism} with which he addresses the question of the
influence of technology on labor. Thinking about labor in realist
terms---i.e., as if it existed \emph{out there}, independently of our ideas and
representations of it---only apprehends its technical dimension, leaving
aside its embeddedness in the ideological and institutional fabric of
society. The technical aspect is but one melody in a polyphonic chant.
Therefore, the influence of technology on labor cannot be understood by
appealing to the economists' idea of social change, as represented by the
mechanical analogy of a market \emph{mechanism} that shuffles people from job to
job as potatoes and ramps up compensation to janitors to prevent them from
optimizing and becoming doctors. As computer scientist Moshe Vardi's joke
about the advent of ``sex robots'' shows, new technologies change the way
we relate to \emph{stuff}, but also to other human beings \autocite{yuhas2016}. Far
from being automatic, these changes are always contested issues that
crystallize into conventions that are sometimes reopened for debate. The
answer to the question ``what is the role of humans in a world or
intelligent machines?'' wont be answered by robots, but by ourselves.

 As recent discussions of the replacement of judges by algorithms show, the
issue is not just of the prowess of machine intelligence, but of our own
understanding of what constitutes the ``production'' of justice: For
example, if we understand the work of a judge to consist \emph{exclusively} in the
comparison of an input---i.e., a case of law---to an existing corpus of
law, the possibility is opened for a textometry algorithm to don the gown
and the wig. Nevertheless, this automation of judges depends on our
\emph{convening} about the right way to produce justice. Undeniably, new
automation technologies will change work for judges by speeding up
information retrieval, but the debate over the complete replacement of
these jobs by machines is also a conceptual question about the nature of
this work. Without a modicum of reflexivity, passing the baton to machines
to eliminate ``human-error'' can only portend disaster. We could risk
creating dangerous sociotechnical systems as in the 1964 movie \emph{Dr.
Strangelove}, in which control of the missile defenses of the United States
and the Soviet Union are automated, resulting in the escalation of nuclear
war due to their ``efficient'' retaliation out of human control. If
machines should ever govern, this will not result from a coup, as in the
1984 movie \emph{Terminator}, but from voluntary regime change.

\chapter{Did Adam Smith Invent the Digital Computer?}
\label{sec:orgab03527}

On November 14, 1957, in an address to the Twelfth National Meeting of the
Operations Research Society of America, Herbert Simon advanced the
provocative proposition that ``physicists and electrical engineers had
little to do with the invention of the digital computer'' because ``the
real inventor was the economist Adam Smith, whose idea was translated into
hardware through successive stage of development by two mathematicians,
Prony and Babbage.'' \autocite[2]{simon_newell1958}. This provocative
statement was no less controversial then, but is it accurate?

To contribute to an alternative to the technicist view, this chapter will
analyze the historical specificity of labor as it manifests in Gaspard
Riche de Prony's project for the calculation of the Cadastre tables and
Charles Babbage's construction of the Difference and Analytical Engines.
What this history shows is that, at least since the nineteenth century,
thinking about the possibility of intelligent machines has accompanied the
construction of sociotechnical systems for disciplining workers. 

\section{Manufacturing Logarithms}
\label{sec:org51d138b}
Gaspard Clair Francois Marie Riche de Prony (1755--1839) was born in
Chamelet in the Beaujolais region of Southern France to a family of the
provincial middle bourgeoisie---the social class that would fill the ranks
of the Revolution and Empire's bureaucracy \autocite{picon_et_al1984}. After
an education in the Classics, in 1776, at twenty-one, he entered the École
des ponts et chaussées in Paris. Prony's life coincides with a period of
the institutionalization of French sciences and techniques with the
foundation in 1794 of the École polytechnique---where he was appointed
professor of analysis and mechanics with Joseph-Louis Lagrange---and the
École normale supérieure. Moreover, this was a time of growing interest
among the savants for applied problems, and the generalization of the
application of mathematical formalisms.

Following the French Revolution, the recently constituted Assemblée
nationale decided in 1790 to replace the ancient taxes by a land tax
\autocite[6]{peaucelle2012a}.\footnote{According to \textcite{peaucelle2012a}, the assembly was inspired in
this regard by the ideas of physiocratic economist François Quesnay.} Therefore in 1791, the Assemblée
founded the Bureau du Cadastre, with de Prony as director, to conduct a
land survey that would draw the boundaries of landed property in every
French commune.\footnote{Although the Cadastre's initial objective was to assist fiscal
policy by levying land taxes, under Prony's direction, the production of
maps and statistical tables took precedence \autocite[p.
76]{peaucelle2012a}.} In connection with this plan, the revolutionary
government decided that a very large set of logarithmic and trigonometric
tables would be produced to supplement the decimal-based metric system.
This was necessary because the traditional sexagesimal division of the
circle---now regarded as quaint and irrational---was replaced by a decimal
division, which rendered obsolete the older trigonometric tables used by
geodesists and astronomers.\footnote{The decimal division of the circle along with the decimal division
of time were later removed from the metric system \autocite[p.
184]{daston1994}.} So it fell upon de Prony to direct a
project to calculate and print these tables.

De Prony estimated that even with the help of three or four skillful
collaborators, the rest of his life would not suffice to finish the
calculations needed for the tables. One day in front of a book shop, he
spotted ``the beautiful English edition'' of Adam Smith's \emph{Wealth of Nations}
\autocite[7]{firmin-didot1820}. As he opened the book haphazardly, he
stumbled upon chapter one, where Smith discusses the manufacture of pins,
and, suddenly, he conceived the idea to apply the division of labor to the
production of the Cadastre tables ``to manufacture logarithms as one
manufactures pins'' \autocite[35]{prony1824}.\footnote{My translation.} Drawing from his
lessons at the École centrale des travaux publics (later École polytechnic)
on the applications of the method of differences to interpolation, he
conceived a plan to break up the calculation of the final values of the
Cadastre tables into a long series of simpler steps. By using logarithms to
simplify complex operations into additions and subtractions, these
simplified operations could be calculated even by a hairdresser familiar
with only the most rudimentary arithmetic. De Prony, thus, organized the
calculation of the tables by dividing the Cadastre employees into three
sections, following a hierarchy of skill: At the apex, the first section
comprised five or six eminent mathematicians who carried out the analytical
part of the work by choosing the mathematical formulae and the initial
values of the angles. The second section comprised seven or eight
``calculators'', with knowledge of both arithmetic and mathematical
analysis, who derived from the formulae of the first section the values of
the initial logarithms and the initial differences of various
orders.\footnote{At this time, in both French and English, calculator (\emph{calculateur})
referred not to a calculating machine, but to a job.} They also prepared the folio sheets used by the third
section ``by laying out the columns of the chosen values and the first row
of entries'' \autocite[109]{grattan-guinness2003}. Furthermore, they
wrote on the page the instructions to be followed by the third section and
verified their calculations. At the base, the third section comprised
between 60 and 80 calculators who only carried out the additions and
subtractions of the intervals between two numbers, as chosen by the second
section. Prony remarked that ``those who knew more [arithmetic] were not
always those subjected to less errors'' \autocite[p.
53]{prony1804}.\footnote{In his famous article \citetitle{grattan-guinness1990},
Grattan-Guinness claimed that ``[m]any of these workers were unemployed
hairdressers'', but this affirmation, although now widespread, has elicited
controversy \autocite[179]{grattan-guinness1990}. \textcite{roegel2011},
who has closely studied the lists of the employees at the Cadastre,
disagrees with Grattan-Guinness's emphasis on the employment of
hairdressers because he suspects that ``there were only two or three of
them'' \parencite[26]{roegel2011}. Nonetheless, Roegel acknowledges
that the first mention of the hairdressers seems to be an \ordinalnum{11}
November 1824 discourse by engineer Charles Dupin---Prony's close friend---
from which Grattan-Guinness drew upon \parencite{dupin1824}. Moreover, Roegel
claims that the number of computers employed probably never exceeded 20 or
25, so Prony exaggerated his figures. See \textcite{roegel2011}, p. 27, footnotes
96--98). Indeed, a contemporary witness estimates them at about fifteen
\autocite[744]{laLande1803}.} The calculation of the tables began in 1793, and although the
core of the project had been completed by mid-1796, verifications took a
few more years, so that the project was finally completed ``at the end of
the 1790s or the beginning of the 1800s'' \autocite[37]{roegel2011}.

Despite the novel rationalization of the organization of labor that it
promoted, the project of the Cadastre tables was isolated among French
productive activities, which remained mostly artisanal. It was intended as
a ``monument'' to rationality and not a mass produced commodity
\autocite[57]{prony1804}. If the thorough industrialization of French
production would have to wait until the twentieth century, the tables
project was an early example of the application of the engineer mentality
to the organization of labor. This is evidenced in its application of the
division of labor to a hierarchical organization of skills. At the time of
de Prony, even if attitudes towards the \emph{mechanical arts} had been changing
beginning in the ninth century \autocite[10]{vatin_pillon2003_2007},
these were still associated with low ranked repetitive labor devoid of
intelligence.\footnote{In this regard, Friedmann cites the ambivalence in the \emph{Encyclopédie}
of the definition of \emph{artist}:``we say of a good chemist that deftly
performs the procedures that others have invented, that he is a good
artist, with the difference that the word artist is always a praise in the
first case, whereas in the second, it is almost a reproach of only
possessing the subordinate part of his profession''
\autocite[my translation, footnote 1, p. 55]{friedmann1953}.} Discussing the workers of the third section of the
tables project, the famous nineteenth century French engineer Charles Dupin
remarked that they ``ended up blessing the change that removed them from
hard work, to devote themselves to occupations that called upon the use of
thought'' \autocite[211]{dupin1824}.\footnote{My translation.} Nevertheless, according to
\textcite{daston1994}, by pooling the talent of mathematical genius with the
brute force of mindless calculation, de Prony's project contributed to
demote calculation to the lowest of mental faculties. In fact, until the
mid-nineteenth century, in both French and English usage, \emph{work} and
\emph{mechanical} were associated by referring to the laboring body, so that
mechanizing calculation degraded this activity from its previous status as
a manifestation of mathematical gifts.\footnote{As an example of the importance of calculation, the German-Dutch
mathematician, Ludolph van Ceulen (1540-1610), devoted long years to the
calculation of \(\pi\) to thirty-five places. This was considered such an
accomplishment that \(\pi\) is sometimes referred to in German as Ludolph's
number (\emph{Ludolphsche Zahl}) \autocite[50]{maor1994}.} In this regard, Daston
considers that de Prony's project is a landmark in the history of
intelligence because ``it pushed calculation away from intelligence and
towards work'' \autocite[190]{daston1994}.

Although the mechanization of intelligence began with de Prony's project,
its contemporary significance comes from Charles Babbage's interpretation
of it to design a series of powerful mechanical calculators: the Difference
Engine I and II, and the Analytical Engine. Specifically, Babbage's
Analytical Engine is considered the precursor to the modern programmable
computer.

\section{Weaving Algebraical Patterns}
\label{sec:orgfc6e4f3}

Charles Babbage (1791--1871) was born in Walworth, Surrey close to London,
at a time, when the capital was the commercial and industrial center of
Great Britain. Although he never secured an employment for very long, he
inherited a large estate from his goldsmith turned banker father, which
allowed him to pursue his scientific interests more-or-less independently
of financial constraints. Babbage was one of the eminent nineteenth century
Romantic gentlemen scientists,\footnote{This figure would disappear with the advent of twentieth century
Big Science: massive personnel, lavish government budgets, and hierarchical
organization. See \textcite{mirowski2011}.} joining numerous clubs and societies
to advocate the application of science to different domains. For example,
he co-founded the Analytical Society (1811), the Royal Astronomical Society
(1820), the British Association for the Advancement of Science (1831), and
the Royal Statistical Society (1834). During his life---and even today---he
was regarded primarily as a gifted mathematician, graduating from Cambridge
in 1814, and becoming there professor of mathematics in 1828.\footnote{According to Romano, Babbage never moved to Cambridge nor ever
deliver a lecture there as professor of mathematics
\autocite[387]{romano1982}.} In
fact, one of his most important contributions is the introduction of French
``abstract'' mathematics to Great Britain.\footnote{One of the likely reasons for Great Britain's resistance to
Continental advances in mathematical abstraction was the priority debate
over the discovery of the calculus between Newton and Leibniz. According to
Maor, this controversy discouraged the introduction of the Leibniz's
notation in Great Britain. See \textcite{romano1982}, chapter 9.}

At some point, in the year 1820 or 1821, the Astronomical Society appointed
a committee consisting of John Herschel and Babbage to prepare statistical
tables for astronomical calculations. One evening, while the two men were
tediously verifying the numbers produced by two hired ``computers,''
Babbage exclaimed that he wished ``we could calculate by steam.'' Babbage
decided to embark in the construction of a series of calculating machines
to automate the production of statistical tables, the first of which was
called the Difference Engine.\footnote{Although the most widely known are Babbage's Difference Engines (I
and II) and the Analytical Engine, he had plans for more calculating
machines.} This machine derived its name from the
method of differences discussed earlier, which Babbage had \emph{translated} into
mechanism through a complex system of wheels and gears. The Difference
Engine was a massive and heavy machine that received its ``input'' by
setting the initial values on the toothed number wheels, which were, then,
activated by a human turning a crank.\footnote{The machine consisted of several parallel columns of rotating piled
up toothed number wheels. Each column containing \(n\) wheels could represent
a number with  \(n - 1\) decimal places.}

\begin{figure}[htbp]
\centering
\includegraphics[width=7cm]{/home/sync0/Pictures/research/difference_engine_1_woodcut.jpg}
\captionbelow{Section from the Difference Engine I as assembled in 1833 (Source: Wikipedia).}
\end{figure}

\begin{figure}[htbp]
\centering
\includegraphics[width=13cm]{/home/sync0/Pictures/research/difference_engine_2_diagram.jpg}
\captionbelow{Difference Engine II Diagram (Source: Wikipedia).}
\end{figure}

\begin{figure}[htbp]
\centering
\includegraphics[width=11cm]{/home/sync0/Pictures/research/difference_engine_2_drawing.jpg}
\captionbelow{Difference Engine II Drawing (Source: Wikipedia).}
\end{figure}


After completing a working model with six figure-wheels in 1822, he
published an open a letter to Sir Humphry Davy---then President of the
Royal society---to secure government financial assistance. In this letter,
Babbage sketches for the first time the concept of the \emph{mental} division of
labor, citing de Prony's tables project as evidence of the possibility to
mechanize calculation. Babbage explains that the third section of de
Prony's project, whom ``the most laborious part of the operations
devolved'' \autocite[214]{babbage1822}, could be reduced in number of
employees to only twelve by the use of calculating machines: Their labor
would be to copy the numbers displayed by the Engine's wheels.

\begin{displayquote}[{\cite[212]{babbage1822}}]
The intolerable labour and fatiguing monotony of a continued repetition of
similar arithmetical calculations, first excited the desire, and afterwards
suggested the idea, of a machine, which, by the aid of gravity or any other
moving power, should become a substitute for one of the lowest operations
of human intellect.
\end{displayquote}

As this quote shows, Babbage believed that de Prony's project incorporated
a hierarchy of tasks. Although calculation was a mental operation, it
required the least application of intelligence. If \textcite{daston1994} is
correct, the association of mechanism with the laboring body of low-skilled
low-paid laborers was the main factor behind the loss of prestige of
calculation. Daston's claim is important because there is an ``economic''
tendency to explain away this change in status by appealing to \emph{scarcity} and
the \emph{laws} of supply and demand, but this argument is misleading. Despite all
the rhetoric, if Babbage's machine had been completed, only large
organizations would have employed it (as was the case with the first
mass-produced electronic computers in the twentieth century), for, as Maas
argues, its sheer size rendered it impractical for most everyday scientific
applications \autocite[103]{maas2005}.\footnote{During Babbage's lifetime, only functioning sections of the Engines
were produced, but not a single one was completed in its entirety. In 1991,
a group of researchers constructed a full size Difference Engine II to
prove that such a machine could be built using the materials and
engineering tolerances available to Babbage. See \textcite{swade2000}.} Instead, the Engine
contributed to shape the nineteenth century ideas in Great Britain about
intelligence by showing that what had been thought to be the prerogative of
the mind could be accomplished by mechanism. It was, thus, the lack of
social prestige of mechanism---and its association with the low class of
citizens that would soon become the proletariat---that undermined the
social status of calculation.

Another important aspect of Babbage's work, with major implications for the
construction of his Engines, was his famous book
\citetitle{babbage1832_2009}, which consecrated him as an expert on
industrial production techniques. As he constructed the Difference Engine,
Babbage was forced to turn his attention to the study of machinery and
manufactures because of the precision engineering required to produce the
tolerance levels of the Engine's pieces. After the death of his wife in
1827, Babbage embarked on a long tour to study Europe's best methods of
production, and, back in England in late 1828, he began working on his
famous book on manufactures. The book, published in 1832, discusses de
Prony's project as if resembling a British cotton or silk-mill, thereby
drawing a close analogy not only between mental labor and bodily labor, but
between calculating machine and manufactory.

\begin{displayquote}[{\cite[157]{babbage1832_2009}}]
When it is stated that the tables thus computed occupy seventeen large
folio volumes, some idea may perhaps be formed of the labour. From that
part executed by the third class, which may almost be termed mechanical,
requiring the least knowledge and by far the greatest labor, the first
class were entirely exempt. \ldots when the completion of a calculating
engine shall have produced a substitute for the whole of the third section
of computers, the attention of analysts will naturally be directed to
simplifying its application, by a new discussion of the methods of
converting analytical formulae into numbers.
\end{displayquote}

Two years after the book's publication, Babbage began drafting plans for a
more ambitious calculating machine: the Analytical Engine, the precursor to
the modern programmable computer.\footnote{Babbage's Engines were forgotten and then rediscovered in the
middle of the twentieth century, most notably by Howard Aiken, inventor of
the Harvard Mark I. See \textcite{aiken1946}, Introduction. Nevertheless, the
influence of Babbage in the development of the twentieth century electronic
computer is only retrospective. I surmise that despite the long history of
calculating machines, between Leibniz's \emph{Staffelwalze} in the seventeenth
century to \textsc{eniac} in the mid-twentieth century, the influence of
most of these inventions to the development of modern computers is,
unfortunately, only retrospective historical reconstruction.} Unlike the Difference Engine,
which was limited to one operation, the Analytical Engine could be
\emph{programmed} by the use of three types of punched cards---Variable cards,
Number cards, and Operations cards---to perform any series of the four
basic arithmetical operations. Maas's claim that the ``Analytical Engine
incorporates in its design the architecture of a factory'' is not
rhetorical flourish \autocite[109]{maas2005}, for this new mechanical
computer was literally modeled after the large scale division of labor of
English industry and de Prony's project. Specifically, the Analytical
Engine consisted of several logical sections, the most important of which
were the memory, the \emph{Store}, and the central processor, the \emph{Mill}. As their
names imply, the Store was responsible for ``storing'' numbers until they
were required (as indicated by the Variable cards) to be ``milled'' at the
Mill. Translated in mechanical terms, this means that the Store kept a
number---represented by the positions of the piled up toothed wheels in a
column---which was transferred to the Mill when called upon by a
calculation. Every operation, from number input (Number cards), to ordering
the sequence of operations (Operations cards), to number conveyance across
the Engine (Variable cards), was controlled by the three types of punched
cards: These were fed to the Engine as long series of tied up cards that
activated the various levers, which in turn, turned the wheels. What this
short description of the \emph{workings} of the Analytical Engine tries to show is
that Babbage's analogy with a silk mill was not a mere pedagogical
expedient but a reference to the working principles of one object to
understand the functioning of another.\footnote{Understanding the history of any science requires the
identification of the analogies that become engines of discovery for a
period of time. These analogies shed light over certain aspects of an
object, while obscuring others. They may also carry along other habits of
thought from one realm into another. In the transition from mechanical to
electronic calculation, the \emph{human mind} superseded the \emph{factory} as the
analogy used to understand the workings of the machine. This change of
analogical referent had profound consequences for twentieth century
behavioral sciences, which, building on the nineteenth century
neurophysiological reductionism of mind to brain activity, understood the
human mind to be a mechanism akin to an electronic computer, and vice versa} In this regard, it is
noteworthy that Babbage borrowed the idea to use punched cards to \emph{command}
the Engine from the Jacquard loom: A weaving machine that threaded complex
patterns in fabrics by following the instructions codified in the punched
cards. These cards were fed into the loom in long series, which activated a
system of levers that threaded the codified pattern. It was, then, in the
context of analogies between factories and calculating machines that
Babbage's friend Ada Lovelace wrote that ``the Analytical Engine \emph{weaves
algebraical patterns} just as the Jacquard-loom weaves flowers''
\autocite[p. 25]{lovelace1843}.

From the point of view of the social significance of technology, the
Analytical Engine was important because, if with its predecessor, the
Difference Engine, thoughts on machine intelligence were blue sky
speculations, the new machine, with its more sophisticated automated
systems, elicited discussions on its ability to think---or, at least to
reproduce those ``lower operations of the human intellect.'' For example,
whereas one Italian engineer who corresponded with Babbage assured that
``the machine is not a thinking being, but a simple automaton that
performed according to the laws that one has traced for it''
\autocite[my translation, p. 358]{menabrea1842}, Ada Lovelace claimed
that:

\begin{displayquote}[{\cite[22]{lovelace1843}}]
It were much to be desired, that when mathematical processes pass through
the human brain instead of through the medium of inanimate mechanism, it
were equally a necessity of things that the reasonings connected with
\emph{operations} should hold the same just place as a clear and well-defined
branch of the subject of analysis \ldots
\end{displayquote}

Although Lovelace's interpretation was probably rare at the time, her way
of thinking---identifying intelligence with operations that could be
carried out in different ``hardware''---proved important for the history of
intelligence because she drew a parallel between logic, mechanical
operations, and the operations of the mind. This same idea would surface
with the publication in 1954 of George Boole's \emph{Laws of Thought}, in which he
presented Boolean algebra both as an abstract theory of logic and a
psychological theory of the operations of the mind.

What is curious is that this discourse about machine intelligence is
contemporaneous to the submission of a whole class of citizens to the
factory system, the wage form, and the contractualization of labor
relations. As \textcite{schaffer1994} argues, these concerns were central to
the manufacture of the Engine's pieces with the right tolerance levels
because their production required not just more precise tools, but also a
reconfiguration of the social space of production. Specifically, the
production of Babbage's Engines accompanied not only the introduction of
advances in machine tools and engineering techniques, but the disembodiment
of the ownership of skills that had been the property of workers. For
example, the newly introduced automatic lathes could produce more precise
pieces, but their operation implied both a re-training of workers to watch
over the machine and a loss of skill, which was traditionally recognized as
the property of the worker \autocite[214]{schaffer1994}. The automatic
tools appropriated and incorporated the skills of the workers. For this
reason, Schaffer interprets Babbage's 1832 book on manufactures as a work
of industrial intelligence to wrest off technical knowledge from workers,
to be appropriated by Babbage as a kind of proto-manager. This knowledge
was then mobilized to build the tools and the workshops where the Engine's
pieces would be produced. As Schaffer mentions, in Babbage's London, tool
manufacturing followed certain customs regarding production techniques,
``intellectual property'', and control and authority over the production
process. The ``technical'' advances required for producing the Engines were
not innocent, in Schaffer's interpretation, because both the discourse
about the organization of work and the social relations of production had
to be transformed for the construction of the Engines to be possible.
Schaffer mentions, among other changes, the dispute over the property and
the rights over the automatic tooling to produce the Engine's
pieces---these were traditionally owned by master engineers---and the
relocation, at Babbage's behest, of the workshop to his house---to have
greater control over the surveillance of production. As Schaffer argues,
the ``philosophers of machinery'' promoted rational valuation of work to
render the labor process transparent, and skill measurable and thus payable
according to a wage set in a marketplace. Workers \emph{seemingly} vanished from
the ``automated'' production process of the new thinking machines because,
as machines became better at mechanical and mental tasks, workers were
subordinated as appendices to these machines. But, the workers never
disappeared; they just became invisible.

\chapter{Conclusion}
\label{sec:orgfd0fb0d}

Even though Adam Smith did not invent the electronic computer, in the sense
that he had no direct role in its construction, de Prony's interpretation
of his concept of the division of labor shows that the first large
\emph{computing} project was both an application of Smith's concept and new form
of subordination of workers according to a hierarchy of mental skill.
Likewise, Babbage's interpretation of de Prony's interpretation as embodied
in his Analytical Engine drew a parallel between the economy---in the sense
of the regulating principles---of the machine and the economy of the
factory. These two projects took place at times of major change for the
laboring classes of France and Great Britain. Indeed, two years before de
Prony ran his calculation \emph{atelier}, the French Asemblée Nationale had passed
the Loi Le Chapelier and the Decree D'Allarde, which eliminated the guilds
(\emph{corporations}) and prepared the framework for the modern understanding of
labor as a primarily a contractual relation. In the same way, Babbage's
project of industrial intelligence to wrest off control of production from
workers took place at the same time as the proletarianization of the
British working class, the rise of the factory system, and the beginning of
discussions about the possibility of intelligent machines---to be
distinguished from earlier discussions about automatons. 

By tracing this epistemological history of labor and intelligence in the
context of Prony's reading of Smith, and Babbage's reading of Prony reading
Smith, I sought to show by a historical example that the question of the
influence of automation technology on labor can be told in a different way.
The technicist view, as one manifestation of a realist epistemology, can
lead to a dangerous naturalization of the social, which prevents
discussions about democratic control over the productive process.
Challenging the technicist view is significant for today's discussions
about artificial intelligence because as certain historians such as
\textcite{heyck2012} have emphasized, in the twentieth century, claims about
the (ir)rationality and epistemic ``handicaps'' of humans---our limits to
process \emph{information}---had important political consequences. Although this
relation between epistemic doctrines and politics is well known in the case
of Hayek's criticism of socialism---no one mind could know enough to plan
the economy---, it has been less discussed in the case of Herbert Simon:
Nobel prize in economics, prophet of bounded rationality, and founder of
the discipline of artificial intelligence. Perhaps, the late Simon's
opposition to industrial democracy should be read in conjunction with his
ideas on ``simulation'' of human thought and the future possibilities of
thinking machines \autocite{simon1983b}. After all, if that Italian engineer
believed Babbage's Analytical Engine could only perform its program, Simon
thought this was no less true for humans in the sense that ``[they] do only
what their genes and their cumulative experiences program them to do''
\autocite[45]{simon1960_1985}.









\printbibliography[title=References]
\end{document}