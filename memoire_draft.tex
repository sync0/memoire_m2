% Intended LaTeX compiler: lualatex
\documentclass[version=last,draft=true,paper=A4,portrait,twoside=true,twocolumn=true,headinclude=false,footinclude=false,fontsize=10,BCOR=20mm,DIV=13,pagesize=auto,titlepage=false,mpinclude=false,open=right,chapterprefix=true,numbers=autoendperiod,headsepline=false,parskip=false]{scrbook}
\usepackage{graphicx}
\usepackage{grffile}
\usepackage{longtable}
\usepackage{wrapfig}
\usepackage{rotating}
\usepackage[normalem]{ulem}
\usepackage{amsmath}
\usepackage{textcomp}
\usepackage{amssymb}
\usepackage{capt-of}
\usepackage{hyperref}
\usepackage{polyglossia}
\setmainlanguage{english}
\setotherlanguages{french,italian}
\usepackage[autostyle=true,english=american,french=guillemets,thresholdtype=words,threshold=3]{csquotes}
\MakeOuterQuote{"}
\MakeForeignQuote{french}{«}{»}
\AtBeginEnvironment{quote}{\itshape}
\usepackage[backend=biber,style=reading,eprint=false,abstract=false,library=false,file=false,entrykey=false,doi=false,isbn=false,url=false]{biblatex}
\addbibresource{~/Documents/mendeley/sync0_library.bib}
\renewbibmacro*{entryhead:full}{\printfield{labeltitle}}
\usepackage{amsmath}
\usepackage{amsthm}
\usepackage{amssymb}
\usepackage{centernot}
\usepackage{hyperref}
\hypersetup{colorlinks,urlcolor=bibleblue,linkcolor=bibleblue,citecolor=bibleblue,filecolor=black}
\usepackage{balance}
\usepackage{array}
\usepackage{tabularx}
\usepackage{booktabs}
\usepackage[most]{tcolorbox}
\usepackage[french]{fmtcount}
\fmtcountsetoptions{french=france}
\usepackage[singlespacing]{setspace}
\usepackage[super]{nth}
\usepackage[protrusion=true,tracking=true]{microtype}
\usepackage{ragged2e}
\usepackage[all]{nowidow}
\usepackage{enumitem}
\usepackage{adforn}
\usepackage{float}
\usepackage{xcolor}
\usepackage{graphicx}
\graphicspath{ {/home/sync0/Dropbox/paris_1/} }
\usepackage{lipsum}
\usepackage{tikz}
\usetikzlibrary{positioning}
\tikzset{main node/.style={circle,fill=gray!45,draw,minimum size=0.5cm,inner sep=0pt},}
\usepackage{fontspec}
\usepackage{unicode-math}
\usepackage[oldstyle,libertine]{libertine}
\defaultfontfeatures{Scale=MatchLowercase}
\newcommand\HUGE{\fontsize{30}{33}\selectfont}
\usepackage{scrlayer-scrpage}
\pagestyle{scrheadings}
\clearscrheadfoot
\automark[chapter]{part}
\cehead{vers la machine à gourverner}
\cohead{\MakeLowercase{\headmark}}
\ohead{\pagemark}
\AfterTOCHead{\singlespacing}
\setkomafont{labelinglabel}{\normalfont\bfseries}
\setkomafont{minisec}{\usekomafont{section}}
\setkomafont{pagehead}{\normalfont\mdseries\scshape}
\setkomafont{pagenumber}{\large\rmfamily\upshape}
\setkomafont{sectioning}{\mdseries}
\setkomafont{caption}{\small}
\setkomafont{captionlabel}{\mdseries\scshape\lowercase}
\setkomafont{part}{\HUGE\scshape\lowercase}
\renewcommand*{\partformat}{}
\renewcommand\partmarkformat{}
\setkomafont{chapter}{\huge\scshape\lowercase}
\renewcommand{\raggedchapter}{\centering}
\renewcommand*\chapterformat{\thechapter\autodot\par\enskip}
\renewcommand*\chaptermarkformat{}
\setkomafont{section}{\large\scshape\lowercase}
\setkomafont{subsection}{\large}
\renewcommand*{\subsubsectionformat}{\mdseries\upshape \thesubsubsection\autodot\enskip}
\setkomafont{subsubsection}{\large\itshape}
\AtBeginDocument{\renewcaptionname{english}\contentsname{contents}}
\addtokomafont{chapterentry}{\mdseries\scshape\lowercase}
\setkomafont{chapterentrypagenumber}{\normalfont}
\usepackage{tocstyle}
\settocfeature{raggedhook}{\raggedright}
\selecttocstyleoption{tocgraduated}
\usetocstyle{nopagecolumn}
\newtcolorbox{note}[2][]{grow to right by=0mm,grow to left by=-1em,boxrule=0pt,boxsep=0pt,opacityback=0.0,breakable,parbox=false,enhanced jigsaw,borderline west={4pt}{0pt}{lightgrey},title={#2},coltitle={black},fonttitle={},attach title to upper={},halign title=right,after title={\smallskip\par}#1}
\newtcolorbox{question}[2][]{grow to right by=0mm,grow to left by=-1em,boxrule=0pt,boxsep=0pt,opacityback=0.0,breakable,parbox=false,enhanced jigsaw,borderline west={4pt}{0pt}{darkgrey},title={#2},coltitle={black},fonttitle={},attach title to upper={},halign title=right,after title={\smallskip\par}#1}
\newtcolorbox{definition}[3][]{grow to right by=0mm,grow to left by=-1em,boxrule=0pt,boxsep=0pt,opacityback=0.0,breakable,enhanced jigsaw,borderline west={4pt}{0pt}{midgrey},title={#2},coltitle={black},fonttitle={\bfseries},fontupper={\normalfont},fontlower={\itshape},lower separated=false,attach title to upper={},after title={\hspace{1em}{\rmfamily\mdseries\itshape #3}\par}#1}
\renewcommand*\labelitemi{\adforn{33}}
\renewcommand*\labelitemii{\adforn{73}}
\renewcommand*\labelitemiii{\adforn{73}}
\renewcommand*\labelitemiv{\adforn{73}}
\definecolor{bibleblue}{HTML}{00339a}
\definecolor{whitegrey}{HTML}{f7f7f7}
\definecolor{lightgrey}{HTML}{cccccc}
\definecolor{midgrey}{HTML}{969696}
\definecolor{darkgrey}{HTML}{636363}
\definecolor{blackgrey}{HTML}{252525}
\newcommand{\notimplies}{\centernot\implies}
\setcounter{secnumdepth}{3}
\setcounter{tocdepth}{1}
\recalctypearea
\setlength{\columnsep}{0.5cm}
\author{Carlos Alberto Rivera Carreño}
\date{}
\title{}
\hypersetup{
 pdfauthor={Carlos Alberto Rivera Carreño},
 pdftitle={},
 pdfkeywords={},
 pdfsubject={},
 pdfcreator={Emacs 26.1 (Org mode 9.2)}, 
 pdflang={English}}
\begin{document}

\nocite{*}
\chapter{Introduction}
\label{sec:org722e8fb}
\section{Accroche: ``The Concept of Work between Society and Nature''}
\label{sec:orgf760722}
In the \emph{accroche}, I will describe how something as self-evident as the work
concept is in fact an ideological construct (ideological in the sense of
the anthropologist Louis Dumont, and not in the Marxist sense). Therefore,
I will briefly indicate that the concept of work begins in political
economy and is then appropriated by natural science to be reimported into
economic science.  

Likewise, I will indicate that the motivation for the the thesis is to
rethink what work is in the 21st century, taking into account the ubiquity
of information technologies. 

\printbibliography[heading=none,keyword=introduction-1]
\section{Main Argument: ``Rethinking Work in the 21st Century''}
\label{sec:orgbbf1c41}
In this section, I will state the main argument of the thesis: namely, that
the origin of the computer is a particular conception of the organization
of work, which we should take into account to understand the transformation
of work (its forms, its meaning, its formal definition in law, etc.) in the
21st century. As such, this argument calls into question the ``popular''
understanding of the one-sided transformation of work by the appeareance
and dissemination of information and communication technologies (ICTs) in
the late \ordinalnum{20} century. Instead, this thesis proposes to read the
computer as a technology for organizing labor, to then use this reading to
understand the transformation of labor that ICTs are supposedly pushing
for. 

\printbibliography[heading=none,keyword=introduction-2]
\section{Structure of the Thesis}
\label{sec:orga322746}
In this section, I simply describe the contents of the two chapters of the
thesis and hint at the conclusion. 

The first chapter traces the origins of the computer in de Prony and
Babbage. The idea is to provide the reader with enough background knowledge
to understand that the computer is not a thing but a concept, and that the
definition of this concept has to be approached historically. This chapter
will define the early computer as a conception of the organization of work
that reflects an engineering and managerial mentality. Likewise, this
chapter will hint at the link between this conception of work and its
definition in law as a relation of subordination.

The second chapter will describe Herbert Simon's interest in Babbage, and
will speculate on how this reading shaped Simon's conception of the
computer, the relation between the natural and the artificial, the changes
in work produced by new technologies (artificial intellgence, computers,
and automation), and the organization of work in society. Moreover, this
chapter will criticize Simon's idea that the organization of work is a
purely technical problem to propose an alternative view of work that
emphases other criteria for organization such as justice, etc.

\section*{References}
\printbibliography[heading=none,keyword=introduction]
\chapter{Chapter 1}
\label{sec:org76238a5}
This chapter introduces the definition of the early computer as a
technology for organizing work. The first objective of this chapter is to
question the belief that the appropriate definition of the computer is in
terms of its components. The second objective is to acquaint the reader
with the history of the early computer by describing the project of the
calculation of the logarithmic tables at the \emph{Bureau du cadastre}, and the
importance that this project had for Charles Babbage's calculating
machines.
\section{Did Adam Smith Invent the Computer?}
\label{sec:orgd3b731e}
A story that although was present for a long time, seems today relegated to
the confines of a few books on the history of computing. 

Test: this is an article from diderot encyclopedia:

A document from the imprimerie Firmin Didot describes the process.

\subsection*{References}
\printbibliography[heading=none,keyword=chapter-1]
\section{De Prony's Tables and Human Computers}
\label{sec:orga99870b}
This section describes de Prony' project in a holistic way. 

The documents used in this section are:

\subsection*{References}
\printbibliography[heading=none,keyword=chapter-1.1]
\section{Can Machine Labor Replace Human Labor?}
\label{sec:org78b52c3}
This section describes the reading that Charles Babbage makes of de Prony. 

\subsection*{References}
\printbibliography[heading=none,keyword=chapter-1.2]
\section{Chapter Conclusion}
\label{sec:org82e422f}
\chapter{Chapter 2}
\label{sec:orge211475}
\section{Herbert Simon Reads Babbage}
\label{sec:org9e2d3ce}
This section discusses Herbert Simon's reading of Charles Babbage. If
possible, we should look for Simon's reading of de Prony's project. After
all, his article from 1958 discusses Babbage's citing of the project in his
book on European manufactures.

This section should discuss how the description of the organization of work
as a simple technical matter has undemocratic undertones. 

It could be interesting to look at what he has written on the organization
of the workplace by machines and artificial intelligence. Should we be able
to draw a parallel between his writing on organization and his views on
democracy, we are all set. 

\subsection*{References}
\printbibliography[heading=none,keyword=chapter-2]
\section{Simon on Organization}
\label{sec:orgac0b1a3}
\subsection*{References}
\printbibliography[heading=none,keyword=chapter-2.1]
\section{Could Computers Be Democratic?}
\label{sec:org2ba0200}
Has Simon discussed the issue of democracy at the workplace besides the one
article on industrial democracy?

I want to talk about economic democracy as Alain Supiot does.  

\subsection*{References}
\printbibliography[heading=none,keyword=chapter-2.2]
\section{Chapter Conclusion}
\label{sec:org4ed2cea}
\printbibliography[heading=none,keyword=chapter-2.3]
\chapter{Conclusion}
\label{sec:org5d79c36}
\lipsum
\end{document}