% Intended LaTeX compiler: lualatex
\documentclass[version=last,draft=false,paper=A4,portrait,twoside=true,twocolumn=true,headinclude=false,footinclude=false,fontsize=12,BCOR=20mm,DIV=13,pagesize=auto,titlepage=firstiscover,mpinclude=false,open=right,chapterprefix=true,numbers=autoendperiod,headsepline=false,parskip=false]{scrbook}
\usepackage{graphicx}
\usepackage{grffile}
\usepackage{longtable}
\usepackage{wrapfig}
\usepackage{rotating}
\usepackage[normalem]{ulem}
\usepackage{amsmath}
\usepackage{textcomp}
\usepackage{amssymb}
\usepackage{capt-of}
\usepackage{hyperref}
\usepackage{polyglossia}
\usepackage[autostyle=true,english=american,french=guillemets,thresholdtype=words,threshold=3]{csquotes}
\usepackage{amsmath}
\usepackage{amsthm}
\usepackage{amssymb}
\usepackage{centernot}
\usepackage{hyperref}
\hypersetup{colorlinks,urlcolor=bibleblue,linkcolor=bibleblue,citecolor=bibleblue,filecolor=black}
\usepackage{balance}
\usepackage{array}
\usepackage{tabularx}
\usepackage{booktabs}
\usepackage[most]{tcolorbox}
\usepackage[french]{fmtcount}
\fmtcountsetoptions{french=france}
\usepackage[singlespacing]{setspace}
\usepackage[super]{nth}
\usepackage{ragged2e}
\usepackage[all]{nowidow}
\usepackage{enumitem}
\usepackage{adforn}
\usepackage{float}
\usepackage{titling}
\usepackage{xcolor}
\usepackage{graphicx}
\graphicspath{ {/home/sync0/Dropbox/paris_1/} }
\usepackage{tikz}
\usetikzlibrary{positioning}
\tikzset{main node/.style={circle,fill=gray!45,draw,minimum size=0.5cm,inner sep=0pt},}
\usepackage{lipsum}
\usepackage[backend=biber,bibstyle=draft,eprint=false,doi=false,isbn=false,url=false]{biblatex}
\addbibresource{~/Documents/pdfs/bibliography.bib}
\urlstyle{sf}
\renewcommand{\subtitlepunct}{\addcolon\addspace}
\renewbibmacro*{date}{\printdate\iffieldundef{origyear}{}{\setunit*{\addspace}\printtext[brackets]{\printorigdate}}}
\usepackage{scrlayer-scrpage}
\pagestyle{scrheadings}
\clearscrheadfoot
\automark[chapter]{part}
\cehead{\MakeLowercase{\thetitle}}
\cohead{\MakeLowercase{\headmark}}
\ohead{\pagemark}
\usepackage{fontspec}
\usepackage{unicode-math}
\usepackage[oldstyle]{libertine}
\defaultfontfeatures{Scale=MatchLowercase}
\setmonofont{Source Code Pro}
\setmathfont[Scale=MatchUppercase]{libertinusmath-regular.otf}
\newfontfamily{\titlefamily}[Scale=2]{Linux Biolinum O}
\newfontfamily{\sbfseries}[UprightFont={* Semibold}]{Linux Libertine O}
\newcommand\hugetitle{\fontsize{45}{50}\selectfont}
\newcommand\HUGE{\fontsize{40}{40}\selectfont}
\newcommand\hugechapter{\fontsize{30}{35}\selectfont}
\setkomafont{labelinglabel}{\normalsize\itshape}
\setkomafont{minisec}{\usekomafont{subsection}}
\setkomafont{pagehead}{\normalsize\mdseries\scshape}
\setkomafont{pagenumber}{\normalsize\rmfamily\upshape}
\setkomafont{sectioning}{\rmfamily\mdseries}
\setkomafont{caption}{\small}
\setkomafont{captionlabel}{\sffamily\mdseries\scshape\lowercase}
\setkomafont{chapter}{\hugechapter\rmfamily}
\renewcommand{\raggedchapter}{\centering}
\renewcommand*\chapterformat{\thechapter\autodot\par\enskip}
\RedeclareSectionCommand[afterskip=6\baselineskip]{chapter}
\setkomafont{section}{\Large\scshape\lowercase}
\setkomafont{subsection}{\large}
\setkomafont{subsubsection}{\large\itshape}
\AtBeginDocument{\renewcaptionname{english}\contentsname{Contents}}
\addtokomafont{chapterentry}{\mdseries\scshape\lowercase}
\setkomafont{chapterentrypagenumber}{\normalsize}
\usepackage{tocstyle}
\settocfeature{raggedhook}{\raggedright}
\selecttocstyleoption{tocgraduated}
\usetocstyle{nopagecolumn}
\newtcolorbox{modified}[1][]{grow to right by=0mm,grow to left by=-1em,boxrule=1pt,boxsep=0pt,breakable,enhanced jigsaw,borderline west={0pt}{0pt}{lightgrey},lower separated=false,arc=00mm,colframe=white, #1}
\newtcolorbox{note}[2][]{grow to right by=0mm,grow to left by=-1em,boxrule=0pt,boxsep=0pt,opacityback=0.0,breakable,parbox=false,enhanced jigsaw,borderline west={4pt}{0pt}{lightgrey},title={#2},coltitle={black},attach title to upper={},halign title=right,after title={\smallskip\par}#1}
\newtcolorbox{question}[2][]{grow to right by=0mm,grow to left by=-1em,boxrule=0pt,boxsep=0pt,opacityback=0.0,breakable,parbox=false,enhanced jigsaw,borderline west={4pt}{0pt}{darkgrey},title={#2},coltitle={black},attach title to upper={},halign title=right,after title={\smallskip\par}#1}
\newtcolorbox{definition}[3][]{grow to right by=0mm,grow to left by=-1em,boxrule=0pt,boxsep=0pt,opacityback=0.0,breakable,enhanced jigsaw,borderline west={4pt}{0pt}{midgrey},title={#2},coltitle={black},fonttitle={\sffamily\bfseries},fontupper={\normalsize},fontlower={\itshape},lower separated=false,attach title to upper={},after title={\hspace{1em}{\rmfamily\mdseries\itshape #3}\par}#1}
\renewcommand*\labelitemi{\adforn{33}}
\renewcommand*\labelitemii{\adforn{73}}
\renewcommand*\labelitemiii{\adforn{73}}
\renewcommand*\labelitemiv{\adforn{73}}
\definecolor{bibleblue}{HTML}{00339a}
\definecolor{whitegrey}{HTML}{f7f7f7}
\definecolor{lightgrey}{HTML}{cccccc}
\definecolor{midgrey}{HTML}{969696}
\definecolor{darkgrey}{HTML}{636363}
\definecolor{blackgrey}{HTML}{252525}
\newcommand{\notimplies}{\centernot\implies}
\setcounter{secnumdepth}{0}
\setcounter{tocdepth}{1}
\setlength{\columnsep}{0.5cm}
\setmainlanguage{english}
\setotherlanguages{french,italian,spanish}
\MakeOuterQuote{"}
\MakeForeignQuote{french}{«}{»}
\usepackage[protrusion=true,tracking=true]{microtype}
\author{Carlos Alberto Rivera Carreño}
\date{}
\title{}
\hypersetup{
 pdfauthor={Carlos Alberto Rivera Carreño},
 pdftitle={},
 pdfkeywords={},
 pdfsubject={},
 pdfcreator={Emacs 26.1 (Org mode 9.2.2)}, 
 pdflang={English}}
\begin{document}

\begin{titlepage}
 \centering
\begin{french}
 {\large \textsc{université paris i panthéon sorbonne} \par}
  \vspace*{0.01\textheight}
 {\large \textsc{ufr} 02 : Sciences économiques  \par}
  \vspace*{0.01\textheight}
 {\large Master 2 : Économie et sciences humaines \par}
  \vspace*{0.01\textheight}
 {\large 2018--2019 \par}
\end{french}
  \vspace*{0.3\textheight}
 {\huge \textsc{vers la machine à gouverner}  \par}
  \vspace*{0.02\textheight}
 {\Large Herbert Simon and the Impossibility of a Democratic Computer \par}
\vfill
\begin{french}
 {\large Présenté et sountenu par : \par}
\end{french}
 {\Large Carlos Alberto Rivera Carreño \par}
  \vspace*{0.05\textheight}
\begin{french}
 {\large Directeur de mémoire : \par}
\end{french}
 {\Large Jean-Sébastien Lenfant \par}
\end{titlepage}

\pagestyle{empty}

\begin{french}
L'Université Paris 1 Panthéon Sorbonne n'entend donner aucune approbation,
ni désapprobation aux opinions émises dans ce mémoire ; elle doivent être
considérées comme propres à leur auteur. 
\end{french}
\vfill

\newpage
\vspace*{\fill}
\noindent
\includegraphics[height=1.5cm]{gpl3.png}\par
\vspace{1\baselineskip}
This text is free: you can redistribute it and/or modify it
under the terms of the \textsc{gnu} General Public License as published by
the Free Software Foundation, either version 3 of the License or any later
version.

This text is distributed in the hope that it will be useful, but \textbf{without
any warranty}; without even the implied warranty of \textbf{merchantability or 
fitness for a particular purpose}. See the \textsc{gnu} General 
Public License for more details.

You should have received a copy of the \textsc{gnu} General Public License along
with this text. If not, see \url{http://www.gnu.org/licenses/}.

\vspace{1\baselineskip}
\noindent
Copyright \textcopyright \textsc{sync0} 2018. 

% \newpage 

% \newpage\null\newpage

% \begin{FlushRight}
% \begin{spanish}
% \textit{Al padre Camilo Torres.}
% \end{spanish}
% \end{FlushRight}

\newpage
\tableofcontents 
\nocite{*}

\chapter{Introduction}
\label{sec:org764ffd0}
\section{Accroche : Metaphors in Economics}
\label{sec:org778b8fe}
In the \emph{accroche}, I will describe how something as self-evident as labor is
in fact an ideological construct (ideological in the sense of the
anthropologist Louis Dumont, and not in the Marxist sense). Therefore, I
will briefly indicate that the concept of Labor begins in political economy
and is then appropriated by natural science to be reimported into economic
science.

Likewise, I will indicate that the motivation for the the thesis is to
rethink what Labor is in the \ordinalnum{21} century, taking into account the ubiquity
of information technologies. 

This section will introduce the topic of metaphors in economic thought,
since the main issue of the thesis is to provide historical and conceptual
elements to understand the transformations of labor in the age of
artificial intelligence and computer and information technologies (ICTs). 

This thesis takes the position that metaphors and analogies play an
important role in the way scientific concepts are understood such as in
pedagogical examples and functional analogies. 

I will state that both labor and the computer are not things but concepts,
and that their definitions must be approached historically. This chapter
will define the early computer as a conception of the organization of work
that reflects an engineering and managerial mentality. Likewise, this
chapter will hint at the link between this conception of work and its
definition in law as a relation of subordination.
In the last part of the introduction, I will summarize the arguments of the
three chapters and hint at the conclusion. 

\section{Rethinking Work in the 21st Century}
\label{sec:org8deb127}
In this section, I will state the main argument of the thesis: namely, that
the origin of the computer is a particular conception of the organization
of labor, which we should take into account to understand the transformation
of labor (its forms, its meaning, its formal definition in law, etc.) in the
\ordinalnum{21} century. As such, this argument calls into question the ``popular''
understanding of the one-sided transformation of labor by the appeareance
and dissemination of information and communication technologies (ICTs) in
the mid \ordinalnum{20} century. Instead, this thesis proposes to read the
computer as a technology for organizing labor, to then use this reading to
understand the transformation of labor that ICTs are supposedly pushing
for. 

\section{Structure of the Thesis}
\label{sec:org25c77eb}
The second chapter will describe Herbert Simon's interest in Babbage, and
will speculate on how this reading shaped Simon's conception of the
computer, the relation between the natural and the artificial, the changes
in work produced by new technologies (artificial intellgence, computers,
and automation), and the organization of work in society. Moreover, this
chapter will criticize Simon's idea that the organization of work is a
purely technical problem to propose an alternative view of work that
emphases other criteria for organization such as justice, etc.

\section*{References}
\printbibliography[heading=none,keyword=introduction]
\chapter{The Conceptual Origins of the Computer}
\label{sec:org2a83b13}
The first chapter introduces the definition of the early computer as a
technology for organizing work. The first objective of this chapter is to
question the belief that the appropriate definition of the computer is in
terms of its components. The second objective is to acquaint the reader
with the history of the early computer by describing the project of the
calculation of the logarithmic tables at the \emph{Bureau du cadastre}, and the
importance that this project had for Charles Babbage's calculating
machines.
\section{Did Adam Smith Invent the Computer?}
\label{sec:org46c2453}
This chapter will open with Simon and Newell's text,
\citetitle{simon_newell1958}, in which they discuss de Prony's project and it
is influence on Charles Babbage.

The introduction to this chapter presents the reader with the story of how
Gaspard-Clair-François-Marie Riche de Prony was inspired by Adam Smith's
concept of the division of labor---as it appears in the pin factory example
of the ``Wealth of Nations''---to organize a group of hairdressers to
produce mathematical tables for the French \emph{Bureau du cadastre}, during the
aftermath of the French Revolution. The point is to show that the
``computer'' is in fact an organization of labor, in which complex
calculation tasks are divided into simpler calculation tasks, which are
then carried out by unqualified ``specialized'' workers (or in computer
science lingo, by \emph{sub-processes}).

The first section of this chapter contextualizes the story of De Prony, by
providing some background information on him and on the project of the
calculation of the logarithmic tables. The idea is to provide an
understanding of the significance of the project at the time, and the
subsequent significance that it had for Charles Babbage.

\section{Babbage's Thinking Machines}
\label{sec:org53fa496}

The second chapter will discuss the question at the heart of Babbage's
project: Can machine labor replace human labor?

The second section of this chapter connects De Prony's story with Babbage's
design of the Difference and the Analytical Engine. The idea is to connect
Babbage's ideas on the organization of work and industry with his thought
on calculating devices. Therefore, the concept of \emph{mental labor} will be
discussed, as it relates to the \ordinalnum{20} century analogy between
mind and computer---which is key to the thought of Herbert Simon.   

\section{Chapter Conclusion}
\label{sec:org7a9c2b2}
\section*{References}
\printbibliography[heading=none,keyword=chapter-1]
\chapter{The Bureaucracy as Model Computer}
\label{sec:org01d435a}
The second chapter describes Herbert Simon's thinking on the computer and
organizations to trace the consequences of these ideas into his thinking
about the role of artificial systems in shaping the workplace and worker
self-determination. Therefore, this chapter will introduce the reader to
the importance of Herbert Simon to the field of artificial intelligence,
which has been more-or-less ignored by economists---who often only focus on
his concept of \emph{bounded rationality}.
\section{The Mind and the Computer}
\label{sec:org05c8ab0}
The introduction of this chapter discusses the ambiguous relation between
nature and artifice in the thought of Herbert Simon---specially, as it
manifests in his understanding of the mind as a computer. The idea is to
give the reader enough background information on Simon's general vision of
\emph{things} to, then, discuss his thinking on the nature of organizations (in
the first section).

\section{Dichotomy: Natural and Artificial}
\label{sec:org3de7faf}
The first section of this chapter connects Simon's ideas on the computer
and the mind to his general thinking on organizations. The idea is to pave
the way for a political understanding of Simon's more abstract writings on
organizations by presenting Simon's own thinking on the concrete social
consequences of his vision---which is done in the second section.

\section{Chapter Conclusion}
\label{sec:orgca0227c}
\section*{References}
\printbibliography[heading=none,keyword=chapter-2]
\chapter{Conclusion}
\label{sec:orgaa71b09}
\end{document}