% Intended LaTeX compiler: lualatex
\documentclass[version=last,draft=false,paper=A4,portrait,twoside=true,twocolumn=true,headinclude=false,footinclude=false,fontsize=12,BCOR=20mm,DIV=13,pagesize=auto,titlepage=false,mpinclude=false,open=right,chapterprefix=true,numbers=autoendperiod,headsepline=false,parskip=false]{scrbook}
\usepackage{graphicx}
\usepackage{grffile}
\usepackage{longtable}
\usepackage{wrapfig}
\usepackage{rotating}
\usepackage[normalem]{ulem}
\usepackage{amsmath}
\usepackage{textcomp}
\usepackage{amssymb}
\usepackage{capt-of}
\usepackage{hyperref}
\usepackage{polyglossia}
\usepackage[autostyle=true,english=american,french=guillemets,thresholdtype=words,threshold=3]{csquotes}
\AtBeginEnvironment{quote}{\itshape}
\AtBeginEnvironment{foreigndisplayquote}{\itshape}
\usepackage{amsmath}
\usepackage{amsthm}
\usepackage{amssymb}
\usepackage{centernot}
\usepackage{hyperref}
\hypersetup{colorlinks,urlcolor=bibleblue,linkcolor=bibleblue,citecolor=bibleblue,filecolor=black}
\usepackage{balance}
\usepackage{array}
\usepackage{tabularx}
\usepackage{booktabs}
\usepackage[most]{tcolorbox}
\usepackage[french]{fmtcount}
\fmtcountsetoptions{french=france}
\usepackage[singlespacing]{setspace}
\usepackage[super]{nth}
\usepackage{ragged2e}
\usepackage[all]{nowidow}
\usepackage{enumitem}
\usepackage{adforn}
\usepackage{float}
\usepackage{titling}
\usepackage{xcolor}
\usepackage{graphicx}
\graphicspath{ {/home/sync0/Dropbox/paris_1/} }
\usepackage{tikz}
\usetikzlibrary{positioning}
\tikzset{main node/.style={circle,fill=gray!45,draw,minimum size=0.5cm,inner sep=0pt},}
\usepackage{lipsum}
\usepackage[backend=biber,bibstyle=reading,defernumbers=true,eprint=false,abstract=false,library=false,file=false,entryhead=full,entrykey=false,annotation=false,doi=false,isbn=false,url=false]{biblatex}
\addbibresource{~/Documents/pdfs/bibliography.bib}
\renewbibmacro*{date}{\printdate\iffieldundef{origyear}{}{\setunit*{\addspace}\printtext[brackets]{\printorigdate}}}
\urlstyle{sf}
\usepackage{scrlayer-scrpage}
\pagestyle{scrheadings}
\clearscrheadfoot
\automark[chapter]{part}
\cehead{\MakeLowercase{\thetitle}}
\cohead{\MakeLowercase{\headmark}}
\ohead{\pagemark}
\usepackage{fontspec}
\usepackage{unicode-math}
\usepackage[oldstyle]{libertine}
\defaultfontfeatures{Scale=MatchLowercase}
\setmonofont{Source Code Pro}
\setmathfont[Scale=MatchUppercase]{libertinusmath-regular.otf}
\newfontfamily{\titlefamily}[Scale=2]{Linux Biolinum O}
\newfontfamily{\sbfseries}[UprightFont={* Semibold}]{Linux Libertine O}
\newcommand\hugetitle{\fontsize{45}{50}\selectfont}
\newcommand\HUGE{\fontsize{40}{40}\selectfont}
\newcommand\hugechapter{\fontsize{30}{35}\selectfont}
\AfterTOCHead{\singlespacing}
\setkomafont{labelinglabel}{\normalsize\itshape}
\setkomafont{minisec}{\usekomafont{subsection}}
\setkomafont{pagehead}{\normalsize\mdseries\scshape}
\setkomafont{pagenumber}{\normalsize\rmfamily\upshape}
\setkomafont{sectioning}{\rmfamily\mdseries}
\setkomafont{caption}{\small}
\setkomafont{captionlabel}{\sffamily\mdseries\scshape\lowercase}
\setkomafont{chapter}{\hugechapter\rmfamily}
\renewcommand{\raggedchapter}{\centering}
\renewcommand*\chapterformat{\thechapter\autodot\par\enskip}
\RedeclareSectionCommand[afterskip=5\baselineskip]{chapter}
\setkomafont{section}{\Large\scshape\lowercase}
\setkomafont{subsection}{\large}
\setkomafont{subsubsection}{\large\itshape}
\newtcolorbox{modified}[1][]{grow to right by=0mm,grow to left by=-1em,boxrule=1pt,boxsep=0pt,breakable,enhanced jigsaw,borderline west={0pt}{0pt}{lightgrey},lower separated=false,arc=00mm,colframe=white, #1}
\newtcolorbox{note}[2][]{grow to right by=0mm,grow to left by=-1em,boxrule=0pt,boxsep=0pt,opacityback=0.0,breakable,parbox=false,enhanced jigsaw,borderline west={4pt}{0pt}{lightgrey},title={#2},coltitle={black},fonttitle={\sffamily},attach title to upper={},halign title=right,after title={\smallskip\par}#1}
\newtcolorbox{question}[2][]{grow to right by=0mm,grow to left by=-1em,boxrule=0pt,boxsep=0pt,opacityback=0.0,breakable,parbox=false,enhanced jigsaw,borderline west={4pt}{0pt}{darkgrey},title={#2},coltitle={black},fonttitle={\sffamily},attach title to upper={},halign title=right,after title={\smallskip\par}#1}
\newtcolorbox{definition}[3][]{grow to right by=0mm,grow to left by=-1em,boxrule=0pt,boxsep=0pt,opacityback=0.0,breakable,enhanced jigsaw,borderline west={4pt}{0pt}{midgrey},title={#2},coltitle={black},fonttitle={\sffamily\bfseries},fontupper={\normalsize},fontlower={\itshape},lower separated=false,attach title to upper={},after title={\hspace{1em}{\rmfamily\mdseries\itshape #3}\par}#1}
\renewcommand*\labelitemi{\adforn{33}}
\renewcommand*\labelitemii{\adforn{73}}
\renewcommand*\labelitemiii{\adforn{73}}
\renewcommand*\labelitemiv{\adforn{73}}
\definecolor{bibleblue}{HTML}{00339a}
\definecolor{whitegrey}{HTML}{f7f7f7}
\definecolor{lightgrey}{HTML}{cccccc}
\definecolor{midgrey}{HTML}{969696}
\definecolor{darkgrey}{HTML}{636363}
\definecolor{blackgrey}{HTML}{252525}
\newcommand{\notimplies}{\centernot\implies}
\setcounter{secnumdepth}{0}
\setcounter{tocdepth}{0}
\setlength{\columnsep}{0.5cm}
\setmainlanguage{english}
\setotherlanguages{french,italian,spanish}
\MakeOuterQuote{"}
\MakeForeignQuote{french}{«}{»}
\usepackage[protrusion=true,tracking=true]{microtype}
\author{Carlos Alberto Rivera Carreño}
\date{}
\title{Vers la machine à gouverner}
\hypersetup{
 pdfauthor={Carlos Alberto Rivera Carreño},
 pdftitle={Vers la machine à gouverner},
 pdfkeywords={},
 pdfsubject={},
 pdfcreator={Emacs 26.1 (Org mode 9.2.2)}, 
 pdflang={English}}
\begin{document}

\maketitle
\nocite{*}
\chapter{Introduction}
\label{sec:org2bedf22}
\section{Accroche: ``The Concept of Work between Society and Nature''}
\label{sec:org2308320}
In the \emph{accroche}, I will describe how something as self-evident as the work
concept is in fact an ideological construct (ideological in the sense of
the anthropologist Louis Dumont, and not in the Marxist sense). Therefore,
I will briefly indicate that the concept of work begins in political
economy and is then appropriated by natural science to be reimported into
economic science.  

Likewise, I will indicate that the motivation for the the thesis is to
rethink what work is in the \ordinalnum{21} century, taking into account the ubiquity
of information technologies. 

\printbibliography[heading=none,keyword=introduction-1]
\section{Main Argument: ``Rethinking Work in the 21st Century''}
\label{sec:org85454dd}
In this section, I will state the main argument of the thesis: namely, that
the origin of the computer is a particular conception of the organization
of work, which we should take into account to understand the transformation
of work (its forms, its meaning, its formal definition in law, etc.) in the
\ordinalnum{21} century. As such, this argument calls into question the ``popular''
understanding of the one-sided transformation of work by the appeareance
and dissemination of information and communication technologies (ICTs) in
the mid \ordinalnum{20} century. Instead, this thesis proposes to read the
computer as a technology for organizing labor, to then use this reading to
understand the transformation of labor that ICTs are supposedly pushing
for. 

\printbibliography[heading=none,keyword=introduction-2]
\section{Structure of the Thesis}
\label{sec:orgc4926e9}
In this section, I simply describe the contents of the two chapters of the
thesis and hint at the conclusion. 

The first chapter traces the origins of the computer in de Prony and
Babbage. The idea is to provide the reader with enough background knowledge
to understand that the computer is not a thing but a concept, and that the
definition of this concept has to be approached historically. This chapter
will define the early computer as a conception of the organization of work
that reflects an engineering and managerial mentality. Likewise, this
chapter will hint at the link between this conception of work and its
definition in law as a relation of subordination.

The second chapter will describe Herbert Simon's interest in Babbage, and
will speculate on how this reading shaped Simon's conception of the
computer, the relation between the natural and the artificial, the changes
in work produced by new technologies (artificial intellgence, computers,
and automation), and the organization of work in society. Moreover, this
chapter will criticize Simon's idea that the organization of work is a
purely technical problem to propose an alternative view of work that
emphases other criteria for organization such as justice, etc.

\section*{References}
\printbibliography[heading=none,keyword=introduction]
\chapter{Chapter 1}
\label{sec:org75e08b6}
The first chapter introduces the definition of the early computer as a
technology for organizing work. The first objective of this chapter is to
question the belief that the appropriate definition of the computer is in
terms of its components. The second objective is to acquaint the reader
with the history of the early computer by describing the project of the
calculation of the logarithmic tables at the \emph{Bureau du cadastre}, and the
importance that this project had for Charles Babbage's calculating
machines.
\section{Did Adam Smith Invent the Computer?}
\label{sec:org161fe11}
The introduction to this chapter presents the reader with the story of how
Gaspard-Clair-François-Marie Riche de Prony was inspired by Adam Smith's
concept of the division of labor---as it appears in the pin factory example
of the ``Wealth of Nations''---to organize a group of hairdressers to
produce mathematical tables for the French \emph{Bureau du cadastre}, during the
aftermath of the French Revolution. The point is to show that the
``computer'' is in fact an organization of labor, in which complex
calculation tasks are divided into simpler calculation tasks, which are
then carried out by unqualified ``specialized'' workers (or in computer
science lingo, by \emph{sub-processes}).

\subsection*{References:}
\printbibliography[heading=none,keyword=chapter-1]
\section{De Prony's Tables and Human Computers}
\label{sec:org604100f}
The first section of this chapter contextualizes the story of De Prony, by
providing some background information on him and on the project of the
calculation of the logarithmic tables. The idea is to provide an
understanding of the significance of the project at the time, and the
subsequent significance that it had for Charles Babbage.

\subsection*{References:}
\printbibliography[heading=none,keyword=chapter-1.1]
\section{Can Machine Labor Replace Human Labor?}
\label{sec:org47eeb70}
The second section of this chapter connects De Prony's story with Babbage's
design of the Difference and the Analytical Engine. The idea is to connect
Babbage's ideas on the organization of work and industry with his thought
on calculating devices. Therefore, the concept of \emph{mental labor} will be
discussed, as it relates to the \ordinalnum{20} century analogy between
mind and computer---which is key to the thought of Herbert Simon.   

\subsection*{References:}
\printbibliography[heading=none,keyword=chapter-1.2]
\section{Chapter Conclusion}
\label{sec:org41cb18c}
\chapter{Chapter 2}
\label{sec:orge6ac0cc}
The second chapter describes Herbert Simon's thinking on the computer and
organizations to trace the consequences of these ideas into his thinking
about the role of artificial systems in shaping the workplace and worker
self-determination. Therefore, this chapter will introduce the reader to
the importance of Herbert Simon to the field of artificial intelligence,
which has been more-or-less ignored by economists---who often only focus on
his concept of \emph{bounded rationality}.

\section{Herbert Simon: Computer as Mind; Mind as Computer}
\label{sec:orge4cbc35}
The introduction of this chapter discusses the ambiguous relation between
nature and artifice in the thought of Herbert Simon---specially, as it
manifests in his understanding of the mind as a computer. The idea is to
give the reader enough background information on Simon's general vision of
\emph{things} to, then, discuss his thinking on the nature of organizations (in
the first section).

\subsection*{References:}
\printbibliography[heading=none,keyword=chapter-2]
\section{Simon on Organizations}
\label{sec:org875cff7}
The first section of this chapter connects Simon's ideas on the computer
and the mind to his general thinking on organizations. The idea is to pave
the way for a political understanding of Simon's more abstract writings on
organizations by presenting Simon's own thinking on the concrete social
consequences of his vision---which is done in the second section.

\subsection*{References:}
\printbibliography[heading=none,keyword=chapter-2.1]
\section{Could Computers Be Democratic?}
\label{sec:org8ed4785}
The second section of this chapter connects Simon's general ideas on
organizations to his thought on the role of artificial systems (machines,
automation, ICTs, artificial intelligence, etc.) in shaping the workplace.
Moreover, this section will discuss his problematic concept of a ``science
of the artificial'' as it relates to the organization of labor---after all,
there's nothing more artificial than the institution of wage earning. The
idea is to show that the description of the organization of labor as a
purely technical matter has terrible consequences for worker
self-determination, and that this understanding of the worker as a just a
cog in the production process is not questioned by Simon, and thus,
unlikely to change under the new information technologies.

\subsection*{References:}
\printbibliography[heading=none,keyword=chapter-2.2]
\section{Chapter Conclusion}
\label{sec:orgeab0b8e}
\printbibliography[heading=none,keyword=chapter-2.3]
\chapter{Conclusion}
\label{sec:org6e2cd91}
\end{document}