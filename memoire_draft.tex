% Intended LaTeX compiler: pdflatex
\documentclass[draft=false, paper=A4,portrait,twoside=true,twocolumn=false,headinclude=false,footinclude=false,fontsize=11,BCOR=15mm,DIV=calc,pagesize=auto,titlepage=firstiscover,mpinclude=true,headings=big,headings=twolinechapter,open=right,chapterprefix=false,headsepline=false,parskip=full]{scrbook}
\usepackage{fontspec}
\usepackage{xunicode}
\usepackage{url}
\usepackage{soul}
\usepackage{polyglossia}
\setmainlanguage{english}
\setotherlanguages{italian,spanish,french}
\usepackage[french=guillemets,thresholdtype=words,threshold=3]{csquotes}
\MakeAutoQuote{«}{»}
\AtBeginEnvironment{quote}{\itshape}
\usepackage[backend=biber,style=authoryear,doi=false,isbn=false,url=false]{biblatex}
\DeclareFieldFormat{booktitle}{\color{blue}\mkbibemph{#1}}
\DeclareFieldFormat{title}{\color{blue}\mkbibemph{#1}}
\addbibresource{~/Documents/mendeley/library.bib}
\usepackage{amsmath}
\usepackage{amsthm}
\usepackage{amssymb}
\usepackage{centernot}
\usepackage{hyperref}
\hypersetup{colorlinks,urlcolor=blue,linkcolor=red,citecolor=red,filecolor=black}
\usepackage{booktabs}
\usepackage[english]{fmtcount}
\usepackage[singlespacing]{setspace}
\usepackage[super]{nth}
\usepackage{microtype}
\usepackage{ragged2e}
\usepackage{enumitem}
\usepackage{adforn}
\usepackage{float}
\usepackage{xcolor}
\usepackage{graphicx}
\graphicspath{ {/home/sync0/Dropbox/paris_1/} }
\usepackage{lipsum}
\usepackage[textsize=scriptsize, linecolor=soothing_green, backgroundcolor=soothing_green]{todonotes}
\usepackage{xunicode}
\usepackage{fontspec}
\usepackage{xltxtra}
\defaultfontfeatures{Scale=MatchLowercase}
\setmainfont[Mapping=tex-text,Numbers=OldStyle,SmallCapsFeatures={LetterSpace=4,Ligatures=NoCommon}]{Linux Libertine O}
\setsansfont[Mapping=tex-text]{Linux Biolinum O}
\setmonofont[Mapping=tex-text]{Inconsolata}
\newfontfamily\titlefamily[Scale=2]{Linux Biolinum O}
\newcommand\HUGE{\fontsize{30}{30}\selectfont}
\usepackage{scrlayer-scrpage}
\pagestyle{scrheadings}
\clearscrheadfoot
\automark[chapter]{part}
\cehead{\headmark}
\cohead{\headmark}
\lehead{\thepage}
\rohead{\thepage}
\renewcommand\partmarkformat{}
\AfterTOCHead{\singlespacing}
\setkomafont{disposition}{\normalfont\normalcolor}
\setkomafont{labelinglabel}{\normalfont\bfseries}
\setkomafont{minisec}{\usekomafont{subsection}}
\addtokomafont{pageheadfoot}{\sffamily\upshape}
\addtokomafont{caption}{\small}
\addtokomafont{captionlabel}{\bfseries}
\addtokomafont{part}{\HUGE\scshape\sffamily\lowercase}
\renewcommand*{\partformat}{}
\addtokomafont{chapter}{\HUGE\scshape\sffamily\lowercase}
\renewcommand{\raggedchapter}{\centering}
\RedeclareSectionCommand[beforeskip=0cm,afterskip=1.5cm]{chapter}
\addtokomafont{section}{\huge\scshape\sffamily\setstretch{0.7}\lowercase}
\addtokomafont{subsection}{\sffamily\Large}
\addtokomafont{subsubsection}{\scshape\sffamily\Large\lowercase}
\addtokomafont{chapterentry}{\normalsize\sffamily\bfseries}
\usepackage[tocindentauto,tocgraduated]{tocstyle}
\usetocstyle{nopagecolumn}
\renewcommand*{\addparttocentry}[2]{\addtocentrydefault{part}{\protect\sffamily\Large\scshape\lowercase{#1}\hspace{1em}}{#2}}
\renewcommand*\labelitemi{\adforn{33}}
\renewcommand*\labelitemii{\adforn{73}}
\renewcommand*\labelitemiii{\adforn{73}}
\renewcommand*\labelitemiv{\adforn{73}}
\definecolor{soothing_green}{HTML}{E1F7DB}
\setcounter{secnumdepth}{\partnumdepth}
\setcounter{tocdepth}{2}
\recalctypearea
\setlength{\marginparwidth}{2\marginparwidth}
\author{Carlos Alberto Rivera Carreño}
\date{}
\title{}
\hypersetup{
 pdfauthor={Carlos Alberto Rivera Carreño},
 pdftitle={},
 pdfkeywords={},
 pdfsubject={},
 pdfcreator={Emacs 26.1 (Org mode 9.1.14)}, 
 pdflang={English}}
\begin{document}

\begin{titlepage}
 \centering
% \includegraphics[width=0.5\textwidth]{logo_noir_fr.png}\par
 \vspace{4\baselineskip}
\begin{french}
 {\Large Université Paris I Panthéon Sorbonne \par}
 {\Large \textsc{ufr} 02 : Sciences économiques  \par}
 {\large Master 2 : Économie et sciences humaines \par}
 {\large 2019 \par}
\end{french}
 \vspace{2\baselineskip}
 {\huge The Automata \& the Engineer \par}
 {\Large Herbert Simon's Quest for the Governing Machine \par}
\vspace*{\fill}
\begin{french}
 {\large Présenté et sountenu par : \par}
\end{french}
 {\large \textsc{carlos alberto rivera carreño}\par}
 \vspace{1\baselineskip}
\begin{french}
 {\large Directeur de mémoire : \par}
\end{french}
 {\large \textsc{jean sébastien lenfant}\par}
\end{titlepage}

\pagestyle{empty}

\begin{french}
L'Université Paris 1 Panthéon Sorbonne n'entend donner aucune approbation,
ni désapprobation aux opinions émises dans ce mémoire ; elle doivent être
considérées comme propres à leur auteur. 
\end{french}

\newpage
\vspace*{\fill}
\noindent
\includegraphics[height=1.5cm]{gpl3.png}\par
\vspace{1\baselineskip}
\begin{english}
This text is free: you can redistribute it and/or modify it
under the terms of the \textsc{gnu} General Public License as published by
the Free Software Foundation, either version 3 of the License or any later
version.

This text is distributed in the hope that it will be useful, but \textbf{without
any warranty}; without even the implied warranty of \textbf{merchantability or 
fitness for a particular purpose}. See the \textsc{gnu} General 
Public License for more details.

You should have received a copy of the \textsc{gnu} General Public License along
with this text. If not, see \url{http://www.gnu.org/licenses/}.

\vspace{1\baselineskip}
\noindent
Copyright \textcopyright \textsc{sync0} 2018. 
\end{english}

\newpage 

\begin{FlushRight}
\begin{spanish}
\textit{Para una lectora lejana.}
\end{spanish}
\end{FlushRight}

\newpage
\tableofcontents 

\frontmatter
\pagestyle{plain}
\chapter{Acknowledgements} 

\lipsum

\chapter{Preface} 

Even when applying different techniques of interpretation to texts, I have
tried to understand how the circumstances of their production could inform
their interpretation. Even just for the sake of consistency, shouldn't I
apply the same standard to this Master's thesis? Shouldn't I provide the
reader with the tools, the context, etc. that informed this research?

Pace Barthes, the author of these words is very alive. Knowing the tragedy
of my country, one might ask me: Why do I wield the pen, as the Colombian
fields bleed? Why not wield the sword? Or, perhaps, why not wield both?
This text was written, certainly, to satisfy an earthly requirement: to
obtain a Master's degree. But beyond that, what is the interest of writing
for me?

Although this thesis is not a political pamphlet, it is grounded in
political perplexities. To the recule of the state, who not only does not
want to fund nor organize vast sections of economic activities, my
generations witnesses a strange mix of ``innovation'',
``entrepreneurship'', and . If the political Left, of my parent's
generation had an iron faith in the working class and the peasants to
change society, today's politically dissafected youth has all hopes in
private enterprise to solve the evils of the world.

As Oscar Wilde once remarked, ``it is easier to have sympathy with
suffering, than it is to have sympathy with thought''. Even thought,

As much as this document is intended to an academic audience, it is also
intended to them and to my father: This is the beginning of a lifelong
attempt to explain \emph{¿Cómo fue que se jodio el país?} 

Should they be right, this article is wrong. 

\mainmatter
\pagestyle{scrheadings}
\part{Preparatory Research}
\label{sec:org189a66c}
\chapter{Draft: Framework}
\label{sec:orge777747}
\section{Main question}
\label{sec:orge61f8be}
In this thesis, I will trace the genealogy of Simon's ideas on the computer
and computation to discuss their influence for his views on automation
of decision making. 
\section{Motivation}
\label{sec:orge5bd378}
The motivation for this thesis is a deep feeling that there is something
wrong with today's society and that given that economics plays such an
important cultural role in mediating people's understanding of the nature
of political problems, it is necessary to understand what is wrong with
economics to understand what is wrong with society at large. 

Given this, I am interested in the context in which economic ideas
originate; I am interested in the interactions between economic ideas,
ideas in other sciences, utopian political ideas, and wider ``pop-culture''
understandings and receptions of those ideas.

In today's society, in which the future utopia is construed as the
apotheosis of technological revolution---and \emph{not} political revolution---,
and in which the utopian discourse of trans-humanism joins corporate
discourses on innovation and neoliberal panegyrics on entrepreneurship, we
can see a relation between a certain conception of politics and the place
and role that machines will play there, and in getting us there.

In the case of economics, following the intellectual debacle of the 2008
global financial crisis, in which those economists who gleefully took 
refuge in baroque mathematics were scorned for the empiric irrelevance of
their theories, behavioral economics was presented as an alternative
version to mainstream economics that would finally present a more realistic
version of economics, by integrating certain ideas from Psychology, to
better understand economic problems, or more simply, reality.  

The reason for writing a thesis on Henry Simon is \emph{not} to write another
thesis on bounded rationality. Rather, the reason for focusing on Simon is
because he was one of the first economists to embrace the postwar cyborg
sciences, and because he played a key role in the foundation of the field
of artificial intelligence. Given that artificial intelligence presents
itself as the next holy grail of science in our generation, I think that it
is worth to study Simon, due to the foundational role he played.

Particularly, I am interested in Simon because he was more \emph{cyborg} than
other postwar economists. Even though the economists close to the Cowles
Commission embraced certain aspects of the cyborg sciences, such as a
version of Game Theory, they were more conservative in their willingness to
question to the Walrasian model, to which Simon wasn't attached. 

In the end, Simon distanced himself from mainstream economics, to build a
project of the sciences of the artificial, in which new modes of organizing
society were envisioned based on his theories on artificial intelligence. 

I believe that one aspect worth discussing of Simon is an ontological
understanding of his idea of nature, society, the human, and the artificial. 
Being a cyborg, these domains blur in his work, and I believe this has
important consequences for his vision of organization, and its political
and economic consequences. 

Since one of the important aspects of his vision is his idea of the
computer, this thesis will explore Simon's vision of the computer. Thus the
title of the thesis: The computer in the thought of Henry Simon.

The idea is to project this thesis into a Ph.D dissertation around the
theme of the governing machine: the automation of political decision making
and the delegation of all political responsibility to machines.

 I am interested in
the wider cultural influence of the ideas of economists.

Particularly, today's futuristic utopias, in which technological change,
but not political change, is wished seem to me to be very problematic. c

Generally speaking, I am interested in ideas; specially, in the way that
people come to understand and define their political utopias, and their
definitions of nature and society.  

Roughly speaking, with this thesis, I expect to understand better the
épistémè of our time.

\section{Thesis Outline}
\label{sec:org7345118}
The thesis will have the following outline:

\begin{itemize}
\item Introduction
\item Chapter 1:
\begin{itemize}
\item Section 1
\item Section 2
\end{itemize}
\item Chapter 2:
\begin{itemize}
\item Section 1
\item Section 2
\end{itemize}
\item Conclusion
\end{itemize}

Next, I will discuss the content of each of these parts. 
\subsection{Introduction (15-20 pages)}
\label{sec:org4c58124}
In the introduction, I will discuss the importance of economic metaphors
\subsection{The importance of the computer in economic though? (15-20)}
\label{sec:org37d4521}
I will discuss the importance of the computer for economic theory, specially in the work of Simon. 

I might being by discussing Jevons's inspiration in the work of Charles
Babbage.
\subsubsection{The genealogy of Computer (Metaphor) in Simon's Work}
\label{sec:orgb04a3e9}
In this chapter, I will trace the genealogy of the computer concept in
Simon's work, and how it evolved. 
\subsubsection{Section 2}
\label{sec:org634d4de}
\subsection{The influence of the computer for Simon's thought on automation (15-20)}
\label{sec:orgdf6be8b}
In this chapter, I will discuss the influence of the computer concept in
Simon's work, particularly in its relation with Simon's ideas on
automation.
\subsubsection{Section 1}
\label{sec:orge4f28e8}
\subsubsection{Section 2}
\label{sec:org40508e6}
\subsection{Conclusion (10-15)}
\label{sec:orga613438}
\section{Methodology}
\label{sec:org625faa6}
I would like to write a history of ideas in the style of Philip Mirowski. 
\section{Bibliography}
\label{sec:org37e55d9}
This is a thematic bibliography organized around keywords. To facilitate
reading, the book titles are highlighted in blue. 

\nocite{*}
\subsection{The Brain}
\label{sec:org57c41fc}
\printbibliography[heading=none,keyword=memoire,keyword=brain]
\subsection{The Cold War}
\label{sec:orge1e5205}
\printbibliography[heading=none,keyword=memoire,keyword=cold-war,notkeyword=brain]
\subsection{The Computer}
\label{sec:orgc34a5a2}
\printbibliography[heading=none,keyword=memoire,keyword=computer,notkeyword=brain,notkeyword=cold-war]
\subsection{Metaphors}
\label{sec:org1d03fa9}
\printbibliography[heading=none,keyword=memoire,keyword=metaphors,notkeyword=brain,notkeyword=cold-war,notkeyword=computer,notkeyword=cyborg]
\subsection{Herbert Simon}
\label{sec:org5aa9b3e}
\printbibliography[heading=none,keyword=memoire,keyword=herbert-simon,notkeyword=brain,notkeyword=cold-war,notkeyword=computer,notkeyword=cyborg,notkeyword=metaphors]
\subsection{The Social Sciences}
\label{sec:orgf0eef57}
\printbibliography[heading=none,keyword=memoire,keyword=social-science,notkeyword=brain,notkeyword=computer,notkeyword=cyborg,notkeyword=metaphors,notkeyword=herbert-simon,notkeyword=cold-war]
\end{document}