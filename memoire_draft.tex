% Intended LaTeX compiler: pdflatex
\documentclass[draft=false, paper=A4,portrait,twoside=true,twocolumn=false,headinclude=false,footinclude=false,fontsize=11,BCOR=15mm,DIV=calc,pagesize=auto,titlepage=firstiscover,mpinclude=true,headings=big,headings=twolinechapter,open=right,chapterprefix=false,headsepline=false,parskip=full]{scrbook}
\usepackage{fontspec}
\usepackage{xunicode}
\usepackage{url}
\usepackage{soul}
\usepackage{polyglossia}
\setmainlanguage{english}
\setotherlanguages{italian,spanish,french}
\usepackage[french=guillemets,thresholdtype=words,threshold=3]{csquotes}
\MakeAutoQuote{«}{»}
\AtBeginEnvironment{quote}{\itshape}
\usepackage[backend=biber,style=authoryear,doi=false,isbn=false,url=false]{biblatex}
\DeclareFieldFormat{booktitle}{\color{blue}\mkbibemph{#1}}
\DeclareFieldFormat{title}{\color{blue}\mkbibemph{#1}}
\addbibresource{~/Documents/mendeley/library.bib}
\usepackage{amsmath}
\usepackage{amsthm}
\usepackage{amssymb}
\usepackage{centernot}
\usepackage{hyperref}
\hypersetup{colorlinks,urlcolor=blue,linkcolor=red,citecolor=red,filecolor=black}
\usepackage{booktabs}
\usepackage[english]{fmtcount}
\usepackage[singlespacing]{setspace}
\usepackage[super]{nth}
\usepackage{microtype}
\usepackage{ragged2e}
\usepackage{enumitem}
\usepackage{adforn}
\usepackage{float}
\usepackage{xcolor}
\usepackage{graphicx}
\graphicspath{ {/home/sync0/Dropbox/paris_1/} }
\usepackage{lipsum}
\usepackage[textsize=scriptsize, linecolor=soothing_green, backgroundcolor=soothing_green]{todonotes}
\usepackage{xunicode}
\usepackage{fontspec}
\usepackage{xltxtra}
\defaultfontfeatures{Scale=MatchLowercase}
\setmainfont[Mapping=tex-text,Numbers=OldStyle,SmallCapsFeatures={LetterSpace=4,Ligatures=NoCommon}]{Linux Libertine O}
\setsansfont[Mapping=tex-text]{Linux Biolinum O}
\setmonofont[Mapping=tex-text]{Inconsolata}
\newfontfamily\titlefamily[Scale=2]{Linux Biolinum O}
\newcommand\HUGE{\fontsize{30}{30}\selectfont}
\usepackage{scrlayer-scrpage}
\pagestyle{scrheadings}
\clearscrheadfoot
\automark[chapter]{part}
\cehead{\headmark}
\cohead{\headmark}
\lehead{\thepage}
\rohead{\thepage}
\renewcommand\partmarkformat{}
\AfterTOCHead{\singlespacing}
\setkomafont{disposition}{\normalfont\normalcolor}
\setkomafont{labelinglabel}{\normalfont\bfseries}
\setkomafont{minisec}{\usekomafont{subsection}}
\addtokomafont{pageheadfoot}{\sffamily\upshape}
\addtokomafont{caption}{\small}
\addtokomafont{captionlabel}{\bfseries}
\addtokomafont{part}{\HUGE\scshape\sffamily\lowercase}
\renewcommand*{\partformat}{}
\addtokomafont{chapter}{\HUGE\scshape\sffamily\lowercase}
\renewcommand{\raggedchapter}{\centering}
\RedeclareSectionCommand[beforeskip=0cm,afterskip=1.5cm]{chapter}
\addtokomafont{section}{\huge\scshape\sffamily\setstretch{0.7}\lowercase}
\addtokomafont{subsection}{\sffamily\Large}
\addtokomafont{subsubsection}{\scshape\sffamily\Large\lowercase}
\addtokomafont{chapterentry}{\normalsize\sffamily\bfseries}
\usepackage[tocindentauto,tocgraduated]{tocstyle}
\usetocstyle{nopagecolumn}
\renewcommand*{\addparttocentry}[2]{\addtocentrydefault{part}{\protect\sffamily\Large\scshape\lowercase{#1}\hspace{1em}}{#2}}
\renewcommand*\labelitemi{\adforn{33}}
\renewcommand*\labelitemii{\adforn{73}}
\renewcommand*\labelitemiii{\adforn{73}}
\renewcommand*\labelitemiv{\adforn{73}}
\definecolor{soothing_green}{HTML}{E1F7DB}
\setcounter{secnumdepth}{\partnumdepth}
\setcounter{tocdepth}{2}
\recalctypearea
\setlength{\marginparwidth}{2\marginparwidth}
\author{Carlos Alberto Rivera Carreño}
\date{}
\title{}
\hypersetup{
 pdfauthor={Carlos Alberto Rivera Carreño},
 pdftitle={},
 pdfkeywords={},
 pdfsubject={},
 pdfcreator={Emacs 26.1 (Org mode 9.1.14)}, 
 pdflang={English}}
\begin{document}

\begin{titlepage}
 \centering
% \includegraphics[width=0.5\textwidth]{logo_noir_fr.png}\par
 \vspace{4\baselineskip}
\begin{french}
 {\Large Université Paris I Panthéon Sorbonne \par}
 {\Large \textsc{ufr} 02 : Sciences économiques  \par}
 {\large Master 2 : Économie et sciences humaines \par}
 {\large 2019 \par}
\end{french}
 \vspace{2\baselineskip}
 {\huge The Automata \& the Engineer \par}
 {\Large Herbert Simon's Quest for the Governing Machine \par}
\vspace*{\fill}
\begin{french}
 {\large Présenté et sountenu par : \par}
\end{french}
 {\large \textsc{carlos alberto rivera carreño}\par}
 \vspace{1\baselineskip}
\begin{french}
 {\large Directeur de mémoire : \par}
\end{french}
 {\large \textsc{jean-sébastien lenfant}\par}
\end{titlepage}

\pagestyle{empty}

\begin{french}
L'Université Paris 1 Panthéon Sorbonne n'entend donner aucune approbation,
ni désapprobation aux opinions émises dans ce mémoire ; elle doivent être
considérées comme propres à leur auteur. 
\end{french}

\newpage
\vspace*{\fill}
\noindent
\includegraphics[height=1.5cm]{gpl3.png}\par
\vspace{1\baselineskip}
\begin{english}
This text is free: you can redistribute it and/or modify it
under the terms of the \textsc{gnu} General Public License as published by
the Free Software Foundation, either version 3 of the License or any later
version.

This text is distributed in the hope that it will be useful, but \textbf{without
any warranty}; without even the implied warranty of \textbf{merchantability or 
fitness for a particular purpose}. See the \textsc{gnu} General 
Public License for more details.

You should have received a copy of the \textsc{gnu} General Public License along
with this text. If not, see \url{http://www.gnu.org/licenses/}.

\vspace{1\baselineskip}
\noindent
Copyright \textcopyright \textsc{sync0} 2018. 
\end{english}

\newpage 

\begin{FlushRight}
\begin{italian}
% \textit{Para una lectora lejana.}
% \textit{Per il professore Giorgio Israel. \newline Mi dispiace, ho mancato il nostro incontro. \linebreak Sono in ritardo, come la coscienza della nostra generazione. \linebreak I campi sanguinano. Tutti lo sanno, ma a nessuno importa.}
\textit{Per il professore Giorgio Israel. \newline Benché i nostri destini fossero uniti da quell'anno fatale del 1492, \linebreak ho mancato il nostro incontro. \linebreak Sono arrivato in ritardo, come la coscienza della nostra generazione. \linebreak I campi sanguinanti sono così prossimi \linebreak che le gocce cospargerebbero gli occhi. \linebreak Eppure, nessuno vede niente. \linebreak Tante informazione, ma così poca conoscenza.}
\end{italian}
\end{FlushRight}

\newpage
\tableofcontents 

\frontmatter
\pagestyle{plain}
\chapter{Acknowledgements} 

\lipsum

\chapter{Preface} 

\mainmatter
\pagestyle{scrheadings}
\part{Preparatory Research}
\label{sec:orgc31c3c8}
\chapter{Draft: Framework}
\label{sec:org532f47d}
\section{Subject-matter}
\label{sec:orgb6ebfbb}
In this thesis, I will trace the genealogy of Simon's ideas on the computer
and computation to understand their influence on his views on automation.
By doing this, I wish to inquire about Simon's pessimism on the prospects
of human rationality and his delegation of decision-making to
``more capable'' systems such as machines and organizations. 
\subsection{Main question}
\label{sec:org3587960}
What is Herbert Simon's concept of the computer, and how did this
concept influence his ideas on automation? 
\subsection{Secondary Questions}
\label{sec:org8b80b80}
What was the relation between the natural and the artificial for Simon, and
how does this relation relate to his late-life project of a Science of
the Artificial?

How does Simon's Science of the Artificial relate to his views on
economics? And, is his Science of the Artificial compatible with
neoclassical economics? 

How do Simon's ideas on automation relate to his political views?

Did Simon draw a clear line between the natural and the artificial, and between
the human and the inhuman? Likewise, did he subscribe to the unity of
science thesis?
\section{Motivation}
\label{sec:org29f4134}
Following the intellectual debacle of the economics profession after the
2008 global financial crisis, behavioral economics presents itself as a
credible alternative to the lack of realism of mainstream neoclassical
theory. According to this view, behavioral economics incorporates certain
results from psychology that would allow a more realistic mathematical
modeling of human behavior. Since Henry Simon was the most important
postwar neoclassical economist who kept a close eye on psychology (before
it became again fashionable with the profession), I think it is worth to
dedicate a thesis to his ideas. 

The reason for writing on Simon is \emph{not} to write another thesis on bounded
rationality---I suspect that this interpretation of Simon's message is just
a palatable reconstruction of his ideas by the economic mainstream. Unlike
other neoclassicals of his generation, Simon came to fully embrace the
postwar cyborg sciences, and even played a key role in the foundation of
the field of artificial intelligence. Given that artificial intelligence
presents itself as the next holy grail of science in our generation, I
think that it is also worth to study Simon for the wider importance of
computer science concepts in popular culture.

Since one of the important aspects of Simon's vision of a \emph{Science of the
Artificial} is the computer---understood not just as a technology, but as an
ontology, a vision, a political project, etc.---, this thesis will explore
Simon's concept of the computer. Hence, the provisional title of the thesis:
The Automata \& the Engineer.

My aim is to project this thesis into a Ph.D dissertation around the theme
of the governing machine: the automation of political decision making and
the delegation of all political responsibility to machines.

\section{Bibliography}
\label{sec:org16d0aae}
This is a thematic bibliography organized around keywords. To facilitate
reading, the book titles are highlighted in blue. 

\nocite{*}
\subsection{The Brain}
\label{sec:org0b34368}
\printbibliography[heading=none,keyword=memoire,keyword=brain]
\subsection{The Cold War}
\label{sec:orgf09a9e7}
\printbibliography[heading=none,keyword=memoire,keyword=cold-war,notkeyword=brain]
\subsection{The Computer}
\label{sec:orgaa4c5db}
\printbibliography[heading=none,keyword=memoire,keyword=computer,notkeyword=brain,notkeyword=cold-war]
\subsection{Metaphors}
\label{sec:orgb81da94}
\printbibliography[heading=none,keyword=memoire,keyword=metaphors,notkeyword=brain,notkeyword=cold-war,notkeyword=computer,notkeyword=cyborg]
\subsection{Herbert Simon}
\label{sec:orgff2982b}
\printbibliography[heading=none,keyword=memoire,keyword=herbert-simon,notkeyword=brain,notkeyword=cold-war,notkeyword=computer,notkeyword=cyborg,notkeyword=metaphors]
\subsection{The Social Sciences}
\label{sec:org726aa82}
\printbibliography[heading=none,keyword=memoire,keyword=social-science,notkeyword=brain,notkeyword=computer,notkeyword=cyborg,notkeyword=metaphors,notkeyword=herbert-simon,notkeyword=cold-war]
\end{document}