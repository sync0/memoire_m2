% Intended LaTeX compiler: pdflatex
\documentclass[paper=A4,portrait,twoside=true,twocolumn=true,headinclude=false,footinclude=false,fontsize=10,BCOR=15mm,DIV=13,pagesize=auto,titlepage=firstiscover,mpinclude=true,headings=twolinechapter,open=right,chapterprefix=false,numbers=noendperiod,headsepline=false,parskip=false]{scrbook}
\usepackage{fontspec}
\usepackage{xunicode}
\usepackage{url}
\usepackage{soul}
\usepackage{polyglossia}
\setmainlanguage{english}
\setotherlanguages{french,italian}
\usepackage[french=guillemets,thresholdtype=words,threshold=3]{csquotes}
\MakeAutoQuote{«}{»}
\AtBeginEnvironment{quote}{\itshape}
\usepackage[backend=biber,style=authoryear,doi=false,isbn=false,url=false]{biblatex}
\addbibresource{~/Documents/mendeley/library.bib}
\usepackage{amsmath}
\usepackage{amsthm}
\usepackage{amssymb}
\usepackage{centernot}
\usepackage{hyperref}
\hypersetup{colorlinks,urlcolor=blue,linkcolor=blue,citecolor=red,filecolor=black}
\usepackage{balance}
\usepackage{booktabs}
\usepackage[french]{fmtcount}
\fmtcountsetoptions{french=france}
\usepackage[singlespacing]{setspace}
\usepackage[super]{nth}
\usepackage{microtype}
\microtypecontext{kerning=french}
\usepackage{ragged2e}
\usepackage[all]{nowidow}
\usepackage{enumitem}
\usepackage{adforn}
\usepackage{float}
\usepackage{xcolor}
\usepackage{graphicx}
\graphicspath{ {/home/sync0/Dropbox/paris_1/} }
\usepackage{xunicode}
\usepackage{fontspec}
\usepackage{xltxtra}
\usepackage{unicode-math}
\usepackage[oldstyle]{libertine}
\newfontfamily\titlefamily[Scale=2]{Linux Biolinum O}
\newcommand\HUGE{\fontsize{30}{30}\selectfont}
\usepackage{scrlayer-scrpage}
\pagestyle{scrheadings}
\clearscrheadfoot
\automark[chapter]{part}
\chead{\headmark}
\ohead{\pagemark}
\AfterTOCHead{\singlespacing}
\setkomafont{labelinglabel}{\normalfont\sffamily\bfseries}
\setkomafont{minisec}{\usekomafont{subsection}}
\setkomafont{pagehead}{\normalfont\itshape}
\setkomafont{pagenumber}{\large\upshape}
\setkomafont{caption}{\small}
\setkomafont{captionlabel}{\sffamily\bfseries}
\setkomafont{part}{\HUGE\scshape\lowercase}
\renewcommand*{\partformat}{}
\renewcommand\partmarkformat{}
\setkomafont{chapter}{\Huge\scshape\lowercase}
\RedeclareSectionCommand[beforeskip=0cm,afterskip=1.5cm]{chapter}
\setkomafont{section}{\LARGE\scshape\lowercase}
\setkomafont{subsection}{\large}
\setkomafont{subsubsection}{\normalfont\sffamily\bfseries}
\renewcommand*{\addparttocentry}[2]{\addtocentrydefault{part}{}{\Large\scshape\sffamily\lowercase{#2}}}
\addtokomafont{chapterentry}{\normalsize\sffamily\bfseries}
\usepackage{tocstyle}
\settocfeature{raggedhook}{\raggedright}
\selecttocstyleoption{tocgraduated}
\usetocstyle{nopagecolumn}
\renewcommand*\labelitemi{\adforn{33}}
\renewcommand*\labelitemii{\adforn{73}}
\renewcommand*\labelitemiii{\adforn{73}}
\renewcommand*\labelitemiv{\adforn{73}}
\definecolor{bibleblue}{HTML}{00339a}
\definecolor{soothing_green}{HTML}{E1F7DB}
\theoremstyle{definition}
\newtheorem{lecture}{Lecture}
\newtheorem*{lecture*}{Lecture}
\newtheorem{problem}{Problème}
\newtheorem*{problem*}{Problème}
\newtheorem{interpretation}{Interpretation}
\newtheorem*{interpretation*}{Interpretation}
\newcommand{\notimplies}{\centernot\implies}
\setcounter{secnumdepth}{\partnumdepth}
\setcounter{tocdepth}{1}
\recalctypearea
\setlength{\columnsep}{0.5cm}
\author{Carlos Alberto Rivera Carreño}
\date{}
\title{}
\hypersetup{
 pdfauthor={Carlos Alberto Rivera Carreño},
 pdftitle={},
 pdfkeywords={},
 pdfsubject={},
 pdfcreator={Emacs 26.1 (Org mode 9.1.14)}, 
 pdflang={English}}
\begin{document}

\begin{titlepage}
 \centering
% \includegraphics[width=0.5\textwidth]{logo_noir_fr.png}\par
 \vspace{4\baselineskip}
\begin{french}
 {\Large Université Paris I Panthéon Sorbonne \par}
 {\Large \textsc{ufr} 02 : Sciences économiques  \par}
 {\large Master 2 : Économie et sciences humaines \par}
 {\large 2018-2019 \par}
\end{french}
 \vspace{2\baselineskip}
 {\huge Vers la machine à gouverner  \par}
 {\Large Herbert Simon and the Impossibility of a Democratic Computer \par}
\vspace*{\fill}
\begin{french}
 {\large Présenté et sountenu par : \par}
\end{french}
 {\large \textsc{carlos alberto rivera carreño}\par}
 \vspace{1\baselineskip}
\begin{french}
 {\large Directeur de mémoire : \par}
\end{french}
 {\large \textsc{jean-sébastien lenfant}\par}
\end{titlepage}

\onecolumn
\pagestyle{empty}

\begin{french}
L'Université Paris 1 Panthéon Sorbonne n'entend donner aucune approbation,
ni désapprobation aux opinions émises dans ce mémoire ; elle doivent être
considérées comme propres à leur auteur. 
\end{french}

\newpage
\vspace*{\fill}
\noindent
\includegraphics[height=1.5cm]{gpl3.png}\par
\vspace{1\baselineskip}
\begin{english}
This text is free: you can redistribute it and/or modify it
under the terms of the \textsc{gnu} General Public License as published by
the Free Software Foundation, either version 3 of the License or any later
version.

This text is distributed in the hope that it will be useful, but \textbf{without
any warranty}; without even the implied warranty of \textbf{merchantability or 
fitness for a particular purpose}. See the \textsc{gnu} General 
Public License for more details.

You should have received a copy of the \textsc{gnu} General Public License along
with this text. If not, see \url{http://www.gnu.org/licenses/}.

\vspace{1\baselineskip}
\noindent
Copyright \textcopyright \textsc{sync0} 2018. 
\end{english}

\newpage 

\begin{FlushRight}
\begin{italian}
% \textit{Para una lectora lejana.}
% \textit{Per il professore Giorgio Israel. \newline Mi dispiace, ho mancato il nostro incontro. \linebreak Sono in ritardo, come la coscienza della nostra generazione. \linebreak I campi sanguinano. Tutti lo sanno, ma a nessuno importa.}
% \textit{Per il professore Giorgio Israel. \newline Benché i nostri destini fossero uniti da quell'anno fatale del 1492, \linebreak ho mancato il nostro incontro. \linebreak Sono arrivato in ritardo, come la coscienza della nostra generazione. \linebreak I campi sanguinanti sono così prossimi \linebreak che le gocce accarezzerebbero gli occhi. \linebreak Eppure, nessuno vede niente. \linebreak Tante informazione, ma così poca conoscenza. \linebreak Anche le lacrime non bastano per addolcire i cuori. \linebreak Nel fratempo, riposa in pace, maestro.}
\textit{Per il professore Giorgio Israel. \newline Benché i nostri destini fossero uniti da quell'anno fatale del 1492, \linebreak ho mancato il nostro incontro. \linebreak Sono arrivato in ritardo, come la coscienza della nostra generazione. \linebreak I campi sanguinanti sono così prossimi, \linebreak ma nessuno vede niente. \linebreak Nell'era dell'informazione, diventiamo più ignoranti. \linebreak Questo è il prezzo del biglietto d'ingresso. \linebreak Grazie a Lei l'ho capito, maestro.}
\end{italian}
\end{FlushRight}

\newpage
\tableofcontents 

\frontmatter
\twocolumn
\pagestyle{plain}

\mainmatter
\pagestyle{scrheadings}
\part{Research}
\label{sec:orgf4c29f5}
\chapter{Research Proposal}
\label{sec:org73a8ab1}
\begin{itemize}
\item[{$\square$}] Main question.
\item[{$\square$}] Secondary questions.
\item[{$\square$}] Scope and interrelations.
\item[{$\square$}] Research hypotheses.
\item[{$\square$}] Methodology.
\begin{itemize}
\item[{$\square$}] Keywords.
\item[{$\square$}] Debates \& controversies.
\item[{$\square$}] Axes d'interpretation.
\end{itemize}
\item[{$\square$}] Sources.
\begin{itemize}
\item[{$\square$}] Update list of key texts.
\item[{$\square$}] Update list of key authors.
\end{itemize}
\end{itemize}
\section{Motivation}
\label{sec:org410c653}
Generally speaking, I am interested in questions of methodology in the
social sciences. More specifically, I am interested in the role of
metaphors in the construction of social theories, and the place that these
have accorded to determinism. Thus, I would like to explore the parallels
between the understandings---about the essence of nature, society, the
mind, and the universe---in economics and those in the other natural and
social sciences. Without any ambition to elevate the regime of truth of
economics, I am genuinely puzzled by its acceptance by the general public,
as if economics was a type of discourse about a certain reality with
portentous social consequences. But what is this social reality that is the
object of economics? How and why have economists come to think about social
reality in this way? With this master's thesis, I want to contribute a tiny
bit towards a better comprehension of this impasse.

As we have discussed previously, I want to focus my master's thesis around
Henry Simon; therefore, the question of how he fits into the picture I just
painted requires an answer. For the record, I am neither sympathetic to
Simon's vision of the social world nor to his project of a general \emph{science
of the artificial}. Rather, I decided to structure the thesis around him
because unlike many neoclassicals of his generation, Simon fully drank the
cyborg Kool-Aid of the postwar systems sciences, and thus, he took to heart
the man-machine analogy to its final consequences---to the point where he
was one of the founders of the field of artificial intelligence. Given that
today artificial intelligence presents itself as the next holy grail of
science, I find it valuable to write about Simon also as an excuse to study
more about computer science and its ontology.\footnote{As Mirowski shows in \emph{Machine Dreams} (2002), the development of the
computer and its entourage of systems sciences have had profound
consequences for postwar neoclassical economics.} 
\section{Subject-matter}
\label{sec:orge6d36cb}
I would like to write about the computer as a political technology,
specially in its guise as the so-called \emph{machine à gouverner}: the ultimate
delegation of political authority and responsibility to machines---or more
generally, to automatic decision mechanisms.\footnote{The idea is certainly not Simon's. In fact, as far as I know, the
term \emph{machine à gouverner} comes from a 1948 article in the newspaper \emph{Le
Monde}, in which a Dominican friar, Père Dubarle, reviewed Norbert Wiener's
book \emph{Cybernetics}. In fact, in his latter book \emph{The Human Use of Human
Beings} (1950), Wiener explicitly cites and discusses Dubarle's idea.} Therefore, in this
thesis, I will trace the genealogy of Simon's ideas on the computer and
computation to understand their influence on his views on social
organization. By doing this, I wish to inquire about Simon's pessimism on
the prospects of human rationality and his delegation of decision-making to
``more capable'' systems such as machines and organizations.

My opinion, so far, on Simon is that his idea of bounded rationality served
as a kind of excuse to subordinate and reduce the individual to play a very
minor role in social organizations and the conduct of human affairs in
general. And, from what I have read, Simon's epistemic pessimism---which,
by the way, is not too distant from Hayek's---has roots both on a kind of
personal obsession of his with the relation between individual
responsibility and ethical choice,\footnote{I am not aware that Simon was a particularly religious man, but these
concerns are quite clearly influenced by christian morality and Simon's
American-style liberal values.} and on his fascination with the
fledgling computer technologies and their prospects. Therefore, in this
master's thesis, I would like to dedicate an important part to Simon's
relation to the computer. This is necessary to account for the importance
of the treatment of information in his theories of organization. In
fact, I surmise that what really preoccupied Simon wasn't 
rationality per se, but information.

In the second part of the thesis, I would like to discuss the political
implications of Simon's ideas on the computer and
computation. Unfortunately, I cannot say more about this, since my reading
hasn't taken me that far, but I will very likely base this section on
Philip Mirowski's \emph{Machine Dreams} and Paul Edwards's \emph{The Closed
World}. Moreover, in Simon's article \emph{Heuristic Problem Solving: The Next
Advance in Operations Research}, he discusses the relation between Charles
Babbage and Adam Smith's ideas, which echoes a discussion in the second
chapter of \emph{Machine Dreams} on these same men. I surmise that there could be
some insightful material therein to construct this section. Although, I
would like to mention Babbage and Jevons at some point to compare their
thinking to Simon's, it is unlikely that time will allow for this.
\subsection{Main question}
\label{sec:orged090cf}
What is Herbert Simon's definition of the computer? How did this concept
influence his ideas of \emph{the human}? And, what are the consequences for the
organization of labor of this vision of man and machine?
\subsection{Secondary Questions}
\label{sec:org0127caf}
What was the relation between the natural and the artificial for Simon, and
how does this relation relate to his late-life project of a Science of
the Artificial?

How do Simon's ideas on automation relate to his political views?

Did Simon subscribe to the unity of science thesis?
\section{Bibliography}
\label{sec:org474f37c}
This is a thematic bibliography organized around keywords. To facilitate
reading, the book titles are highlighted in blue. 

\nocite{*}
\subsection{The Brain}
\label{sec:orga7db58d}
\printbibliography[heading=none,keyword=memoire,keyword=brain]
\subsection{The Cold War}
\label{sec:orgd1f832c}
\printbibliography[heading=none,keyword=memoire,keyword=cold-war,notkeyword=brain]
\subsection{The Computer}
\label{sec:org2f14920}
\printbibliography[heading=none,keyword=memoire,keyword=computer,notkeyword=brain,notkeyword=cold-war]
\subsection{Metaphors}
\label{sec:orgd1be263}
\printbibliography[heading=none,keyword=memoire,keyword=metaphors,notkeyword=brain,notkeyword=cold-war,notkeyword=computer,notkeyword=cyborg]
\subsection{Herbert Simon}
\label{sec:org49c097b}
\printbibliography[heading=none,keyword=memoire,keyword=herbert-simon,notkeyword=brain,notkeyword=cold-war,notkeyword=computer,notkeyword=cyborg,notkeyword=metaphors]
\subsection{The Social Sciences}
\label{sec:org7a44487}
\printbibliography[heading=none,keyword=memoire,keyword=social-science,notkeyword=brain,notkeyword=computer,notkeyword=cyborg,notkeyword=metaphors,notkeyword=herbert-simon,notkeyword=cold-war]
\chapter{Research Plan}
\label{sec:org98e165f}
Look up into the history of Herbert Simon's 
\chapter{Thesis Plan}
\label{sec:orgf401c6c}
\section{Skeleton}
\label{sec:org069f469}
\begin{center}
\begin{tabular}{llr}
\textbf{Section} &  & \textbf{Length}\\
Introduction &  & 20\\
Chapter 1 &  & 25\\
 & Section 1 & \\
 & Section 2 & \\
 & Section 3 & \\
Chapter 2 &  & 25\\
 & Section 1 & \\
 & Section 2 & \\
 & Section 3 & \\
Conclusion &  & 5\\
\textbf{Total} &  & 80\\
\end{tabular}
\end{center}
\section{Draft Outline}
\label{sec:org4c78789}
\subsection{Introduction}
\label{sec:org733e086}
\subsection{Chapter 1}
\label{sec:org2742b82}
\begin{labeling}[~]{Section 1} 
\item[Section 1] Background on the cold war and the computer. 
\item[Section 2] Simon's place in the context just given. That is, historical information on Simon.
\item[Section 3] Simon's history with the computer and artifical intelligence.
\end{labeling}
\subsection{Chapter 2}
\label{sec:org492410d}
Discuss Simon's article on economic democracy and his views of the
organization of labor. 
\begin{labeling}[~]{Section 1} 
\item[Section 1] 
\item[Section 2] 
\item[Section 3] 
\end{labeling}
\subsection{Conclusion}
\label{sec:org09ac90b}
\chapter{Task list}
\label{sec:orga3e8ae3}
\begin{itemize}
\item[{$\square$}] Check with adviser.
\item[{$\square$}] Check with selected readers.
\item[{$\square$}] Update the Framework.
\end{itemize}
\chapter{Research}
\label{sec:org4744676}
\begin{itemize}
\item[{$\square$}] Final Sifting.
\begin{itemize}
\item[{$\square$}] Update list of key texts.
\item[{$\square$}] Update list of key authors.
\end{itemize}
\item[{$\square$}] If necessary, update any component of the Framework.
\end{itemize}
\end{document}