\begin{titlepage}
 \centering
\begin{french}
 {\large \textsc{université paris i panthéon sorbonne} \par}
  \vspace*{0.01\textheight}
 {\large \textsc{ufr} 02 : Sciences économiques  \par}
  \vspace*{0.01\textheight}
 {\large Master 2 : Économie et sciences humaines \par}
  \vspace*{0.01\textheight}
 {\large 2018--2019 \par}
\end{french}
  \vspace*{0.3\textheight}
 {\huge \textsc{Watched Over by Machines of Loving Grace?}  \par}
  \vspace*{0.02\textheight}
 {\Large Prolegomenon to a Historical Epistemolgy of  Labor, Intelligence, and the Machine \par}
\vfill
\begin{french}
 {\large\itshape présenté par  \par}
\end{french}
 {\Large Carlos Alberto Rivera Carreño \par}
  \vspace*{0.05\textheight}
\begin{french}
 {\large\itshape sous la direction de \par}
\end{french}
 {\Large Jean-Sébastien Lenfant \par}
\end{titlepage}

\pagestyle{empty}

\begin{french}
L'Université Paris 1 Panthéon Sorbonne n'entend donner aucune approbation,
ni désapprobation aux opinions émises dans ce mémoire ; elle doivent être
considérées comme propres à leur auteur. 
\end{french}
\vfill

\newpage
\begin{center}
\vspace*{\fill}
\noindent
\includegraphics[height=1.5cm]{gpl3.png}\par
\vspace{1\baselineskip}

As a commitment to free software, this text used  \textsc{gnu}
Emacs for writing, 12 pt. Linux Libertine as font,  \hologo{\fmtname}
and \KOMAScript for typesetting, and  \hologo{LuaTeX} for compilation.\\
\vspace{1\baselineskip}
This text and its source files are free and available for download at:\\
\vspace{1\baselineskip}
\url{https://github.com/sync0/memoire_m2}
\vspace{1\baselineskip}

\noindent You can redistribute them and/or modify them under the terms of
the \textsc{gnu} General Public License as published by the Free Software
Foundation, either version 3 of the License or any later version.

This text and its source files are distributed in the hope that they will
be useful, but \textbf{without any warranty}; without even the implied
warranty of \textbf{merchantability or fitness for a particular purpose}.
See the \textsc{gnu} General Public License for more details at:\\ 
\vspace{1\baselineskip}
\url{https://www.gnu.org/licenses/gpl-3.0.en.html} 

\vspace{1\baselineskip}
\noindent
\textcopyright Carlos Alberto Rivera Carreño 2019.
\end{center}


% \newpage 
% \newpage\null\newpage

% \begin{FlushRight}
% \begin{spanish}
% \textit{Al padre Camilo Torres.}
% \end{spanish}
% \end{FlushRight}

\newpage 

\dictum[Thomas Hobbes, {\itshape Leviathan}]{Nature (the art whereby God hath made and
  governs the world) is by the art of man, as in many other things, so in
  this also imitated, that it can make an artificial animal. For seeing
  life is but a motion of limbs, the beginning whereof is in some principal
  part within, why may we not say that all automata (engines that move
  themselves by springs and wheels as doth a watch) have an artificial
  life? For what is the heart, but a spring; and the nerves, but so many
  strings; and the joints, but so many wheels, giving motion to the whole
  body, such as was intended by the Artificer? Art goes yet further,
  imitating that rational and most excellent work of Nature, man.}

\newpage
\tableofcontents 