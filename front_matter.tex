\frontmatter
\pagestyle{plain}
\chapter*{Abbreviations} 
\section*{Archival Sources}

When citing archival materials, I indicate their location within the
following archives:

\begin{labeling}[:]{HSMC} 
% \item[HSCM] Herbert Simon Papers, Carnegie Mellon University Archives.
\item[ENPC] École nationale des ponts et chaussées.
\item[BNF] Bibliothèque nationale de France. 
\end{labeling}

I accessed them through their respective on-line repositories.

\begin{labeling}[:]{HSMC} 
% \item[HSCM] \href{https://digitalcollections.library.cmu.edu/portal/index.jsp}{Carnegie Mellon University Libraries Digital Collections.} 
\item[ENPC] \href{https://patrimoine.enpc.fr/}{Bibliothèque numérique patrimoniale des ponts et chaussées}.
\item[BNF] \href{https://gallica.bnf.fr/}{Gallica.}
\end{labeling}

% Only the pdf version of this document includes hyperlinks to the source
% files, when available.

\chapter{Acknowledgements} 
This Master's thesis, even with all its shortcomings, would have been
impossible to write without the help of many people.

I would like to thank my Master's thesis adviser, professor Jean-Sébastien
Lenfant, for his patience and dedication, and professors Annie L. Cot and
Jérôme Lallement for their support to all the students at the
\textit{Réseau en Épistémologie et en Histoire de la Pensée Économique Récente} (\textsc{rehpere}) at Paris 1 University.

The arguments presented herein have been chiseled by the insighfult
comments and suggestions of Justine Loulergue, Guillaume Noblet, Guillaume
Lancereau, Pedro Javier Ortiz, Seung Hoon Hahm, and my fellow students of
the M2 \textit{Économie et sciences humaines} at Paris 1.

% Moreover, rummaging through the Herbert Simon Papers would have been
% impossible without the assitance of Emily Davis at Carnegie Mellon
% University Libraries.

I would like to thank my wife's family and my own for their love and
support. Last but not least, I would like to thank my wife, Min Jung. This
text is dedicated to her.

\vspace{2\baselineskip}
\begin{FlushRight}
  Carlos Alberto Rivera Carreño\\
  \today\\
  Bagneux, France
\end{FlushRight}


% \markboth{\MakeMarkcase{Preface}}{\MakeMarkcase{Préface}}

% \chapter{Note aux lecteurs francophones} 
% Le fait que ce mémoire fut préparé au sein d'une université française
% m'oblige moralement à \ldots aux lecteurs francophones. 


\chapter[Preface]{Preface:\\ Technicism as Presentism} 
% \markboth{\MakeMarkcase{Preface}}{\MakeMarkcase{Préface}}
% \mainmatter


Sunday night September 23, 1962, the animated sitcom \textit{The Jetsons}
aired for the first time, introducing audiences across the United States to
the futuristic life of the Jetson family. The show celebrated the American
way of life in a future of \textit{private} flying cars, nuclear family
arrangements, and---oddly enough---salaried work. George Jetson, the
American \textit{family man}, worked three-hour shifts three times a week
only pressing a button at a company, while his wife Jane Jetson, relieved
from drudgery by her robot maid and automated apartment, was an obedient
homemaker. The problem with the Jetsons's depiction of the future is its
presentism: Neither political nor social, but only \textit{technological}
change is possible. With all the technological advances presented in the
show, why would George, or anybody, work ? If automation holds the key to a
lifetime of leisure, why would there be vacations? For the Jetson family,
\textit{stuff} has changed, but the old world stays the same.

Contrary to this technicist dream, this thesis seeks to show that technical
changes are embedded in wider social and ideological discussions: What is
work? What is the place of work in society? What is the nature of
subordinated work? What is a worker? Whatever has prevented thinking about
these issues, be it the resilience of a ``broad church'' positivism in the
social sciences or the belief in the transparency and self-sufficiency of
``facts,'' they must be addressed in present discussions about the
influence of artificial intelligence on the future of labor. Therefore,
this thesis sketches an epistemological history of the relation between
thinking about the possibility of machine intelligence an changes in the
status of work and workers. My main argument is that the dominant
technicist view of the influence of automation technology on the future of
labor is inadequate because it ignores that discussions on machine
intelligence, automation, and sociotechnical systems for disciplining
workers have been connected since at least the nineteenth century.

\section{Organization of the Thesis}


In the first chapter, I will argue that the dominant technicist view, which
analyzes the influence of technology on the future of labor in terms of the
possibilities to automate different (categories of) tasks, is inadequate
because it ignores the conventionalist dimension of labor relations. After
discussing two examples that show the historical specificity of labor, in
the second chapter, I sketch the history of the relation between machine
intelligence and the organization of labor by discussing the influence of
de Prony's project for the calculation of the Cadastre tables in the
construction of Charles Babbage's calculating machines. My goal is to
contribute to a different perspective on the influence of technology on
labor by showing that the project to build the first ``computer'' was
inspired by the application of the \textit{mental} division of labor to the
first great calculation project, with the consequence that workers were
progressively deprived of knowledge and control over the production
process.

% \lipsum

\mainmatter
\pagestyle{scrheadings}