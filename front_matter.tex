\frontmatter
\pagestyle{plain}
\chapter*{Abbreviations} 
\section*{Archival Sources}

When citing archival materials, I indicate their location within the
following archives:

\begin{labeling}[:]{HSMC} 
\item[HSCM] Herbert Simon Papers, Carnegie Mellon University Archives.
\item[ENPC] École nationale des ponts et chaussées.
\item[BNF] Bibliothèque nationale de France. 
\end{labeling}

I accessed them through their respective on-line repositories.

\begin{labeling}[:]{HSMC} 
\item[HSCM] \href{https://digitalcollections.library.cmu.edu/portal/index.jsp}{Carnegie Mellon University Libraries Digital Collections.} 
\item[ENPC] \href{https://patrimoine.enpc.fr/}{Bibliothèque numérique patrimoniale des ponts et chaussées}.
\item[BNF] \href{https://gallica.bnf.fr/}{Gallica.}
\end{labeling}

Only the pdf version of this document includes hyperlinks to the source
files, when available.

\chapter{Acknowledgements} 
This Master's thesis, even with all its shortcomings, would have been
impossible to write without the help of many people.

I would like to thank my Master's thesis adviser, professor Jean-Sébastien
Lenfant, for his patience and dedication, and professors Annie L. Cot and
Jérôme Llalement for their support to all the students at the
\textit{Réseau en Épistémologie et en Histoire de la Pensée Économique Récente} (\textsc{rehpere}) at Paris 1 University.

The arguments presented herein have been chiseled by the insighfult
comments and suggestions of Justine Loulergue, Guillaume Noblet, Guillaume
Lancereau, Pedro Javier Ortiz, Seung Hoon Hahm, and my fellow students of
the M2 \textit{Économie et sciences humaines} at Paris 1.

Moreover, rummaging through the Herbert Simon Papers would have been
impossible without the assitance of Emily Davis at Carnegie Mellon
University Libraries.

I would like to thank my wife's family and my own for their love and
support. Last but not least, I would like to thank my wife, Min Jung; this
text is dedicated to her.

\vspace{2\baselineskip}
\begin{FlushRight}
  Carlos Alberto Rivera Carreño\\
  \today\\
  Bagneux, France
\end{FlushRight}


% \markboth{\MakeMarkcase{Preface}}{\MakeMarkcase{Préface}}

\chapter{Note aux lecteurs francophones} 
Le fait que ce mémoire fut préparé au sein d'une université française
m'oblige moralement à \ldots aux lecteurs francophones. 

Ce mémoire traite de la relation entre l'histoire du concept de travail,
l'histoire de l'orinateur, l'histoire de l'intelligence, et la pensée de
Herbert Simon. J'essai de replacer l'histoire de travail au sein des
questions sur  l'application des analogies entre les sciences sociales et
les sciences naturelles.  

Dans le premier chapitre je raconte \ldots 

Dans le deuxième chapitre je raconte \ldots 

\lipsum

\chapter[Preface]{Preface:\\ Science and History} 
% \markboth{\MakeMarkcase{Preface}}{\MakeMarkcase{Préface}}
% \mainmatter

\lipsum
\mainmatter
\pagestyle{scrheadings}