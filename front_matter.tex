\frontmatter
\pagestyle{plain}
\chapter*{Abbreviations} 
\section*{Archival Sources}

When citing archival materials, I indicate their location within the
following archives:

\begin{labeling}[:]{HSMC} 
\item[HSCM] Herbert Simon Papers, Carnegie Mellon University Archives.
\item[ENPC] École nationale des ponts et chaussées.
\item[BNF] Bibliothèque nationale de France. 
\end{labeling}

I accessed them through their respective on-line repositories.

\begin{labeling}[:]{HSMC} 
\item[HSCM] \href{https://digitalcollections.library.cmu.edu/portal/index.jsp}{Carnegie Mellon University Libraries Digital Collections.} 
\item[ENPC] \href{https://patrimoine.enpc.fr/}{Bibliothèque numérique patrimoniale des ponts et chaussées}.
\item[BNF] \href{https://gallica.bnf.fr/}{Gallica.}
\end{labeling}

Only the pdf version of this document includes hyperlinks to the source
files, when available.

\chapter{Acknowledgements} 
\lipsum
% \markboth{\MakeMarkcase{Preface}}{\MakeMarkcase{Préface}}
\chapter{Note aux lecteurs francophones} 
Le fait que ce mémoire fut préparé au sein d'une université française
m'oblige moralement à \ldots aux lecteurs francophones. 

Ce mémoire traite de la relation entre l'histoire du concept de travail,
l'histoire de l'orinateur, l'histoire de l'intelligence, et la pensée de
Herbert Simon. J'essai de replacer l'histoire de travail au sein des
questions sur  l'application des analogies entre les sciences sociales et
les sciences naturelles.  

Dans le premier chapitre je raconte \ldots 

Dans le deuxième chapitre je raconte \ldots 

\lipsum

\chapter[Preface]{Preface:\\ Science and History} 
% \markboth{\MakeMarkcase{Preface}}{\MakeMarkcase{Préface}}
% \mainmatter

\lipsum
\mainmatter
\pagestyle{scrheadings}