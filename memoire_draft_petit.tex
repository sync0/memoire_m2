% Intended LaTeX compiler: pdflatex
\documentclass[paper=B6,portrait,twoside=true,twocolumn=false,headinclude=true,footinclude=false,fontsize=12,BCOR=10mm,DIV=calc,pagesize=auto,titlepage=firstiscover,mpinclude=false,headings=normal,headings=twolinechapter,open=right,toc=graduated,chapterprefix=false,numbers=endperiod,parskip=half+]{scrbook}
\usepackage{fontspec}
\usepackage{xunicode}
\usepackage{url}
\usepackage{soul}
\usepackage{polyglossia}
\setmainlanguage{french}
\setotherlanguages{english,spanish}
\usepackage{xeCJK}
\setCJKmainfont{Baekmuk Batang}
\usepackage[french=guillemets,thresholdtype=words,threshold=3]{csquotes}
\MakeAutoQuote{«}{»}
\AtBeginEnvironment{quote}{\itshape}
\usepackage{amsmath}
\usepackage{amsthm}
\usepackage{amssymb}
\usepackage{centernot}
\usepackage{hyperref}
\hypersetup{colorlinks,urlcolor=blue,linkcolor=red,citecolor=red,filecolor=black}
\usepackage{booktabs}
\usepackage[french]{fmtcount}
\fmtcountsetoptions{french=france}
\usepackage[singlespacing]{setspace}
\usepackage[super]{nth}
\usepackage{microtype}
\microtypecontext{kerning=french}
\usepackage{ragged2e}
\usepackage[all]{nowidow}
\usepackage{enumitem}
\usepackage{adforn}
\usepackage{float}
\usepackage{xcolor}
\usepackage{graphicx}
\graphicspath{ {/home/sync0/Dropbox/paris_1/} }
\usepackage{lipsum}
\usepackage[textsize=scriptsize, linecolor=soothing_green, backgroundcolor=soothing_green]{todonotes}
\usepackage{xunicode}
\usepackage{fontspec}
\usepackage{xltxtra}
\defaultfontfeatures{Scale=MatchLowercase}
\setmainfont[Mapping=tex-text,Numbers=OldStyle,SmallCapsFeatures={LetterSpace=4,Ligatures=NoCommon}]{Linux Libertine O}
\setsansfont[Mapping=tex-text]{Linux Biolinum O}
\setmonofont[Mapping=tex-text]{Inconsolata}
\newfontfamily\titlefamily[Scale=1.5]{Linux Biolinum O}
\newcommand\HUGE{\fontsize{30}{30}\selectfont}
\usepackage{scrlayer-scrpage}
\pagestyle{scrheadings}
\clearscrheadfoot
\automark[chapter]{part}
\chead{\headmark}
\ohead{\thepage}
\renewcommand\partmarkformat{}
\AfterTOCHead{\singlespacing}
\setkomafont{disposition}{\normalfont\normalcolor}
\setkomafont{labelinglabel}{\normalfont\bfseries}
\setkomafont{minisec}{\usekomafont{subsection}}
\addtokomafont{pageheadfoot}{\sffamily\upshape}
\addtokomafont{caption}{\small}
\addtokomafont{captionlabel}{\bfseries}
\addtokomafont{part}{\HUGE\scshape\sffamily\lowercase}
\renewcommand*{\partformat}{\partname}
\addtokomafont{chapter}{\huge\scshape\bfseries\sffamily\lowercase}
\RedeclareSectionCommand[beforeskip=0cm,afterskip=1.5cm]{chapter}
\addtokomafont{section}{\LARGE\scshape\sffamily\lowercase}
\addtokomafont{subsection}{\large\sffamily\bfseries}
\addtokomafont{subsubsection}{\large\sffamily\itshape}
\renewcommand*{\addparttocentry}[2]{\addtocentrydefault{part}{}{\Large\scshape\sffamily\lowercase{#2}}}
\addtokomafont{chapterentry}{\normalsize\sffamily\bfseries}
\usepackage[tocflat,tocindentauto]{tocstyle}
\usetocstyle{nopagecolumn}
\unsettoc{toc}{onecolumn}
\renewcommand*\labelitemi{\adforn{33}}
\renewcommand*\labelitemii{\adforn{73}}
\renewcommand*\labelitemiii{\adforn{73}}
\renewcommand*\labelitemiv{\adforn{73}}
\renewcommand*{\dictumwidth}{.8\textwidth}
\renewcommand*{\raggeddictum}{\centering}
\renewcommand*{\raggeddictumtext}{\centering}
\addtokomafont{dictum}{\large\rmfamily}
\definecolor{bibleblue}{HTML}{00339a}
\definecolor{soothing_green}{HTML}{E1F7DB}
\theoremstyle{definition}
\newtheorem{lecture}{Lecture}
\newtheorem*{lecture*}{Lecture}
\newtheorem{problem}{Problème}
\newtheorem*{problem*}{Problème}
\newtheorem{interpretation}{Interpretation}
\newtheorem*{interpretation*}{Interpretation}
\setcounter{secnumdepth}{\partnumdepth}
\setcounter{tocdepth}{1}
\recalctypearea
\author{Carlos Alberto Rivera Carreño}
\date{}
\title{}
\hypersetup{
 pdfauthor={Carlos Alberto Rivera Carreño},
 pdftitle={},
 pdfkeywords={},
 pdfsubject={},
 pdfcreator={Emacs 26.1 (Org mode 9.1.14)}, 
 pdflang={Frenchb}}
\begin{document}

\begin{titlepage}
 \centering
% \includegraphics[width=0.5\textwidth]{logo_noir_fr.png}\par
 \vspace{4\baselineskip}
 {\Huge The Liberal Democratic Governing Machine \par}
 \vspace{1\baselineskip}
 {\Large The Rationality of Governance and the Governance of Irrationality in Post-War American Economics \par}
\vspace*{\fill}
 {\Large Mémoire de \textsc{m2} \par}
 \vspace{2\baselineskip}
 {\large Par: \par}
 {\large \textsc{carlos alberto rivera carreño}\par}
 \vspace{1\baselineskip}
 {\large Directeur de thèse: \par}
 {\large \textsc{jean françois lenfant}\par}
\end{titlepage}

\pagestyle{empty}

\vspace*{\fill}
\noindent
\includegraphics[height=1.5cm]{gpl3.png}\par
\vspace{1\baselineskip}
\begin{english}
This text is free: you can redistribute it and/or modify it
under the terms of the \textsc{gnu} General Public License as published by
the Free Software Foundation, either version 3 of the License or any later
version.

This text is distributed in the hope that it will be useful, but \textbf{without
any warranty}; without even the implied warranty of \textbf{merchantability or 
fitness for a particular purpose}. See the \textsc{gnu} General 
Public License for more details.

You should have received a copy of the \textsc{gnu} General Public License along
with this text. If not, see \url{http://www.gnu.org/licenses/}.

\vspace{1\baselineskip}
\noindent
Copyright \textcopyright \textsc{sync0} 2018. 
\end{english}

\newpage 
\vspace*{\fill}


\begin{spanish}
Despierta la conciencia popular para volverse grito.
\end{spanish}

\vspace*{\fill}

\newpage
\tableofcontents 

\frontmatter
\pagestyle{plain}
\chapter{Acknowledgements} 

\lipsum

\chapter{Preface} 

Even when applying different techniques of interpretation to texts, I have
tried to understand how the circumstances of their production could inform
their interpretation. Even just for the sake of consistency, shouldn't I
apply the same standard to this Master's thesis? Shouldn't I provide the
reader with the tools, the context, etc. that informed this research?

Pace Barthes, the author of these words is very alive. Knowing the tragedy
of my country, one might ask me: Why do I wield the pen, as the Colombian
fields bleed? Why not wield the sword? Or, perhaps, why not wield both?
This text was written, certainly, to satisfy an earthly requirement: to
obtain a Master's degree. But beyond that, what is the interest of writing
for me?

Although this thesis is not a political pamphlet, it is grounded in
political perplexities. To the recule of the state, who not only does not
want to fund nor organize vast sections of economic activities, my
generations witnesses a strange mix of ``innovation'',
``entrepreneurship'', and . If the political Left, of my parent's
generation had an iron faith in the working class and the peasants to
change society, today's politically dissafected youth has all hopes in
private enterprise to solve the evils of the world.

As Oscar Wilde once remarked, ``it is easier to have sympathy with
suffering, than it is to have sympathy with thought''. Even thought,

As much as this document is intended to an academic audience, it is also
intended to them and to my father: This is the beginning of a lifelong
attempt to explain \emph{¿Cómo fue que se jodio el país?} 

\blockcquote{Alvarez2013}{The designation of the NPE [Nobel Prize in Economics] yesterday teaches us again the richness of economic theory, capable of producing  alternative points of view, and the risk that it represents to a society that does not know whom to follow. The power of this discipline is unusual, but just as it creates irrational exuberance, it also explains our own irrational behavior.}

Should they be right, this article is wrong. 

\mainmatter
\pagestyle{scrheadings}
\part{Preparatory Research}
\label{sec:orgbdbf277}
\chapter{First Sifting}
\label{sec:org8c4c0cd}
   \begin{labeling}[~]{Subject-matter} 
\item[Subject-matter] Lorem ipsum dolor sit amet
\end{labeling}
\section{Key text}
\label{sec:orgb4ddc28}
\begin{itemize}
\item[{$\square$}] Supiot, 2012, La gouvernance par les nombres
\item[{$\square$}] Hughes, 1958, Consciousness and Society: The Reorientation of European Social Thought, 1890-1930
\item[{$\square$}] Ross, 1994, Modernist Impulses in the Human Sciences, 1870-1930
\item[{$\square$}] Purcell, 1973, The Crisis of Democratic Theory: Scientific Naturalism and the Problem of Value
\item[{$\square$}] Butsch, 2008, The Citizen Audience: Crowds, Publics, and Individuals
\item[{$\square$}] Miller, 1955, Toward a general theory for the behavioral sciences
\item[{$\square$}] Deutsch, 1963, The Nerves of Government: Models of Political Communication and Control
\item[{$\square$}] Cohen-Cole, 2009, The Creative American: Cold War salons, social science, and the cure for modern society.
\item[{$\square$}] Imhotep,
\item[{$\square$}] Vicedo, 2012, Cold War emotions: The war over human nature
\item[{$\square$}] Heyck, 2005, Herbert Simon: The Bounds of Reason in Modern America
\item[{$\square$}] Heyck, 2015, Age of System: Understanding the development of modern social science
\item[{$\square$}] Heyck, 2005, Mind and Network
\item[{$\square$}] Heyck, 1999, Georges Miller, language, and the computer metaphor of mind
\item[{$\square$}] Backhouse, 2010, The unsocial social science: Economics and Neighboring Disciplines Since 1945
\item[{$\square$}] Dupuy, 2005, Aux origines des sciences cognitives
\item[{$\square$}] Chomsky, 1998, The Cold War \& the University: Toward an Intellectual History of the Postwar Years
\item[{$\square$}] Mikulark, 1966, ``Cybernetics and Marxism-Leninism'' in The Social Impact of Cybernetics, ed. Charles Dechert
\item[{$\square$}] Israel, 2004, Meccanicismo
\item[{$\square$}] Edwards, 1996, The Closed World: Computers and the Politics of Discourse in Cold War America
\end{itemize}
\section{Important Text}
\label{sec:org90db3cc}
\lipsum
\section{Ancillary Text}
\label{sec:orgdd2c2e5}
\lipsum
\section{Boundary Text}
\label{sec:org6e9e93d}
\lipsum
\chapter{Draft: Framework}
\label{sec:org9c0c74c}
\section{Main question}
\label{sec:orgc7e306b}
\lipsum
\section{Scope and interrelations}
\label{sec:orge6a3f33}
\lipsum
\section{Research hypotheses.}
\label{sec:org6353d6e}
\lipsum
\section{Methodology}
\label{sec:org259ddda}
\lipsum
\section{Keywords}
\label{sec:orgfc681d3}
\lipsum
\section{Debates \& Controversies}
\label{sec:orgbad6469}
\lipsum
\section{Axes d'interpretation}
\label{sec:org5659612}
\lipsum
\section{Bibliography}
\label{sec:org20b2ca2}
\end{document}