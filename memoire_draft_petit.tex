% Intended LaTeX compiler: pdflatex
\documentclass[paper=B6,portrait,twoside=true,twocolumn=false,headinclude=true,footinclude=false,fontsize=12,BCOR=10mm,DIV=calc,pagesize=auto,titlepage=firstiscover,mpinclude=false,headings=normal,headings=twolinechapter,open=right,toc=graduated,chapterprefix=false,numbers=endperiod,parskip=half+]{scrbook}
\usepackage{fontspec}
\usepackage{xunicode}
\usepackage{url}
\usepackage{soul}
\usepackage{polyglossia}
\setmainlanguage{english}
\setotherlanguages{french,spanish}
\usepackage[french=guillemets,thresholdtype=words,threshold=3]{csquotes}
\MakeAutoQuote{«}{»}
\AtBeginEnvironment{quote}{\itshape}
\usepackage[backend=biber,style=authoryear,arxiv=false,doi=false,isbn=false,url=false]{biblatex}
\AtEveryBibitem{\ifentrytype{book}{\color{blue}}{}}
\addbibresource{~/Documents/mendeley/library.bib}
\usepackage{amsmath}
\usepackage{amsthm}
\usepackage{amssymb}
\usepackage{centernot}
\usepackage{hyperref}
\hypersetup{colorlinks,urlcolor=blue,linkcolor=red,citecolor=red,filecolor=black}
\usepackage{booktabs}
\usepackage[french]{fmtcount}
\fmtcountsetoptions{french=france}
\usepackage[singlespacing]{setspace}
\usepackage[super]{nth}
\usepackage{microtype}
\usepackage{ragged2e}
\usepackage[all]{nowidow}
\usepackage{enumitem}
\usepackage{adforn}
\usepackage{float}
\usepackage[svgnames]{xcolor}
\usepackage{graphicx}
\graphicspath{ {/home/sync0/Dropbox/paris_1/} }
\usepackage{lipsum}
\usepackage{xunicode}
\usepackage{fontspec}
\usepackage{xltxtra}
\defaultfontfeatures{Scale=MatchLowercase}
\setmainfont[Mapping=tex-text,Numbers=OldStyle,SmallCapsFeatures={LetterSpace=4,Ligatures=NoCommon}]{Linux Libertine O}
\setsansfont[Mapping=tex-text]{Linux Biolinum O}
\setmonofont[Mapping=tex-text]{Inconsolata}
\newfontfamily\titlefamily[Scale=1.5]{Linux Biolinum O}
\newcommand\HUGE{\fontsize{30}{30}\selectfont}
\usepackage{scrlayer-scrpage}
\pagestyle{scrheadings}
\clearscrheadfoot
\automark[chapter]{part}
\chead{\headmark}
\ohead{\thepage}
\renewcommand\partmarkformat{}
\AfterTOCHead{\singlespacing}
\setkomafont{disposition}{\normalfont\normalcolor}
\setkomafont{labelinglabel}{\normalfont\bfseries}
\setkomafont{minisec}{\usekomafont{subsection}}
\addtokomafont{pageheadfoot}{\sffamily\upshape}
\addtokomafont{caption}{\small}
\addtokomafont{captionlabel}{\bfseries}
\addtokomafont{part}{\HUGE\scshape\sffamily\lowercase}
\renewcommand*{\partformat}{\partname}
\addtokomafont{chapter}{\huge\scshape\bfseries\sffamily\lowercase}
\RedeclareSectionCommand[beforeskip=0cm,afterskip=1.5cm]{chapter}
\addtokomafont{section}{\LARGE\scshape\sffamily\lowercase}
\addtokomafont{subsection}{\large\sffamily\bfseries}
\addtokomafont{subsubsection}{\large\sffamily\itshape}
\renewcommand*{\addparttocentry}[2]{\addtocentrydefault{part}{}{\Large\scshape\sffamily\lowercase{#2}}}
\addtokomafont{chapterentry}{\normalsize\sffamily\bfseries}
\usepackage[tocflat,tocindentauto]{tocstyle}
\usetocstyle{nopagecolumn}
\unsettoc{toc}{onecolumn}
\renewcommand*\labelitemi{\adforn{33}}
\renewcommand*\labelitemii{\adforn{73}}
\renewcommand*\labelitemiii{\adforn{73}}
\renewcommand*\labelitemiv{\adforn{73}}
\renewcommand*{\dictumwidth}{.8\textwidth}
\renewcommand*{\raggeddictum}{\centering}
\renewcommand*{\raggeddictumtext}{\centering}
\addtokomafont{dictum}{\large\rmfamily}
\definecolor{bibleblue}{HTML}{00339a}
\definecolor{soothing_green}{HTML}{E1F7DB}
\theoremstyle{definition}
\newtheorem{lecture}{Lecture}
\newtheorem*{lecture*}{Lecture}
\newtheorem{problem}{Problème}
\newtheorem*{problem*}{Problème}
\newtheorem{interpretation}{Interpretation}
\newtheorem*{interpretation*}{Interpretation}
\setcounter{secnumdepth}{\partnumdepth}
\setcounter{tocdepth}{1}
\recalctypearea
\author{Carlos Alberto Rivera Carreño}
\date{}
\title{}
\hypersetup{
 pdfauthor={Carlos Alberto Rivera Carreño},
 pdftitle={},
 pdfkeywords={},
 pdfsubject={},
 pdfcreator={Emacs 26.1 (Org mode 9.1.14)}, 
 pdflang={English}}
\begin{document}

\begin{titlepage}
 \centering
% \includegraphics[width=0.5\textwidth]{logo_noir_fr.png}\par
 \vspace{4\baselineskip}
\begin{french}
 {\large Université Paris I Panthéon Sorbonne \par}
 {\large \textsc{ufr} 02 : Sciences économiques  \par}
 {\normalsize Master 2 : Économie et sciences humaines \par}
 {\normalsize 2018 \par}
\end{french}
 \vspace{2\baselineskip}
 {\huge The Quest for the Governing Machine \par}
 {\large The Rationality of Governance and the Governance of Irrationality \par}
\vspace*{\fill}
\begin{french}
 {\normalsize Présenté et sountenu par : \par}
\end{french}
 {\normalsize \textsc{carlos alberto rivera carreño}\par}
 \vspace{1\baselineskip}
\begin{french}
 {\normalsize Directeur de mémoire : \par}
\end{french}
 {\normalsize \textsc{jean sébastien lenfant}\par}
\end{titlepage}

\pagestyle{empty}

\begin{french}
L'Université Paris 1 Panthéon Sorbonne n'entend donner aucune approbation,
ni désapprobation aux opinions émises dans ce mémoire ; elle doivent être
considérées comme propres à leur auteur. 
\end{french}

\newpage
\vspace*{\fill}
\noindent
\includegraphics[height=1.5cm]{gpl3.png}\par
\vspace{1\baselineskip}
\begin{english}
This text is free: you can redistribute it and/or modify it
under the terms of the \textsc{gnu} General Public License as published by
the Free Software Foundation, either version 3 of the License or any later
version.

This text is distributed in the hope that it will be useful, but \textbf{without
any warranty}; without even the implied warranty of \textbf{merchantability or 
fitness for a particular purpose}. See the \textsc{gnu} General 
Public License for more details.

You should have received a copy of the \textsc{gnu} General Public License along
with this text. If not, see \url{http://www.gnu.org/licenses/}.

\vspace{1\baselineskip}
\noindent
Copyright \textcopyright \textsc{sync0} 2018. 
\end{english}

\newpage 

\begin{FlushRight}
\begin{spanish}
\textit{Para una lectora lejana.}
\end{spanish}
\end{FlushRight}

\newpage
\tableofcontents 

\frontmatter
\pagestyle{plain}
\chapter{Acknowledgements} 

\lipsum

\chapter{Preface} 

Even when applying different techniques of interpretation to texts, I have
tried to understand how the circumstances of their production could inform
their interpretation. Even just for the sake of consistency, shouldn't I
apply the same standard to this Master's thesis? Shouldn't I provide the
reader with the tools, the context, etc. that informed this research?

Pace Barthes, the author of these words is very alive. Knowing the tragedy
of my country, one might ask me: Why do I wield the pen, as the Colombian
fields bleed? Why not wield the sword? Or, perhaps, why not wield both?
This text was written, certainly, to satisfy an earthly requirement: to
obtain a Master's degree. But beyond that, what is the interest of writing
for me?

Although this thesis is not a political pamphlet, it is grounded in
political perplexities. To the recule of the state, who not only does not
want to fund nor organize vast sections of economic activities, my
generations witnesses a strange mix of ``innovation'',
``entrepreneurship'', and . If the political Left, of my parent's
generation had an iron faith in the working class and the peasants to
change society, today's politically dissafected youth has all hopes in
private enterprise to solve the evils of the world.

As Oscar Wilde once remarked, ``it is easier to have sympathy with
suffering, than it is to have sympathy with thought''. Even thought,

As much as this document is intended to an academic audience, it is also
intended to them and to my father: This is the beginning of a lifelong
attempt to explain \emph{¿Cómo fue que se jodio el país?} 

Should they be right, this article is wrong. 

\mainmatter
\pagestyle{scrheadings}
\part{Preparatory Research}
\label{sec:org704f466}
\chapter{First Sifting}
\label{sec:orgb3944e7}
   \begin{labeling}[~]{Subject-matter} 
\item[Subject-matter] Lorem ipsum dolor sit amet
\end{labeling}
\section{[0/17] Key text}
\label{sec:org52e59ff}
\begin{enumerate}
\item\relax [1/10] Solovey, Cold War Social Science: Knowledge Production, Liberal
Democracy, and Human Nature
\begin{itemize}
\item[{$\square$}] Solovey, Cold War Social Science: Specter, Reality, or Useful Concept?
\item[{$\square$}] Tolon, Futures Studies: A New Social Science Rooted in Cold War Strategic Thinking
\item[{$\square$}] Isaac, Epistemic Design: Theory and Data in Harvard’s Department of Social Relations
\item[{$\boxtimes$}] Heyck, Producing Reason
\item[{$\square$}] Cravens, Column Right, March! Nationalism, Scientific Positivism, and the Conservative Turn of the American Social Sciences in the Cold War Era
\item[{$\square$}] Brick, Neo- Evolutionist Anthropology, the Cold War, and the Beginnings of the World Turn in U.S. Scholarship
\item[{$\square$}] Jones-Imhotep, Maintaining Humans
\item[{$\square$}] Bycroft, Psychology, Psychologists, and the Creativity Movement: The Lives of Method Inside and Outside the Cold War
\item[{$\square$}] Weidman, An Anthropologist on TV: Ashley Montagu and the Biological
Basis of Human Nature, 1945–1960
\item[{$\square$}] Vicedo, Cold War Emotions: Mother Love and the War over Human Nature
\end{itemize}
\item\relax [0/5] Jamie, The Open Mind: Cold War Politics and the Sciences of Human
Nature
\begin{itemize}
\item[{$\square$}] Democratic Minds for a Complex Society
\item[{$\square$}] Scientists as the Model of Human Nature
\item[{$\square$}] Insituting Cognitive Science
\item[{$\square$}] Cognitive Theory and the Making of Liberal Americans
\item[{$\square$}] A Fractured Politics of Human Nature
\end{itemize}
\item[{$\square$}] Miller, 1955, Toward a general theory for the behavioral sciences
\item\relax [0/10] Supiot, 2012, La gouvernance par les nombres
\begin{itemize}
\item[{$\square$}] En quête de la machine à gouverner
\begin{itemize}
\item Poétique du gouvernement
\item L'homme machine
\item Du gouvernement à la gouvernance
\end{itemize}
\item[{$\square$}] Les aventures d'un idéal: le règne de la loi
\begin{itemize}
\item Le \emph{nomos} grec
\item La \emph{lex} en droit romain
\item La révolution gregorienne
\item \emph{Common Law} et droit continental
\item La tradition juridique occidental
\end{itemize}
\item[{$\square$}] Le rêve de l'harmonie par le calcul
\begin{itemize}
\item Les accords parfaits du nombre
\item La fonction instituante de la discorde
\end{itemize}
\item[{$\square$}] L'essor des usages normatifs de la quantification
\begin{itemize}
\item Rendre compte
\item Administrer
\item Juger
\item Légiférer
\end{itemize}
\item[{$\square$}] L’asservissement de la Loi au Nombre: du Gosplan au Marché total
\begin{itemize}
\item Le renversement du règne de la loi
\item Le droit, outil de planification
\item L'hybridation du communisme et capitalisme
\end{itemize}
\item[{$\square$}] Calculer l'incalculable: la doctrine Law and Economics
\begin{itemize}
\item La théorie des jeux
\item La théorie de l'agence
\item Le théorème de Coase et la théorie des \emph{property rights}
\item La \emph{New Comparative Analysis} et le marché du droit
\end{itemize}
\item[{$\square$}] La dynamique juridique de la gouvernance par les nombres
\begin{itemize}
\item La gouvernance individuelle
\item La gouvernance de l’entreprise
\item La gouvernance étatique
\item La gouvernance européenne
\item La gouvernance mondiale
\end{itemize}
\item[{$\square$}] Les impasses de la gouvernance par les nombres
\begin{itemize}
\item Les effets de structure de la gouvernance par les nombres
\item Les résistances du Droit à la gouvernance par les nombres
\end{itemize}
\item[{$\square$}] Le dépérissement de l'état
\begin{itemize}
\item La sacralité de la chose publique
\item La direction scientifique des hommes
\item L'inversion de la hiérarchie publique/privé
\item La loi pour soi et soi pour la loi
\item Sans foi ni loi: la société insoutenable
\end{itemize}
\item[{$\square$}] Comment en sortir
\end{itemize}
\item[{$\square$}] Ross, 1994, Modernist Impulses in the Human Sciences, 1870-1930
\item[{$\square$}] Purcell, 1973, The Crisis of Democratic Theory: Scientific Naturalism and the Problem of Value
\item[{$\square$}] Deutsch, 1963, The Nerves of Government: Models of Political Communication and Control
\item\relax [0/15] Heyck, Herbert Simon: The Bounds of Reason in Modern America
\begin{itemize}
\item[{$\square$}] Unbounded rationality
\item[{$\square$}] The garden of forking paths
\item[{$\square$}] The Chicago school and the sciences of control
\item[{$\square$}] Mathematics, logic, and the sciences of choice
\item[{$\square$}] Research and reform
\item[{$\square$}] \emph{Homo administrativus}, or Choice under control
\item[{$\square$}] Decisions and revisions
\item[{$\square$}] Structuring his environment
\item[{$\square$}] Islands of theory
\item[{$\square$}] A new model of mind and machine
\item[{$\square$}] The program \emph{is} the theory
\item[{$\square$}] The cognitive revolution
\item[{$\square$}] \emph{Homo adaptativus}, the Finite problem solver
\item[{$\square$}] Scientist of the artificial
\item[{$\square$}] The expert problem solver
\end{itemize}
\item\relax [0/8] Heyck, Age of System: Understanding the development of modern social science
\begin{itemize}
\item[{$\square$}] The Organizational Revolution and the Human Sciences
\item[{$\square$}] High modern social science: A bird's eye view
\item[{$\square$}] Patrons of the revolution: Ideas, Ideals, and Institutions in Postwar Social Science
\item[{$\square$}] The magical year 1956, plus or minus one
\item[{$\square$}] Producing reason
\item[{$\square$}] Modernity and social change in American social science
\item[{$\square$}] A model science?
\item[{$\square$}] History and Legacy, Tree and the Web
\end{itemize}
\item[{$\square$}] Heyck, Mind and Network
\item[{$\square$}] Heyck, Georges Miller, language, and the computer metaphor of mind
\item[{$\square$}] Heyck, Defining the Computer: Herbert Simon and the Bureaucratic Mind, Part 1
\item[{$\square$}] Heyck Defining the Computer: Herbert Simon and the Bureaucratic Mind, Part 2
\item\relax [0/5] Ronald Kline, The Cybernetics Moment
\begin{itemize}
\item[{$\square$}] War and Information Theory
\item[{$\square$}] The Cybernetics Craze
\item[{$\square$}] The Information Bandwagon
\item[{$\square$}] Machines as Humans
\item[{$\square$}] Humans as Machines
\end{itemize}
\item\relax [0/4] Maas, William Stanley Jevons and the Making of Modern Economics
\begin{itemize}
\item[{$\square$}] The Prying Eyes of the Natural Scientist
\item[{$\square$}] Engines of Discovery
\begin{itemize}
\item Babbage and his calculating engines
\item God is a programmer
\item An intelligent machine
\item Is the mind a reasoning machine?
\end{itemize}
\item[{$\square$}] The Machinery of the Mind
\begin{itemize}
\item The Logical Abacus
\item The Logical Machine
\item The machine of the mind
\item Induction - the inverse of deduction
\item To decide what things are similar
\end{itemize}
\item[{$\square$}] The Image of Economics
\begin{itemize}
\item Bridging the natural and the social
\item Mechanical dreams
\item Economics as natural science
\end{itemize}
\end{itemize}
\item\relax [1/8] Mirowski, Machine Dreams
\begin{itemize}
\item[{$\boxtimes$}] Cyborg Agonists
\begin{itemize}
\item[{$\boxtimes$}] Rooms with a view
\item[{$\boxtimes$}] Where the cyborgs are
\item[{$\boxtimes$}] The natural sciences and the history of economics
\item[{$\boxtimes$}] Anatomy of a cyborg
\item[{$\boxtimes$}] Attack of the cyborgs
\item[{$\boxtimes$}] The new automaton theatre
\end{itemize}
\item[{$\square$}] Some Cyborg Genealogies; or How the Demon Got Its Bots
\begin{itemize}
\item[{$\square$}] The little engines that could've
\item[{$\square$}] Adventures of a red-hot demon
\item[{$\square$}] Cybernetics
\item[{$\square$}] The devil that made us do it
\item[{$\square$}] The advent of complexity
\end{itemize}
\item[{$\square$}] John von Neumann and the Cyborg Incursion into Economics
\begin{itemize}
\item[{$\square$}] Economics at one remove
\item[{$\square$}] Purity
\item[{$\square$}] Impurity
\item[{$\square$}] Wordliness
\end{itemize}
\item[{$\square$}] The Military, the Scientist, and the Revised Rules of the Game
\begin{itemize}
\item[{$\square$}] What did you do in the war, daddy?
\item[{$\square$}] The cybord character of science mobilization in the WWII
\item[{$\square$}] Operations Research
\item[{$\square$}] The Ballad of Hotelling and Schultz
\item[{$\square$}] SRG, RAND, Rad Lab
\end{itemize}
\item[{$\square$}] Do Cyborgs Dream of Efficient Markets?
\begin{itemize}
\item[{$\square$}] From Red Vienna to Computopia
\item[{$\square$}] The Goals of Cowles, and Red Afterglows
\item[{$\square$}] Every Man His Own Stat Package
\item[{$\square$}] On the Impossibility of a Democratic Computer
\end{itemize}
\item[{$\square$}] The Empire Strikes Back
\begin{itemize}
\item[{$\square$}] Previews of Cunning Abstractions
\item[{$\square$}] Its a World Eat World Dog: Game Theory at RAND
\item[{$\square$}] The High Cost of Information in Postwar Neoclassical Theory
\item[{$\square$}] Rigor Mortis in the First Casualty of War
\item[{$\square$}] Does the Rational Agent Compute?
\end{itemize}
\item[{$\square$}] Core Wars
\begin{itemize}
\item[{$\square$}] Inhuman, All Too Inhuman
\item[{$\square$}] Herbert Simon: Simulacra vs Automata
\item[{$\square$}] Showdown at the OR Corral
\item[{$\square$}] Send in the Clones
\end{itemize}
\item[{$\square$}] Machines Who Think vs Machines that Sell
\begin{itemize}
\item[{$\square$}] Where is the Computer Taking Us?
\item[{$\square$}] Five Alternative Scenarios for the Future of Computational
Economics
\item[{$\square$}] They Hayek Hypothesis and Experimental Economics
\item[{$\square$}] Gode and Sunder Go Roboshoppin
\item[{$\square$}] Contingency, Irony, and Computation
\end{itemize}
\end{itemize}
\item\relax [0/17] Mirowski, The Knowledge We Lost in Information
\begin{itemize}
\item[{$\square$}] It's not Rational
\item[{$\square$}] The Standard Narrative and the Bigger Picture
\item[{$\square$}] Natural Science Inspirations
\item[{$\square$}] The Nobels and the Neoliberals
\item[{$\square$}] The Socialist Calculation Controversy as the Starting Point of the
Economics of Information
\item[{$\square$}] Hayek Changes his Mind
\item[{$\square$}] The Neoclassical Economics of Information Was Incubated at Cowles
\item[{$\square$}] Three Different Modalities of Information in Neoclassical Theory
\item[{$\square$}] Going the Market One Better
\item[{$\square$}] The History of Markets and the Theory of Market Design
\item[{$\square$}] The Walrasian School of Design
\item[{$\square$}] The Bayes-Nash School of Design
\item[{$\square$}] The Experimentalist School of Design
\item[{$\square$}] Hayek and the Schools of Design
\item[{$\square$}] Designs on the Market: The FCC Spectrum Auctions
\item[{$\square$}] Private Intellectuals and Public Perplexity : The TARP
\item[{$\square$}] Artificial Ignorance
\end{itemize}
\item\relax [2/4] Backhouse, New Directions in Economic Methodology
\begin{itemize}
\item[{$\square$}] McCloskey, How to Do a Rhetorical Analysis, and Why
\item[{$\square$}] Lawson, A Realist Theory for Economics
\item[{$\boxtimes$}] Mirowski, What are the Questions?
\item[{$\boxtimes$}] Henderson, Metaphor and Economics
\end{itemize}
\item[{$\square$}] Backhouse, The unsocial social science: Economics and Neighboring Disciplines Since 1945
\item\relax [0/3] Backhouse, They History of the Social Sciences since 1945
\begin{itemize}
\item[{$\square$}] Ash, Psychology
\item[{$\square$}] Backhouse, Economics
\item[{$\square$}] Bevir, Political Science
\end{itemize}
\item[{$\square$}] Gigerenzer, Mind as Computer: Birth of a Metaphor
\item[{$\square$}] Marshall, Minds, Machines and Metaphors
\item[{$\square$}] Vicedo, Cold War emotions: The war over human nature
\item\relax [0/1] Dupuy, Aux origines des sciences cognitives
\begin{itemize}
\item[{$\square$}] 
\end{itemize}
\item[{$\square$}] Chomsky, The Cold War \& the University: Toward an Intellectual History of the Postwar Years
\item[{$\square$}] Mikulark, ``Cybernetics and Marxism-Leninism'' in The Social Impact of Cybernetics, ed. Charles Dechert
\item[{$\square$}] Israel,  Meccanicismo
\item[{$\square$}] Israel, La machina vivente: contre le visione meccanicistiche del uomo
\item[{$\square$}] Edwards, 1996, The Closed World: Computers and the Politics of Discourse in Cold War America
\item\relax [0/9] Amadae, Rationalizing Capitalist Democracy: The Cold War Origins of
Rational Choice Liberalism
\begin{itemize}
\item[{$\square$}] Managing the National Securtity State: Decision Technologies and Policy Science
\item[{$\square$}] Arrow's Social Choice and Individual Values
\item[{$\square$}] Buchanan and Tullocks' Public Choice Theory
\item[{$\square$}] Riker's Positive Political Theory
\item[{$\square$}] Rational Choice and Capitalist Democracy
\item[{$\square$}] Adam Smith's System of Natural Liberty
\item[{$\square$}] Rational Mechanics, Marginalist Economics, and Rational Choice
\item[{$\square$}] Consolidating Rational Choice Liberalism 1970-2000
\item[{$\square$}] From the Panopticon to the Prisoner's Dilemma
\end{itemize}
\end{enumerate}
\section{[0/23] Important text}
\label{sec:orge455609}
\begin{itemize}
\item\relax [1/11] Solovey, Cold War Social Science: Knowledge Production, Liberal
Democracy, and Human Nature
\begin{itemize}
\item[{$\square$}] Solovey, Cold War Social Science: Specter, Reality, or Useful Concept?
\item[{$\square$}] Tolon, Futures Studies: A New Social Science Rooted in Cold War Strategic Thinking
\item[{$\square$}] Martin-Nilsen, “It Was All Connected”: Computers and Linguistics in Early Cold War America
\item[{$\square$}] Isaac, Epistemic Design: Theory and Data in Harvard’s Department of Social Relations
\item[{$\boxtimes$}] Heyck, Producing Reason
\item[{$\square$}] Cravens, Column Right, March! Nationalism, Scientific Positivism, and the Conservative Turn of the American Social Sciences in the Cold War Era
\item[{$\square$}] Brick, Neo- Evolutionist Anthropology, the Cold War, and the Beginnings of the World Turn in U.S. Scholarship
\item[{$\square$}] Jones-Imhotep, Maintaining Humans
\item[{$\square$}] Bycroft, Psychology, Psychologists, and the Creativity Movement: The Lives of Method Inside and Outside the Cold War
\item[{$\square$}] Weidman, An Anthropologist on TV: Ashley Montagu and the Biological
Basis of Human Nature, 1945–1960
\item[{$\square$}] Vicedo, Cold War Emotions: Mother Love and the War over Human Nature
\end{itemize}
\item\relax [0/5] Jamie, The Open Mind: Cold War Politics and the Sciences of Human
Nature
\begin{itemize}
\item[{$\square$}] Democratic Minds for a Complex Society
\item[{$\square$}] Scientists as the Model of Human Nature
\item[{$\square$}] Insituting Cognitive Science
\item[{$\square$}] Cognitive Theory and the Making of Liberal Americans
\item[{$\square$}] A Fractured Politics of Human Nature
\end{itemize}
\item[{$\square$}] Miller, 1955, Toward a general theory for the behavioral sciences
\item\relax [0/15] Supiot, 2012, La gouvernance par les nombres
\begin{itemize}
\item[{$\square$}] En quête de la machine à gouverner
\begin{itemize}
\item Poétique du gouvernement
\item L'homme machine
\item Du gouvernement à la gouvernance
\end{itemize}
\item[{$\square$}] Les aventures d'un idéal: le règne de la loi
\begin{itemize}
\item Le \emph{nomos} grec
\item La \emph{lex} en droit romain
\item La révolution gregorienne
\item \emph{Common Law} et droit continental
\item La tradition juridique occidental
\end{itemize}
\item[{$\square$}] Autres points de vue sur les lois
\item[{$\square$}] Le rêve de l'harmonie par le calcul
\begin{itemize}
\item Les accords parfaits du nombre
\item La fonction instituante de la discorde
\end{itemize}
\item[{$\square$}] L'essor des usages normatifs de la quantification
\begin{itemize}
\item Rendre compte
\item Administrer
\item Juger
\item Légiférer
\end{itemize}
\item[{$\square$}] L’asservissement de la Loi au Nombre: du Gosplan au Marché total
\begin{itemize}
\item Le renversement du règne de la loi
\item Le droit, outil de planification
\item L'hybridation du communisme et capitalisme
\end{itemize}
\item[{$\square$}] Calculer l'incalculable: la doctrine Law and Economics
\begin{itemize}
\item La théorie des jeux
\item La théorie de l'agence
\item Le théorème de Coase et la théorie des \emph{property rights}
\item La \emph{New Comparative Analysis} et le marché du droit
\end{itemize}
\item[{$\square$}] La dynamique juridique de la gouvernance par les nombres
\begin{itemize}
\item La gouvernance individuelle
\item La gouvernance de l’entreprise
\item La gouvernance étatique
\item La gouvernance européenne
\item La gouvernance mondiale
\end{itemize}
\item[{$\square$}] Les impasses de la gouvernance par les nombres
\begin{itemize}
\item Les effets de structure de la gouvernance par les nombres
\item Les résistances du Droit à la gouvernance par les nombres
\end{itemize}
\item[{$\square$}] Le dépérissement de l'état
\begin{itemize}
\item La sacralité de la chose publique
\item La direction scientifique des hommes
\item L'inversion de la hiérarchie publique/privé
\item La loi pour soi et soi pour la loi
\item Sans foi ni loi: la société insoutenable
\end{itemize}
\item[{$\square$}] La résurgence du gouvernement par les hommes
\item[{$\square$}] De la mobilisation totale à la crise du Fordisme
\begin{itemize}
\item Le compromis Fordiste
\item La déconstruction du droit du travail
\item Les voies d'un nouveau compromis
\end{itemize}
\item[{$\square$}] De l'échange quantifié à l’allégeance des personnes
\begin{itemize}
\item La mobilisation totale au travail
\item Les nouveaux droits attachés à la personne
\end{itemize}
\item[{$\square$}] La structure des liens d’allégeance
\begin{itemize}
\item L'allégeance dans les réseaux d'entreprises
\item L'allégeance des multinationales aux États impériaux
\end{itemize}
\item[{$\square$}] Comment en sortir
\end{itemize}
\item[{$\square$}] Hughes, 1958, Consciousness and Society: The Reorientation of European Social Thought, 1890-1930
\item[{$\square$}] Ross, 1994, Modernist Impulses in the Human Sciences, 1870-1930
\item[{$\square$}] Purcell, 1973, The Crisis of Democratic Theory: Scientific Naturalism and the Problem of Value
\item[{$\square$}] Butsch, 2008, The Citizen Audience: Crowds, Publics, and Individuals
\item[{$\square$}] Deutsch, 1963, The Nerves of Government: Models of Political Communication and Control
\item[{$\square$}] Cohen-Cole, 2009, The Creative American: Cold War salons, social science, and the cure for modern society.
\item\relax [0/15] Heyck, Herbert Simon: The Bounds of Reason in Modern America
\begin{itemize}
\item[{$\square$}] Unbounded rationality
\item[{$\square$}] The garden of forking paths
\item[{$\square$}] The Chicago school and the sciences of control
\item[{$\square$}] Mathematics, logic, and the sciences of choice
\item[{$\square$}] Research and reform
\item[{$\square$}] \emph{Homo administrativus}, or Choice under control
\item[{$\square$}] Decisions and revisions
\item[{$\square$}] Structuring his environment
\item[{$\square$}] Islands of theory
\item[{$\square$}] A new model of mind and machine
\item[{$\square$}] The program \emph{is} the theory
\item[{$\square$}] The cognitive revolution
\item[{$\square$}] \emph{Homo adaptativus}, the Finite problem solver
\item[{$\square$}] Scientist of the artificial
\item[{$\square$}] The expert problem solver
\end{itemize}
\item\relax [0/8] Heyck, Age of System: Understanding the development of modern social science
\begin{itemize}
\item[{$\square$}] The Organizational Revolution and the Human Sciences
\item[{$\square$}] High modern social science: A bird's eye view
\item[{$\square$}] Patrons of the revolution: Ideas, Ideals, and Institutions in Postwar Social Science
\item[{$\square$}] The magical year 1956, plus or minus one
\item[{$\square$}] Producing reason
\item[{$\square$}] Modernity and social change in American social science
\item[{$\square$}] A model science?
\item[{$\square$}] History and Legacy, Tree and the Web
\end{itemize}
\item[{$\square$}] Heyck, Mind and Network
\item[{$\square$}] Heyck, Georges Miller, language, and the computer metaphor of mind
\item[{$\square$}] Heyck, Defining the Computer: Herbert Simon and the Bureaucratic Mind, Part 1
\item[{$\square$}] Heyck Defining the Computer: Herbert Simon and the Bureaucratic Mind, Part 2
\item\relax [0/9] Ronald Kline, The Cybernetics Moment
\begin{itemize}
\item[{$\square$}] War and Information Theory
\item[{$\square$}] Circular Causality
\item[{$\square$}] The Cybernetics Craze
\item[{$\square$}] The Information Bandwagon
\item[{$\square$}] Machines as Humans
\item[{$\square$}] Humans as Machines
\item[{$\square$}] Cybernetics in Crisis
\item[{$\square$}] Inventing an Information Age
\item[{$\square$}] Two Cybernetic Frontiers
\end{itemize}
\item\relax [0/1] Koyré, Études d'histoire de la pensée philosophique
\begin{itemize}
\item[{$\square$}] Les philosophes et la machine
\begin{itemize}
\item L'appreciation du machinisme
\item Les origines du machinisme
\end{itemize}
\end{itemize}
\item\relax [0/5] Maas, William Stanley Jevons and the Making of Modern Economics
\begin{itemize}
\item[{$\square$}] The Prying Eyes of the Natural Scientist
\item[{$\square$}] Engines of Discovery
\begin{itemize}
\item Babbage and his calculating engines
\item God is a programmer
\item An intelligent machine
\item Is the mind a reasoning machine?
\end{itemize}
\item[{$\square$}] The Machinery of the Mind
\begin{itemize}
\item The Logical Abacus
\item The Logical Machine
\item The machine of the mind
\item Induction - the inverse of deduction
\item To decide what things are similar
\end{itemize}
\item[{$\square$}] The Laws of Human Enjoyment
\begin{itemize}
\item The factory system and the division of labor
\item Ruskin's aesthetic-driven criticism of the factory system
\item Mill and the gospel of work
\item Work and fatigue
\end{itemize}
\item[{$\square$}] The Image of Economics
\begin{itemize}
\item Bridging the natural and the social
\item Mechanical dreams
\item Economics as natural science
\end{itemize}
\end{itemize}
\item\relax [0/8] Mirowski, Machine Dreams
\begin{itemize}
\item[{$\square$}] Cyborg Agonists
\begin{itemize}
\item[{$\square$}] Rooms with a view
\item[{$\square$}] Where the cyborgs are
\item[{$\square$}] The natural sciences and the history of economics
\item[{$\square$}] Anatomy of a cyborg
\item[{$\square$}] Attack of the cyborgs
\item[{$\square$}] The new automaton theatre
\end{itemize}
\item[{$\square$}] Some Cyborg Genealogies; or How the Demon Got Its Bots
\begin{itemize}
\item[{$\square$}] The little engines that could've
\item[{$\square$}] Adventures of a red-hot demon
\item[{$\square$}] Cybernetics
\item[{$\square$}] The devil that made us do it
\item[{$\square$}] The advent of complexity
\end{itemize}
\item[{$\square$}] John von Neumann and the Cyborg Incursion into Economics
\begin{itemize}
\item[{$\square$}] Economics at one remove
\item[{$\square$}] Purity
\item[{$\square$}] Impurity
\item[{$\square$}] Wordliness
\end{itemize}
\item[{$\square$}] The Military, the Scientist, and the Revised Rules of the Game
\begin{itemize}
\item[{$\square$}] What did you do in the war, daddy?
\item[{$\square$}] The cybord character of science mobilization in the WWII
\item[{$\square$}] Operations Research
\item[{$\square$}] The Ballad of Hotelling and Schultz
\item[{$\square$}] SRG, RAND, Rad Lab
\end{itemize}
\item[{$\square$}] Do Cyborgs Dream of Efficient Markets?
\begin{itemize}
\item[{$\square$}] From Red Vienna to Computopia
\item[{$\square$}] The Goals of Cowles, and Red Afterglows
\item[{$\square$}] Every Man His Own Stat Package
\item[{$\square$}] On the Impossibility of a Democratic Computer
\end{itemize}
\item[{$\square$}] The Empire Strikes Back
\begin{itemize}
\item[{$\square$}] Previews of Cunning Abstractions
\item[{$\square$}] Its a World Eat World Dog: Game Theory at RAND
\item[{$\square$}] The High Cost of Information in Postwar Neoclassical Theory
\item[{$\square$}] Rigor Mortis in the First Casualty of War
\item[{$\square$}] Does the Rational Agent Compute?
\end{itemize}
\item[{$\square$}] Core Wars
\begin{itemize}
\item[{$\square$}] Inhuman, All Too Inhuman
\item[{$\square$}] Herbert Simon: Simulacra vs Automata
\item[{$\square$}] Showdown at the OR Corral
\item[{$\square$}] Send in the Clones
\end{itemize}
\item[{$\square$}] Machines Who Think vs Machines that Sell
\begin{itemize}
\item[{$\square$}] Where is the Computer Taking Us?
\item[{$\square$}] Five Alternative Scenarios for the Future of Computational
Economics
\item[{$\square$}] They Hayek Hypothesis and Experimental Economics
\item[{$\square$}] Gode and Sunder Go Roboshoppin
\item[{$\square$}] Contingency, Irony, and Computation
\end{itemize}
\end{itemize}
\item[{$\square$}] Mirowski, More Heat than Light
\item[{$\square$}] Mirowski, Against Mechanism
\item\relax [0/17] Mirowski, The Knowledge We Lost in Information
\begin{itemize}
\item[{$\square$}] It's not Rational
\item[{$\square$}] The Standard Narrative and the Bigger Picture
\item[{$\square$}] Natural Science Inspirations
\item[{$\square$}] The Nobels and the Neoliberals
\item[{$\square$}] The Socialist Calculation Controversy as the Starting Point of the
Economics of Information
\item[{$\square$}] Hayek Changes his Mind
\item[{$\square$}] The Neoclassical Economics of Information Was Incubated at Cowles
\item[{$\square$}] Three Different Modalities of Information in Neoclassical Theory
\item[{$\square$}] Going the Market One Better
\item[{$\square$}] The History of Markets and the Theory of Market Design
\item[{$\square$}] The Walrasian School of Design
\item[{$\square$}] The Bayes-Nash School of Design
\item[{$\square$}] The Experimentalist School of Design
\item[{$\square$}] Hayek and the Schools of Design
\item[{$\square$}] Designs on the Market: The FCC Spectrum Auctions
\item[{$\square$}] Private Intellectuals and Public Perplexity : The TARP
\item[{$\square$}] Artificial Ignorance
\end{itemize}
\item\relax [2/4] Backhouse, New Directions in Economic Methodology
\begin{itemize}
\item[{$\square$}] McCloskey, How to Do a Rhetorical Analysis, and Why
\item[{$\square$}] Lawson, A Realist Theory for Economics
\item[{$\boxtimes$}] Mirowski, What are the Questions?
\item[{$\boxtimes$}] Henderson, Metaphor and Economics
\end{itemize}
\item[{$\square$}] Backhouse, The unsocial social science: Economics and Neighboring Disciplines Since 1945
\item\relax [0/3] Backhouse, They History of the Social Sciences since 1945
\begin{itemize}
\item[{$\square$}] Ash, Psychology
\item[{$\square$}] Backhouse, Economics
\item[{$\square$}] Bevir, Political Science
\end{itemize}
\item[{$\square$}] Gigerenzer, Mind as Computer: Birth of a Metaphor
\item[{$\square$}] Marshall, Minds, Machines and Metaphors
\item[{$\square$}] Vicedo, Cold War emotions: The war over human nature
\item[{$\square$}] Dupuy, Aux origines des sciences cognitives
\item[{$\square$}] Chomsky, The Cold War \& the University: Toward an Intellectual History of the Postwar Years
\item[{$\square$}] Mikulark, ``Cybernetics and Marxism-Leninism'' in The Social Impact of Cybernetics, ed. Charles Dechert
\item[{$\square$}] Israel,  Meccanicismo
\item[{$\square$}] Israel, La machina vivente: contre le visione meccanicistiche del uomo
\item[{$\square$}] Edwards, 1996, The Closed World: Computers and the Politics of Discourse in Cold War America
\item\relax [0/9] Amadae, Rationalizing Capitalist Democracy: The Cold War Origins of
Rational Choice Liberalism
\begin{itemize}
\item[{$\square$}] Managing the National Securtity State: Decision Technologies and Policy Science
\item[{$\square$}] Arrow's Social Choice and Individual Values
\item[{$\square$}] Buchanan and Tullocks' Public Choice Theory
\item[{$\square$}] Riker's Positive Political Theory
\item[{$\square$}] Rational Choice and Capitalist Democracy
\item[{$\square$}] Adam Smith's System of Natural Liberty
\item[{$\square$}] Rational Mechanics, Marginalist Economics, and Rational Choice
\item[{$\square$}] Consolidating Rational Choice Liberalism 1970-2000
\item[{$\square$}] From the Panopticon to the Prisoner's Dilemma
\end{itemize}
\end{itemize}
\section{[0/23] Ancillary text}
\label{sec:org2a45ead}
\begin{itemize}
\item\relax [1/11] Solovey, Cold War Social Science: Knowledge Production, Liberal
Democracy, and Human Nature
\begin{itemize}
\item[{$\square$}] Solovey, Cold War Social Science: Specter, Reality, or Useful Concept?
\item[{$\square$}] Tolon, Futures Studies: A New Social Science Rooted in Cold War Strategic Thinking
\item[{$\square$}] Martin-Nilsen, “It Was All Connected”: Computers and Linguistics in Early Cold War America
\item[{$\square$}] Isaac, Epistemic Design: Theory and Data in Harvard’s Department of Social Relations
\item[{$\boxtimes$}] Heyck, Producing Reason
\item[{$\square$}] Cravens, Column Right, March! Nationalism, Scientific Positivism, and the Conservative Turn of the American Social Sciences in the Cold War Era
\item[{$\square$}] Brick, Neo- Evolutionist Anthropology, the Cold War, and the Beginnings of the World Turn in U.S. Scholarship
\item[{$\square$}] Jones-Imhotep, Maintaining Humans
\item[{$\square$}] Bycroft, Psychology, Psychologists, and the Creativity Movement: The Lives of Method Inside and Outside the Cold War
\item[{$\square$}] Weidman, An Anthropologist on TV: Ashley Montagu and the Biological
Basis of Human Nature, 1945–1960
\item[{$\square$}] Vicedo, Cold War Emotions: Mother Love and the War over Human Nature
\end{itemize}
\item\relax [0/5] Jamie, The Open Mind: Cold War Politics and the Sciences of Human
Nature
\begin{itemize}
\item[{$\square$}] Democratic Minds for a Complex Society
\item[{$\square$}] Scientists as the Model of Human Nature
\item[{$\square$}] Insituting Cognitive Science
\item[{$\square$}] Cognitive Theory and the Making of Liberal Americans
\item[{$\square$}] A Fractured Politics of Human Nature
\end{itemize}
\item[{$\square$}] Miller, 1955, Toward a general theory for the behavioral sciences
\item\relax [0/15] Supiot, 2012, La gouvernance par les nombres
\begin{itemize}
\item[{$\square$}] En quête de la machine à gouverner
\begin{itemize}
\item Poétique du gouvernement
\item L'homme machine
\item Du gouvernement à la gouvernance
\end{itemize}
\item[{$\square$}] Les aventures d'un idéal: le règne de la loi
\begin{itemize}
\item Le \emph{nomos} grec
\item La \emph{lex} en droit romain
\item La révolution gregorienne
\item \emph{Common Law} et droit continental
\item La tradition juridique occidental
\end{itemize}
\item[{$\square$}] Autres points de vue sur les lois
\item[{$\square$}] Le rêve de l'harmonie par le calcul
\begin{itemize}
\item Les accords parfaits du nombre
\item La fonction instituante de la discorde
\end{itemize}
\item[{$\square$}] L'essor des usages normatifs de la quantification
\begin{itemize}
\item Rendre compte
\item Administrer
\item Juger
\item Légiférer
\end{itemize}
\item[{$\square$}] L’asservissement de la Loi au Nombre: du Gosplan au Marché total
\begin{itemize}
\item Le renversement du règne de la loi
\item Le droit, outil de planification
\item L'hybridation du communisme et capitalisme
\end{itemize}
\item[{$\square$}] Calculer l'incalculable: la doctrine Law and Economics
\begin{itemize}
\item La théorie des jeux
\item La théorie de l'agence
\item Le théorème de Coase et la théorie des \emph{property rights}
\item La \emph{New Comparative Analysis} et le marché du droit
\end{itemize}
\item[{$\square$}] La dynamique juridique de la gouvernance par les nombres
\begin{itemize}
\item La gouvernance individuelle
\item La gouvernance de l’entreprise
\item La gouvernance étatique
\item La gouvernance européenne
\item La gouvernance mondiale
\end{itemize}
\item[{$\square$}] Les impasses de la gouvernance par les nombres
\begin{itemize}
\item Les effets de structure de la gouvernance par les nombres
\item Les résistances du Droit à la gouvernance par les nombres
\end{itemize}
\item[{$\square$}] Le dépérissement de l'état
\begin{itemize}
\item La sacralité de la chose publique
\item La direction scientifique des hommes
\item L'inversion de la hiérarchie publique/privé
\item La loi pour soi et soi pour la loi
\item Sans foi ni loi: la société insoutenable
\end{itemize}
\item[{$\square$}] La résurgence du gouvernement par les hommes
\item[{$\square$}] De la mobilisation totale à la crise du Fordisme
\begin{itemize}
\item Le compromis Fordiste
\item La déconstruction du droit du travail
\item Les voies d'un nouveau compromis
\end{itemize}
\item[{$\square$}] De l'échange quantifié à l’allégeance des personnes
\begin{itemize}
\item La mobilisation totale au travail
\item Les nouveaux droits attachés à la personne
\end{itemize}
\item[{$\square$}] La structure des liens d’allégeance
\begin{itemize}
\item L'allégeance dans les réseaux d'entreprises
\item L'allégeance des multinationales aux États impériaux
\end{itemize}
\item[{$\square$}] Comment en sortir
\end{itemize}
\item[{$\square$}] Hughes, 1958, Consciousness and Society: The Reorientation of European Social Thought, 1890-1930
\item[{$\square$}] Ross, 1994, Modernist Impulses in the Human Sciences, 1870-1930
\item[{$\square$}] Purcell, 1973, The Crisis of Democratic Theory: Scientific Naturalism and the Problem of Value
\item[{$\square$}] Butsch, 2008, The Citizen Audience: Crowds, Publics, and Individuals
\item[{$\square$}] Deutsch, 1963, The Nerves of Government: Models of Political Communication and Control
\item[{$\square$}] Cohen-Cole, 2009, The Creative American: Cold War salons, social science, and the cure for modern society.
\item\relax [0/15] Heyck, Herbert Simon: The Bounds of Reason in Modern America
\begin{itemize}
\item[{$\square$}] Unbounded rationality
\item[{$\square$}] The garden of forking paths
\item[{$\square$}] The Chicago school and the sciences of control
\item[{$\square$}] Mathematics, logic, and the sciences of choice
\item[{$\square$}] Research and reform
\item[{$\square$}] \emph{Homo administrativus}, or Choice under control
\item[{$\square$}] Decisions and revisions
\item[{$\square$}] Structuring his environment
\item[{$\square$}] Islands of theory
\item[{$\square$}] A new model of mind and machine
\item[{$\square$}] The program \emph{is} the theory
\item[{$\square$}] The cognitive revolution
\item[{$\square$}] \emph{Homo adaptativus}, the Finite problem solver
\item[{$\square$}] Scientist of the artificial
\item[{$\square$}] The expert problem solver
\end{itemize}
\item\relax [0/8] Heyck, Age of System: Understanding the development of modern social science
\begin{itemize}
\item[{$\square$}] The Organizational Revolution and the Human Sciences
\item[{$\square$}] High modern social science: A bird's eye view
\item[{$\square$}] Patrons of the revolution: Ideas, Ideals, and Institutions in Postwar Social Science
\item[{$\square$}] The magical year 1956, plus or minus one
\item[{$\square$}] Producing reason
\item[{$\square$}] Modernity and social change in American social science
\item[{$\square$}] A model science?
\item[{$\square$}] History and Legacy, Tree and the Web
\end{itemize}
\item[{$\square$}] Heyck, Mind and Network
\item[{$\square$}] Heyck, Georges Miller, language, and the computer metaphor of mind
\item[{$\square$}] Heyck, Defining the Computer: Herbert Simon and the Bureaucratic Mind, Part 1
\item[{$\square$}] Heyck Defining the Computer: Herbert Simon and the Bureaucratic Mind, Part 2
\item\relax [0/9] Ronald Kline, The Cybernetics Moment
\begin{itemize}
\item[{$\square$}] War and Information Theory
\item[{$\square$}] Circular Causality
\item[{$\square$}] The Cybernetics Craze
\item[{$\square$}] The Information Bandwagon
\item[{$\square$}] Machines as Humans
\item[{$\square$}] Humans as Machines
\item[{$\square$}] Cybernetics in Crisis
\item[{$\square$}] Inventing an Information Age
\item[{$\square$}] Two Cybernetic Frontiers
\end{itemize}
\item\relax [0/1] Koyré, Études d'histoire de la pensée philosophique
\begin{itemize}
\item[{$\square$}] Les philosophes et la machine
\begin{itemize}
\item L'appreciation du machinisme
\item Les origines du machinisme
\end{itemize}
\end{itemize}
\item\relax [0/5] Maas, William Stanley Jevons and the Making of Modern Economics
\begin{itemize}
\item[{$\square$}] The Prying Eyes of the Natural Scientist
\item[{$\square$}] Engines of Discovery
\begin{itemize}
\item Babbage and his calculating engines
\item God is a programmer
\item An intelligent machine
\item Is the mind a reasoning machine?
\end{itemize}
\item[{$\square$}] The Machinery of the Mind
\begin{itemize}
\item The Logical Abacus
\item The Logical Machine
\item The machine of the mind
\item Induction - the inverse of deduction
\item To decide what things are similar
\end{itemize}
\item[{$\square$}] The Laws of Human Enjoyment
\begin{itemize}
\item The factory system and the division of labor
\item Ruskin's aesthetic-driven criticism of the factory system
\item Mill and the gospel of work
\item Work and fatigue
\end{itemize}
\item[{$\square$}] The Image of Economics
\begin{itemize}
\item Bridging the natural and the social
\item Mechanical dreams
\item Economics as natural science
\end{itemize}
\end{itemize}
\item\relax [0/8] Mirowski, Machine Dreams
\begin{itemize}
\item[{$\square$}] Cyborg Agonists
\begin{itemize}
\item[{$\square$}] Rooms with a view
\item[{$\square$}] Where the cyborgs are
\item[{$\square$}] The natural sciences and the history of economics
\item[{$\square$}] Anatomy of a cyborg
\item[{$\square$}] Attack of the cyborgs
\item[{$\square$}] The new automaton theatre
\end{itemize}
\item[{$\square$}] Some Cyborg Genealogies; or How the Demon Got Its Bots
\begin{itemize}
\item[{$\square$}] The little engines that could've
\item[{$\square$}] Adventures of a red-hot demon
\item[{$\square$}] Cybernetics
\item[{$\square$}] The devil that made us do it
\item[{$\square$}] The advent of complexity
\end{itemize}
\item[{$\square$}] John von Neumann and the Cyborg Incursion into Economics
\begin{itemize}
\item[{$\square$}] Economics at one remove
\item[{$\square$}] Purity
\item[{$\square$}] Impurity
\item[{$\square$}] Wordliness
\end{itemize}
\item[{$\square$}] The Military, the Scientist, and the Revised Rules of the Game
\begin{itemize}
\item[{$\square$}] What did you do in the war, daddy?
\item[{$\square$}] The cybord character of science mobilization in the WWII
\item[{$\square$}] Operations Research
\item[{$\square$}] The Ballad of Hotelling and Schultz
\item[{$\square$}] SRG, RAND, Rad Lab
\end{itemize}
\item[{$\square$}] Do Cyborgs Dream of Efficient Markets?
\begin{itemize}
\item[{$\square$}] From Red Vienna to Computopia
\item[{$\square$}] The Goals of Cowles, and Red Afterglows
\item[{$\square$}] Every Man His Own Stat Package
\item[{$\square$}] On the Impossibility of a Democratic Computer
\end{itemize}
\item[{$\square$}] The Empire Strikes Back
\begin{itemize}
\item[{$\square$}] Previews of Cunning Abstractions
\item[{$\square$}] Its a World Eat World Dog: Game Theory at RAND
\item[{$\square$}] The High Cost of Information in Postwar Neoclassical Theory
\item[{$\square$}] Rigor Mortis in the First Casualty of War
\item[{$\square$}] Does the Rational Agent Compute?
\end{itemize}
\item[{$\square$}] Core Wars
\begin{itemize}
\item[{$\square$}] Inhuman, All Too Inhuman
\item[{$\square$}] Herbert Simon: Simulacra vs Automata
\item[{$\square$}] Showdown at the OR Corral
\item[{$\square$}] Send in the Clones
\end{itemize}
\item[{$\square$}] Machines Who Think vs Machines that Sell
\begin{itemize}
\item[{$\square$}] Where is the Computer Taking Us?
\item[{$\square$}] Five Alternative Scenarios for the Future of Computational
Economics
\item[{$\square$}] They Hayek Hypothesis and Experimental Economics
\item[{$\square$}] Gode and Sunder Go Roboshoppin
\item[{$\square$}] Contingency, Irony, and Computation
\end{itemize}
\end{itemize}
\item[{$\square$}] Mirowski, More Heat than Light
\item[{$\square$}] Mirowski, Against Mechanism
\item\relax [0/17] Mirowski, The Knowledge We Lost in Information
\begin{itemize}
\item[{$\square$}] It's not Rational
\item[{$\square$}] The Standard Narrative and the Bigger Picture
\item[{$\square$}] Natural Science Inspirations
\item[{$\square$}] The Nobels and the Neoliberals
\item[{$\square$}] The Socialist Calculation Controversy as the Starting Point of the
Economics of Information
\item[{$\square$}] Hayek Changes his Mind
\item[{$\square$}] The Neoclassical Economics of Information Was Incubated at Cowles
\item[{$\square$}] Three Different Modalities of Information in Neoclassical Theory
\item[{$\square$}] Going the Market One Better
\item[{$\square$}] The History of Markets and the Theory of Market Design
\item[{$\square$}] The Walrasian School of Design
\item[{$\square$}] The Bayes-Nash School of Design
\item[{$\square$}] The Experimentalist School of Design
\item[{$\square$}] Hayek and the Schools of Design
\item[{$\square$}] Designs on the Market: The FCC Spectrum Auctions
\item[{$\square$}] Private Intellectuals and Public Perplexity : The TARP
\item[{$\square$}] Artificial Ignorance
\end{itemize}
\item\relax [2/4] Backhouse, New Directions in Economic Methodology
\begin{itemize}
\item[{$\square$}] McCloskey, How to Do a Rhetorical Analysis, and Why
\item[{$\square$}] Lawson, A Realist Theory for Economics
\item[{$\boxtimes$}] Mirowski, What are the Questions?
\item[{$\boxtimes$}] Henderson, Metaphor and Economics
\end{itemize}
\item[{$\square$}] Backhouse, The unsocial social science: Economics and Neighboring Disciplines Since 1945
\item\relax [0/3] Backhouse, They History of the Social Sciences since 1945
\begin{itemize}
\item[{$\square$}] Ash, Psychology
\item[{$\square$}] Backhouse, Economics
\item[{$\square$}] Bevir, Political Science
\end{itemize}
\item[{$\square$}] Gigerenzer, Mind as Computer: Birth of a Metaphor
\item[{$\square$}] Marshall, Minds, Machines and Metaphors
\item[{$\square$}] Vicedo, Cold War emotions: The war over human nature
\item[{$\square$}] Dupuy, Aux origines des sciences cognitives
\item[{$\square$}] Chomsky, The Cold War \& the University: Toward an Intellectual History of the Postwar Years
\item[{$\square$}] Mikulark, ``Cybernetics and Marxism-Leninism'' in The Social Impact of Cybernetics, ed. Charles Dechert
\item[{$\square$}] Israel,  Meccanicismo
\item[{$\square$}] Israel, La machina vivente: contre le visione meccanicistiche del uomo
\item[{$\square$}] Edwards, 1996, The Closed World: Computers and the Politics of Discourse in Cold War America
\item\relax [0/9] Amadae, Rationalizing Capitalist Democracy: The Cold War Origins of
Rational Choice Liberalism
\begin{itemize}
\item[{$\square$}] Managing the National Securtity State: Decision Technologies and Policy Science
\item[{$\square$}] Arrow's Social Choice and Individual Values
\item[{$\square$}] Buchanan and Tullocks' Public Choice Theory
\item[{$\square$}] Riker's Positive Political Theory
\item[{$\square$}] Rational Choice and Capitalist Democracy
\item[{$\square$}] Adam Smith's System of Natural Liberty
\item[{$\square$}] Rational Mechanics, Marginalist Economics, and Rational Choice
\item[{$\square$}] Consolidating Rational Choice Liberalism 1970-2000
\item[{$\square$}] From the Panopticon to the Prisoner's Dilemma
\end{itemize}
\end{itemize}
\section{[0/23] Boundary text}
\label{sec:org3e06083}
\begin{itemize}
\item\relax [1/11] Solovey, Cold War Social Science: Knowledge Production, Liberal
Democracy, and Human Nature
\begin{itemize}
\item[{$\square$}] Solovey, Cold War Social Science: Specter, Reality, or Useful Concept?
\item[{$\square$}] Tolon, Futures Studies: A New Social Science Rooted in Cold War Strategic Thinking
\item[{$\square$}] Martin-Nilsen, “It Was All Connected”: Computers and Linguistics in Early Cold War America
\item[{$\square$}] Isaac, Epistemic Design: Theory and Data in Harvard’s Department of Social Relations
\item[{$\boxtimes$}] Heyck, Producing Reason
\item[{$\square$}] Cravens, Column Right, March! Nationalism, Scientific Positivism, and the Conservative Turn of the American Social Sciences in the Cold War Era
\item[{$\square$}] Brick, Neo- Evolutionist Anthropology, the Cold War, and the Beginnings of the World Turn in U.S. Scholarship
\item[{$\square$}] Jones-Imhotep, Maintaining Humans
\item[{$\square$}] Bycroft, Psychology, Psychologists, and the Creativity Movement: The Lives of Method Inside and Outside the Cold War
\item[{$\square$}] Weidman, An Anthropologist on TV: Ashley Montagu and the Biological
Basis of Human Nature, 1945–1960
\item[{$\square$}] Vicedo, Cold War Emotions: Mother Love and the War over Human Nature
\end{itemize}
\item\relax [0/5] Jamie, The Open Mind: Cold War Politics and the Sciences of Human
Nature
\begin{itemize}
\item[{$\square$}] Democratic Minds for a Complex Society
\item[{$\square$}] Scientists as the Model of Human Nature
\item[{$\square$}] Insituting Cognitive Science
\item[{$\square$}] Cognitive Theory and the Making of Liberal Americans
\item[{$\square$}] A Fractured Politics of Human Nature
\end{itemize}
\item[{$\square$}] Miller, 1955, Toward a general theory for the behavioral sciences
\item\relax [0/15] Supiot, 2012, La gouvernance par les nombres
\begin{itemize}
\item[{$\square$}] En quête de la machine à gouverner
\begin{itemize}
\item Poétique du gouvernement
\item L'homme machine
\item Du gouvernement à la gouvernance
\end{itemize}
\item[{$\square$}] Les aventures d'un idéal: le règne de la loi
\begin{itemize}
\item Le \emph{nomos} grec
\item La \emph{lex} en droit romain
\item La révolution gregorienne
\item \emph{Common Law} et droit continental
\item La tradition juridique occidental
\end{itemize}
\item[{$\square$}] Autres points de vue sur les lois
\item[{$\square$}] Le rêve de l'harmonie par le calcul
\begin{itemize}
\item Les accords parfaits du nombre
\item La fonction instituante de la discorde
\end{itemize}
\item[{$\square$}] L'essor des usages normatifs de la quantification
\begin{itemize}
\item Rendre compte
\item Administrer
\item Juger
\item Légiférer
\end{itemize}
\item[{$\square$}] L’asservissement de la Loi au Nombre: du Gosplan au Marché total
\begin{itemize}
\item Le renversement du règne de la loi
\item Le droit, outil de planification
\item L'hybridation du communisme et capitalisme
\end{itemize}
\item[{$\square$}] Calculer l'incalculable: la doctrine Law and Economics
\begin{itemize}
\item La théorie des jeux
\item La théorie de l'agence
\item Le théorème de Coase et la théorie des \emph{property rights}
\item La \emph{New Comparative Analysis} et le marché du droit
\end{itemize}
\item[{$\square$}] La dynamique juridique de la gouvernance par les nombres
\begin{itemize}
\item La gouvernance individuelle
\item La gouvernance de l’entreprise
\item La gouvernance étatique
\item La gouvernance européenne
\item La gouvernance mondiale
\end{itemize}
\item[{$\square$}] Les impasses de la gouvernance par les nombres
\begin{itemize}
\item Les effets de structure de la gouvernance par les nombres
\item Les résistances du Droit à la gouvernance par les nombres
\end{itemize}
\item[{$\square$}] Le dépérissement de l'état
\begin{itemize}
\item La sacralité de la chose publique
\item La direction scientifique des hommes
\item L'inversion de la hiérarchie publique/privé
\item La loi pour soi et soi pour la loi
\item Sans foi ni loi: la société insoutenable
\end{itemize}
\item[{$\square$}] La résurgence du gouvernement par les hommes
\item[{$\square$}] De la mobilisation totale à la crise du Fordisme
\begin{itemize}
\item Le compromis Fordiste
\item La déconstruction du droit du travail
\item Les voies d'un nouveau compromis
\end{itemize}
\item[{$\square$}] De l'échange quantifié à l’allégeance des personnes
\begin{itemize}
\item La mobilisation totale au travail
\item Les nouveaux droits attachés à la personne
\end{itemize}
\item[{$\square$}] La structure des liens d’allégeance
\begin{itemize}
\item L'allégeance dans les réseaux d'entreprises
\item L'allégeance des multinationales aux États impériaux
\end{itemize}
\item[{$\square$}] Comment en sortir
\end{itemize}
\item[{$\square$}] Hughes, 1958, Consciousness and Society: The Reorientation of European Social Thought, 1890-1930
\item[{$\square$}] Ross, 1994, Modernist Impulses in the Human Sciences, 1870-1930
\item[{$\square$}] Purcell, 1973, The Crisis of Democratic Theory: Scientific Naturalism and the Problem of Value
\item[{$\square$}] Butsch, 2008, The Citizen Audience: Crowds, Publics, and Individuals
\item[{$\square$}] Deutsch, 1963, The Nerves of Government: Models of Political Communication and Control
\item[{$\square$}] Cohen-Cole, 2009, The Creative American: Cold War salons, social science, and the cure for modern society.
\item\relax [0/15] Heyck, Herbert Simon: The Bounds of Reason in Modern America
\begin{itemize}
\item[{$\square$}] Unbounded rationality
\item[{$\square$}] The garden of forking paths
\item[{$\square$}] The Chicago school and the sciences of control
\item[{$\square$}] Mathematics, logic, and the sciences of choice
\item[{$\square$}] Research and reform
\item[{$\square$}] \emph{Homo administrativus}, or Choice under control
\item[{$\square$}] Decisions and revisions
\item[{$\square$}] Structuring his environment
\item[{$\square$}] Islands of theory
\item[{$\square$}] A new model of mind and machine
\item[{$\square$}] The program \emph{is} the theory
\item[{$\square$}] The cognitive revolution
\item[{$\square$}] \emph{Homo adaptativus}, the Finite problem solver
\item[{$\square$}] Scientist of the artificial
\item[{$\square$}] The expert problem solver
\end{itemize}
\item\relax [0/8] Heyck, Age of System: Understanding the development of modern social science
\begin{itemize}
\item[{$\square$}] The Organizational Revolution and the Human Sciences
\item[{$\square$}] High modern social science: A bird's eye view
\item[{$\square$}] Patrons of the revolution: Ideas, Ideals, and Institutions in Postwar Social Science
\item[{$\square$}] The magical year 1956, plus or minus one
\item[{$\square$}] Producing reason
\item[{$\square$}] Modernity and social change in American social science
\item[{$\square$}] A model science?
\item[{$\square$}] History and Legacy, Tree and the Web
\end{itemize}
\item[{$\square$}] Heyck, Mind and Network
\item[{$\square$}] Heyck, Georges Miller, language, and the computer metaphor of mind
\item[{$\square$}] Heyck, Defining the Computer: Herbert Simon and the Bureaucratic Mind, Part 1
\item[{$\square$}] Heyck Defining the Computer: Herbert Simon and the Bureaucratic Mind, Part 2
\item\relax [0/9] Ronald Kline, The Cybernetics Moment
\begin{itemize}
\item[{$\square$}] War and Information Theory
\item[{$\square$}] Circular Causality
\item[{$\square$}] The Cybernetics Craze
\item[{$\square$}] The Information Bandwagon
\item[{$\square$}] Machines as Humans
\item[{$\square$}] Humans as Machines
\item[{$\square$}] Cybernetics in Crisis
\item[{$\square$}] Inventing an Information Age
\item[{$\square$}] Two Cybernetic Frontiers
\end{itemize}
\item\relax [0/1] Koyré, Études d'histoire de la pensée philosophique
\begin{itemize}
\item[{$\square$}] Les philosophes et la machine
\begin{itemize}
\item L'appreciation du machinisme
\item Les origines du machinisme
\end{itemize}
\end{itemize}
\item\relax [0/5] Maas, William Stanley Jevons and the Making of Modern Economics
\begin{itemize}
\item[{$\square$}] The Prying Eyes of the Natural Scientist
\item[{$\square$}] Engines of Discovery
\begin{itemize}
\item Babbage and his calculating engines
\item God is a programmer
\item An intelligent machine
\item Is the mind a reasoning machine?
\end{itemize}
\item[{$\square$}] The Machinery of the Mind
\begin{itemize}
\item The Logical Abacus
\item The Logical Machine
\item The machine of the mind
\item Induction - the inverse of deduction
\item To decide what things are similar
\end{itemize}
\item[{$\square$}] The Laws of Human Enjoyment
\begin{itemize}
\item The factory system and the division of labor
\item Ruskin's aesthetic-driven criticism of the factory system
\item Mill and the gospel of work
\item Work and fatigue
\end{itemize}
\item[{$\square$}] The Image of Economics
\begin{itemize}
\item Bridging the natural and the social
\item Mechanical dreams
\item Economics as natural science
\end{itemize}
\end{itemize}
\item\relax [0/8] Mirowski, Machine Dreams
\begin{itemize}
\item[{$\square$}] Cyborg Agonists
\begin{itemize}
\item[{$\square$}] Rooms with a view
\item[{$\square$}] Where the cyborgs are
\item[{$\square$}] The natural sciences and the history of economics
\item[{$\square$}] Anatomy of a cyborg
\item[{$\square$}] Attack of the cyborgs
\item[{$\square$}] The new automaton theatre
\end{itemize}
\item[{$\square$}] Some Cyborg Genealogies; or How the Demon Got Its Bots
\begin{itemize}
\item[{$\square$}] The little engines that could've
\item[{$\square$}] Adventures of a red-hot demon
\item[{$\square$}] Cybernetics
\item[{$\square$}] The devil that made us do it
\item[{$\square$}] The advent of complexity
\end{itemize}
\item[{$\square$}] John von Neumann and the Cyborg Incursion into Economics
\begin{itemize}
\item[{$\square$}] Economics at one remove
\item[{$\square$}] Purity
\item[{$\square$}] Impurity
\item[{$\square$}] Wordliness
\end{itemize}
\item[{$\square$}] The Military, the Scientist, and the Revised Rules of the Game
\begin{itemize}
\item[{$\square$}] What did you do in the war, daddy?
\item[{$\square$}] The cybord character of science mobilization in the WWII
\item[{$\square$}] Operations Research
\item[{$\square$}] The Ballad of Hotelling and Schultz
\item[{$\square$}] SRG, RAND, Rad Lab
\end{itemize}
\item[{$\square$}] Do Cyborgs Dream of Efficient Markets?
\begin{itemize}
\item[{$\square$}] From Red Vienna to Computopia
\item[{$\square$}] The Goals of Cowles, and Red Afterglows
\item[{$\square$}] Every Man His Own Stat Package
\item[{$\square$}] On the Impossibility of a Democratic Computer
\end{itemize}
\item[{$\square$}] The Empire Strikes Back
\begin{itemize}
\item[{$\square$}] Previews of Cunning Abstractions
\item[{$\square$}] Its a World Eat World Dog: Game Theory at RAND
\item[{$\square$}] The High Cost of Information in Postwar Neoclassical Theory
\item[{$\square$}] Rigor Mortis in the First Casualty of War
\item[{$\square$}] Does the Rational Agent Compute?
\end{itemize}
\item[{$\square$}] Core Wars
\begin{itemize}
\item[{$\square$}] Inhuman, All Too Inhuman
\item[{$\square$}] Herbert Simon: Simulacra vs Automata
\item[{$\square$}] Showdown at the OR Corral
\item[{$\square$}] Send in the Clones
\end{itemize}
\item[{$\square$}] Machines Who Think vs Machines that Sell
\begin{itemize}
\item[{$\square$}] Where is the Computer Taking Us?
\item[{$\square$}] Five Alternative Scenarios for the Future of Computational
Economics
\item[{$\square$}] They Hayek Hypothesis and Experimental Economics
\item[{$\square$}] Gode and Sunder Go Roboshoppin
\item[{$\square$}] Contingency, Irony, and Computation
\end{itemize}
\end{itemize}
\item[{$\square$}] Mirowski, More Heat than Light
\item[{$\square$}] Mirowski, Against Mechanism
\item\relax [0/17] Mirowski, The Knowledge We Lost in Information
\begin{itemize}
\item[{$\square$}] It's not Rational
\item[{$\square$}] The Standard Narrative and the Bigger Picture
\item[{$\square$}] Natural Science Inspirations
\item[{$\square$}] The Nobels and the Neoliberals
\item[{$\square$}] The Socialist Calculation Controversy as the Starting Point of the
Economics of Information
\item[{$\square$}] Hayek Changes his Mind
\item[{$\square$}] The Neoclassical Economics of Information Was Incubated at Cowles
\item[{$\square$}] Three Different Modalities of Information in Neoclassical Theory
\item[{$\square$}] Going the Market One Better
\item[{$\square$}] The History of Markets and the Theory of Market Design
\item[{$\square$}] The Walrasian School of Design
\item[{$\square$}] The Bayes-Nash School of Design
\item[{$\square$}] The Experimentalist School of Design
\item[{$\square$}] Hayek and the Schools of Design
\item[{$\square$}] Designs on the Market: The FCC Spectrum Auctions
\item[{$\square$}] Private Intellectuals and Public Perplexity : The TARP
\item[{$\square$}] Artificial Ignorance
\end{itemize}
\item\relax [2/4] Backhouse, New Directions in Economic Methodology
\begin{itemize}
\item[{$\square$}] McCloskey, How to Do a Rhetorical Analysis, and Why
\item[{$\square$}] Lawson, A Realist Theory for Economics
\item[{$\boxtimes$}] Mirowski, What are the Questions?
\item[{$\boxtimes$}] Henderson, Metaphor and Economics
\end{itemize}
\item[{$\square$}] Backhouse, The unsocial social science: Economics and Neighboring Disciplines Since 1945
\item\relax [0/3] Backhouse, They History of the Social Sciences since 1945
\begin{itemize}
\item[{$\square$}] Ash, Psychology
\item[{$\square$}] Backhouse, Economics
\item[{$\square$}] Bevir, Political Science
\end{itemize}
\item[{$\square$}] Gigerenzer, Mind as Computer: Birth of a Metaphor
\item[{$\square$}] Marshall, Minds, Machines and Metaphors
\item[{$\square$}] Vicedo, Cold War emotions: The war over human nature
\item[{$\square$}] Dupuy, Aux origines des sciences cognitives
\item[{$\square$}] Chomsky, The Cold War \& the University: Toward an Intellectual History of the Postwar Years
\item[{$\square$}] Mikulark, ``Cybernetics and Marxism-Leninism'' in The Social Impact of Cybernetics, ed. Charles Dechert
\item[{$\square$}] Israel,  Meccanicismo
\item[{$\square$}] Israel, La machina vivente: contre le visione meccanicistiche del uomo
\item[{$\square$}] Edwards, 1996, The Closed World: Computers and the Politics of Discourse in Cold War America
\item\relax [0/9] Amadae, Rationalizing Capitalist Democracy: The Cold War Origins of
Rational Choice Liberalism
\begin{itemize}
\item[{$\square$}] Managing the National Securtity State: Decision Technologies and Policy Science
\item[{$\square$}] Arrow's Social Choice and Individual Values
\item[{$\square$}] Buchanan and Tullocks' Public Choice Theory
\item[{$\square$}] Riker's Positive Political Theory
\item[{$\square$}] Rational Choice and Capitalist Democracy
\item[{$\square$}] Adam Smith's System of Natural Liberty
\item[{$\square$}] Rational Mechanics, Marginalist Economics, and Rational Choice
\item[{$\square$}] Consolidating Rational Choice Liberalism 1970-2000
\item[{$\square$}] From the Panopticon to the Prisoner's Dilemma
\end{itemize}
\end{itemize}
\chapter{Draft: Framework}
\label{sec:orgb769190}
\section{Main question}
\label{sec:org85e0780}
This thesis will study the influence of the brain-computer metaphor in
Herbert Simon's concept of rationality. 
\section{Thesis Outline}
\label{sec:org50f90cb}
\subsection{Introduction: What to Make of Metaphors in Economics?}
\label{sec:org3096982}
This section will introduce the topic of metaphors in economics. 
\subsection{The Genealogy of the Brain Computer Metaphor in Simon's Work.}
\label{sec:orgc1abbd5}
\subsection{The Influence of the Brain-Computer Metaphor in Simon's Concept of Rationality.}
\label{sec:org32e1228}
\subsection{Conclusion}
\label{sec:org02a3724}
\section{Scope and interrelations}
\label{sec:org6505528}
\lipsum
\section{Research hypotheses.}
\label{sec:org29387f1}
\lipsum
\section{Methodology}
\label{sec:orgbed2251}
\lipsum
\section{Keywords}
\label{sec:orgcb3d39e}
\lipsum
\section{Debates \& Controversies}
\label{sec:org4193f88}
\lipsum
\section{Axes d'interpretation}
\label{sec:org1028d47}
\lipsum
\section{Bibliography}
\label{sec:org19416e4}
I have separated the bibliography according to the big themes of the
thesis. 

\nocite{*}
\subsection{The Brain}
\label{sec:org88d5a16}
\printbibliography[heading=none,keyword=memoire,keyword=brain]
\subsection{The Cold War}
\label{sec:org647fed7}
\printbibliography[heading=none,keyword=memoire,keyword=cold-war,notkeyword=brain]
\subsection{The Computer}
\label{sec:orgec1b085}
\printbibliography[heading=none,keyword=memoire,keyword=computer,notkeyword=brain,notkeyword=cold-war]
\subsection{Metaphors}
\label{sec:orgeaf1c3b}
\printbibliography[heading=none,keyword=memoire,keyword=metaphors,notkeyword=brain,notkeyword=cold-war,notkeyword=computer,notkeyword=cyborg]
\subsection{Herbert Simon}
\label{sec:orgd58bfd4}
\printbibliography[heading=none,keyword=memoire,keyword=herbert-simon,notkeyword=brain,notkeyword=cold-war,notkeyword=computer,notkeyword=cyborg,notkeyword=metaphors]
\subsection{The Social Sciences}
\label{sec:org905c844}
\printbibliography[heading=none,keyword=memoire,keyword=social-science,notkeyword=brain,notkeyword=computer,notkeyword=cyborg,notkeyword=metaphors,notkeyword=herbert-simon,notkeyword=cold-war]
\end{document}